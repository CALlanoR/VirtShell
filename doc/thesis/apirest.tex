\chapter{API REST}
\label{capapi}

Este capitulo esta destinado a los desarrolladores que deseen interactuar con el VirtShell API, para realizar aprovisionamientos autom'aticos desde cualquier plataforma de desarrollo. El VirtShell API es un API REST que provee accesso a los objetos en el VirtShell Server, esto incluye los hosts, imagenes, archivos, templates, aprovisionadores y usuarios. Por medio del API podr'a crear ambientes, m'aquinas virtuales y contenedores personalizados, realizar configuraciones y administrar los recursos f'isicos y virtuales de manera program'atica. 

\section{VirtShell API conceptos}
VirtShell esta disenado sobre ocho conceptos:

\begin{description}
\item [Hosts] Representa una m'aquina f'isica.
\item [Images] Representa la imagen exacta de un sistema operativo o de un contenedor.
\item [Provisioners] Scripts en shell que permiten configurar uno o mas recursos virtuales.
\item [Instances] Permiten administrar los recursos virtuales de un host f'isico.
\item [Properties] Permite obtener informacion de los recursos virtuales o de los f'isicos.
\item [Packages] Permiten administrar paquetes en los recursos virtuales.
\item [Files] Representan archivos necesarios para instalar en las instancias.
\item [Users] Representan los usuarios que tienen acceso a los recursos virtualizados
\end{description}

\section{Provisioning API data model}
Un recurso es un entidad de datos individual con un 'unico identificador. VirtShell API opera con ocho tipos de recursos:

\subsection{/hosts}
Representa un nodo; un nodo es un contenedor de maquinas virtuales o contenedores.
\\
Un endpoint hosts tiene los siguientes metodos http:

\begin{center}
 \begin{tabular}{| l | l | l | l |}
 \hline
  \rowcolor{blueapi}
  \textbf{Acci'on} & \textbf{Metodo HTTP} & \textbf{Solicitud HTTP} & \textbf{Descripci'on} \\ [0.5ex] 
  \hline\hline
  get & GET & /hosts/id & Gets one host by ID. \\
  \hline
  list & GET & /hosts & Retrieves the list of hosts. \\
  \hline  
  create & POST & /hosts/ & Inserts a new host configuration. \\
  \hline
  delete & DELETE & /hosts/id & Deletes an existing host. \\
  \hline  
  update & PUT & /hosts/id & Updates an existing host. \\ [1ex] 
  \hline
\end{tabular}
\end{center}

Representaci'on del recurso de un host:

\medskip
\begin{lstlisting}[style=json]
{
  "uuid": string,
  "name": string,
  "os": string,
  "memory": string,
  "capacity": string,
  "enabled": string,
  "type":string,
  "local_ipv4": string,
  "local_ipv6": string,
  "public_ipv4": string,
  "public_ipv6": string,
  "instances": [ instance_resource],
  "created":["at": number, "by": number]
}
\end{lstlisting}

Ejemplo:

\medskip
\begin{lstlisting}[style=json]
{
  "uuid": "ab8076c0-db91-11e2-82ce-0002a5d5c51b",
  "name": "host-01-pdn",
  "os": "Ubuntu_12.04_3.5.0-23.x86_64",
  "memory": "16GB",
  "capacity": "120GB",
  "enabled": "true|false",
  "type":"StorageOptimized|GeneralPurpose|HighPerformance",
  "local_ipv4": "15.54.88.19",
  "local_ipv6": "ff06:0:0:0:0:0:0:c3",
  "public_ipv4": "10.54.88.19",
  "public_ipv6": "yt06:0:0:0:0:0:0:c3",
  "instances": [
    ... instances resource is here
  ],
  "created":["at":"timestamp", "by":1234]
}
\end{lstlisting}

\subsubsection{Ejemplos de peticiones HTTP}

