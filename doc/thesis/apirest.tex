\chapter{API}
\label{capapi}

\section{Definición de API}
API significa ``Application Programming Interface", y como término, especifica cómo debe interactuar el software.\\
\\
En términos generales, cuando nos referimos a las API de hoy, nos referimos más concretamente a las API web, que son manejadas a través del protocolo de transferencia de hipertexto (HTTP). Para este caso específico, entonces, una API especifica cómo un consumidor puede consumir el servicio que el API expone: cuales URI están disponibles, qué métodos HTTP puede utilizarse con cada URI, que parámetros de consulta se acepta, lo que los datos que pueden ser enviados en el cuerpo de la petición, y lo que el consumidor puede esperar como respuesta.

\subsection{VirtShell API REST}
En el VirtShell API REST un usuario enviá una solicitud al servidor para realizar una acción determinada (como la creación, recuperación, actualización o eliminación de un recurso virtual), y el servidor realiza la acción y enviá una respuesta, a menudo en la forma de una representación del recurso especificado.\\
\\
En el VirtShell API, el usuario especifica una acción con un verbo HTTP como POST, GET, PUT o DELETE. Especificando un recurso por un URI único global de la siguiente forma: \\
\\
https://[host]:[port]/virtshell/api/v1/resourcePath?parameters\\
\\
Debido a que todos los recursos del API tienen una única URI HTTP accesible, REST permite el almacenamiento en cache de datos y esta optimizado para trabajar con una infraestructura distribuida de la web.\\
\\
En esta sección se detalla los recursos y operaciones que puede realizar un usuario del API para realizar aprovisionamientos automáticos desde cualquier plataforma de desarrollo. El VirtShell API provee acceso a los objetos en el VirtShell Server, esto incluye los hosts, imágenes, archivos, templates, aprovisionadores, instancias, grupos y usuarios. Por medio del API podrá crear ambientes, maquinas virtuales y contenedores personalizados, realizar configuraciones y administrar los recursos físicos y virtuales de manera programática. \\

\section{Formato de entrada y salida}
JSON (JavaScript Object Notation) es un formato de datos común, independiente del lenguaje que proporciona una representación de texto simple de estructuras de datos arbitrarias. Para obtener mas información, ver json.org.\\
\\
El VirtShell API solo soporta el formato json para intercambio de información. Cualquier solicitud que no se encuentre en formato json resultara en un error con código 406 (Content Not Acceptable Error).

\section{Codigos de error}
Aqui se presenta una lista de codigos de error que pueden resultar de una petici'on al API en cualquier recurso.

\begin{itemize}
\item \textbf{400 Bad Request} La solicitud no pudo ser procesada con 'exito porque el URI no era v'alido. El cuerpo de la respuesta contendr'a una raz'on del fracaso de la petici'on. Esta respuesta indica error permanente.

\item \textbf{403 Forbidden} La solicitud no pudo ser procesada con 'exito porque la identidad del usuario no tiene acceso suficiente para procesar la solicitud. Esta respuesta indica error permanente.

\item \textbf{406 Content Not Acceptable} Un recurso genera este error de acuerdo al tipo de cabeceras enviadas en la petici'on. Esta respuesta indica un error permanete e indica un formato de salida no soportado. La respuesta de este tipo de error no contiene un contenido debido a la inhabilidad del servidor para generar una respuesta en el formato solicitado.

\item \textbf{404 Not Found} La solicitud no pudo ser procesada con 'exito porque la solicitud no era v'alida. Lo m'as probable es que no se encontró la url. Esta respuesta indica error permanente.

\item \textbf{500 Server Error} La solicitud no pudo ser procesada debido a que el servidor encontr'o una condici'on inesperada que le impidi'o cumplir con la petici'on.

\item \textbf{501 Not Implemented} La solicitud no se pudo completar porque el servidor o bien no reconoce el m'etodo de petici'on o el recurso solicitado no existe.

\end{itemize}

Los errores que no sean de codigo 406 (Content Not Acceptable) contienen una respuesta en formato json, que contiene un breve mensaje explicado el error con m'as detalle. Por ejemplo, una consulta POST /virtshell/api/v1/hosts, con un cuerpo vacio, dar'ia lugar a la siguiente respuesta:

\vspace{1cm}
\begin{lstlisting}[style=json]
HTTP/1.1 400 Bad Request
Content-Type: application/json

{"error": "Missing input for create instance"}
\end{lstlisting}

\section{API Resources}

\subsection{Groups}
Representan los grupos registrados en VirtShell. Los metodos soportados son:

\begin{center}
 \begin{tabular}{| l | l | l | l |}
 \hline
  \rowcolor{blueapi}
  \textbf{Acci'on} & \textbf{Metodo HTTP} & \textbf{Solicitud HTTP} & \textbf{Descripci'on} \\ [0.5ex] 
  \hline\hline
  get & GET & /users/id & Gets one group by ID. \\
  \hline
  list & GET & /hosts & Retrieves the list of groups. \\  
  \hline
  create & POST & /users/ & creates a new group. \\
  \hline
  delete & DELETE & /users/id & Deletes an existing group. \\
  \hline
\end{tabular}
\end{center}

\vspace{1cm}
Representaci'on del recurso de un grupo:
\vspace{1cm}

\begin{lstlisting}[style=json]
{
  "uuid": "ab8076c0-db91-11e2-82ce-0002a5d5c51b",
  "name": "web_development_team",
  "users": [ ... list of members of the group ...],  
  "created":[ {"at":"timestamp"}, {"by":user_id}]
}
\end{lstlisting}

Ejemplo:

\medskip
\begin{lstlisting}[style=json]
{
  "uuid": "ab8076c0-db91-11e2-82ce-0002a5d5c51b",
  "name": "web_development_team",
  "users": [ 
      {"username": "user1", "id": "a146cae4-8c90-11e5-8994-feff819cdc9f"},
      {"username": "user2", "id": "a146d00c-8c90-11e5-8994-feff819cdc9f"}
  ]
  "created":[{"at":"1447696674"}, {"by":"a379e8e6-8c8b-11e5-8994-feff819cdc9f"}]
}
\end{lstlisting}

\subsubsection{Ejemplos de peticiones HTTP}

\paragraph{Crear un nuevo grupo - POST /virtshell/api/v1/grupos} ~\\

\begin{lstlisting}[style=json]
curl -X POST \
  -H 'accept: application/json' \
  -H 'X-VirtShell-Authorization: UserId:Signature' \
  -H "Content-Type: multipart/form-data" \
  -d '{"name": "database_team"}' \
  'http://<host>:<port>/api/virtshell/v1/groups'
\end{lstlisting}

\vspace{1cm}
Respuesta:
\vspace{1cm}

\begin{lstlisting}[style=json]
HTTP/1.1 200 OK
Content-Type: application/json
{ "create": "success" }
\end{lstlisting}

\paragraph{Obtener un grupo - GET /virtshell/api/v1/groups/:id} ~\\

\begin{lstlisting}[style=json]
curl -sv -H 'accept: application/json' 
     -H 'X-VirtShell-Authorization: UserId:Signature' \ 
     'http://<host>:<port>/api/virtshell/v1/groups/?id=ab8076c0-db91-11e2-82ce-0002a5d5c51b'
\end{lstlisting}

\vspace{1cm}
Respuesta:
\vspace{1cm}

\begin{lstlisting}[style=json]
HTTP/1.1 200 OK
Content-Type: application/json
{
  "uuid": "ab8076c0-db91-11e2-82ce-0002a5d5c51b",
  "name": "web_development_team",
  "users": [ 
      {"username": "user1", "id": "a146cae4-8c90-11e5-8994-feff819cdc9f"},
      {"username": "user2", "id": "a146d00c-8c90-11e5-8994-feff819cdc9f"}
  ]
  "created":[{"at":"1447696674"}, {"by":"a379e8e6-8c8b-11e5-8994-feff819cdc9f"}]
}
\end{lstlisting}

\paragraph{Obtener todos los grupos - GET /virtshell/api/v1/groups} ~\\

\begin{lstlisting}[style=json]
curl -sv -H 'accept: application/json' 
     -H 'X-VirtShell-Authorization: UserId:Signature' \ 
     'http://localhost:8080/api/virtshell/v1/groups'
\end{lstlisting}

\vspace{1cm}
Respuesta:
\vspace{1cm}

\begin{lstlisting}[style=json]
HTTP/1.1 200 OK
Content-Type: application/json
{
  "groups": [
    {
      "uuid": "ab8076c0-db91-11e2-82ce-0002a5d5c51b",
      "name": "web_development_team",
      "users": [ 
          {"username": "user1", "id": "a146cae4-8c90-11e5-8994-feff819cdc9f"},
          {"username": "user2", "id": "a146d00c-8c90-11e5-8994-feff819cdc9f"}
      ],     
      "created":[{"at":"1447696833"}, {"by":"d2372efa-8c8b-11e5-8994-feff819cdc9f"}]
    },
    {
      "uuid": "a379f19c-8c8b-11e5-8994-feff819cdc9f",
      "name": "math_team",
      "users": [ 
          {"username": "user3", "id": "a146cae4-8c90-11e5-8994-feff819cdc9f"}
      ],     
      "created":[{"at":"1421431233"}, {"by":"18489280-8c91-11e5-8994-feff819cdc9f"}]
    },
    {
      "uuid": "a379f3d6-8c8b-11e5-8994-feff819cdc9f",
      "name": "chemical_team",
      "users": [ 
          {"username": "user4", "id": "F8489280-8c91-11e5-8994-feff819cdc9f"},
          {"username": "user5", "id": "18489780-8c91-11e5-8994-feff819cdc9f"}
      ],       
      "created":[{"at":"1424109633"}, {"by":"d2373576-8c8b-11e5-8994-feff819cdc9f"}]
    },        
}  
\end{lstlisting}

\paragraph{Eliminar un grupo - DELETE /virtshell/api/v1/groups/:id} ~\\

Para eliminar un grupo se debe tener en cuenta que no debe tener usuarios asociados a el.

\begin{lstlisting}[style=json]
curl -sv -X DELETE \
   -H 'accept: application/json' \
   -H 'X-VirtShell-Authorization: UserId:Signature' \
   'http://localhost:8080/api/virtshell/v1/groups?id=73cff0b0-8c8e-11e5-8994-feff819cdc9f'
\end{lstlisting}

\vspace{1cm}
Respuesta:
\vspace{1cm}

\begin{lstlisting}[style=json]
HTTP/1.1 200 OK
Content-Type: application/json
```
```json
{ "delete": "success" }
\end{lstlisting}

\subsection{Users}
Representan los usuarios registrados en VirtShell. Los metodos soportados son:

\begin{center}
 \begin{tabular}{| l | l | l | l |}
 \hline
  \rowcolor{blueapi}
  \textbf{Acci'on} & \textbf{Metodo HTTP} & \textbf{Solicitud HTTP} & \textbf{Descripci'on} \\ [0.5ex] 
  \hline\hline
  get & GET & /users/id & Gets one user by ID. \\
  \hline
  create & POST & /users/ & creates a new user. \\
  \hline
  list & GET & /users & Retrieves the list of users. \\  
  \hline
  delete & DELETE & /users/id & Deletes an existing user. \\
  \hline  
  update & PUT & /users/id & Updates an existing user. \\ [1ex]  
  \hline
\end{tabular}
\end{center}

\vspace{1cm}
Representaci'on del recurso de un usuario:
\vspace{1cm}

\begin{lstlisting}[style=json]
{
  "uuid": "ab8076c0-db91-11e2-82ce-0002a5d5c51b",
  "username": "virtshell",
  "type": "system/regular",
  "login": "user@mail.com",
  "groups": [ ... list of users ...],
  "created": {"at": timestamp, "by": user_uuid},
  "modified": {"at": timestamp, "by": user_uuid}
}
\end{lstlisting}

Ejemplo:

\medskip
\begin{lstlisting}[style=json]
{
  "uuid": "ab8076c0-db91-11e2-82ce-0002a5d5c51b",
  "username": "virtshell",
  "type": "system/regular",
  "login": "user@mail.com",
  "groups": [ {"uuid": "a146cae4-8c90-11e5-8994-feff819cdc9f"},
              {"uuid": "a146d00c-8c90-11e5-8994-feff819cdc9f"}
  ],
  "created": {"at":"1429207233", "by":"92d30f0c-8c9c-11e5-8994-feff819cdc9f"},
  "modified": {"at":"1529207233", "by":"92d31132-8c9c-11e5-8994-feff819cdc9f"}
}
\end{lstlisting}

\subsubsection{Ejemplos de peticiones HTTP}

\paragraph{Crear un nuevo usuario - POST /api/virtshell/v1/users} ~\\

\begin{lstlisting}[style=json]
curl -X POST \
  -H 'accept: application/json' \
  -H 'X-VirtShell-Authorization: UserId:Signature' \
  -H "Content-Type: multipart/form-data" \
  -d {"uuid": "ab8076c0-db91-11e2-82ce-0002a5d5c51b",
       "username": "virtshell", 
       "type": "system/regular",
       "login": "user@mail.com",
       "groups": [ {"uuid": "a146cae4-8c90-11e5-8994-feff819cdc9f"},
                   {"uuid": "a146d00c-8c90-11e5-8994-feff819cdc9f"}
        ],
       "created": {"at":"1429207233", "by":"92d30f0c-8c9c-11e5-8994-feff819cdc9f"},
       "modified": {"at":"1529207233", "by":"92d31132-8c9c-11e5-8994-feff819cdc9f"}
      } \
  'http://<host>:<port>/api/virtshell/v1/users'
\end{lstlisting}

\vspace{1cm}
Respuesta:
\vspace{1cm}

\begin{lstlisting}[style=json]
HTTP/1.1 200 OK
Content-Type: application/json
{ "create": "success" }
\end{lstlisting}

\paragraph{Obtener un usuario - GET /api/virtshell/v1/users/:id} ~\\

\begin{lstlisting}[style=json]
curl -sv -H 'accept: application/json' 
     -H 'X-VirtShell-Authorization: UserId:Signature' \ 
     'http://<host>:<port>/api/virtshell/v1/users/?id=ab8076c0-db91-11e2-82ce-0002a5d5c51b'
\end{lstlisting}

\vspace{1cm}
Respuesta:
\vspace{1cm}

\begin{lstlisting}[style=json]
HTTP/1.1 200 OK
Content-Type: application/json
{
  "uuid": "ab8076c0-db91-11e2-82ce-0002a5d5c51b",
  "username": "virtshell",
  "type": "system/regular",
  "login": "user@mail.com",
  "groups": [ {"uuid": "a146cae4-8c90-11e5-8994-feff819cdc9f"}],
  "created": {"at":"1429207233", "by":"92d30f0c-8c9c-11e5-8994-feff819cdc9f"},
  "modified": {"at":"1529207233", "by":"92d31132-8c9c-11e5-8994-feff819cdc9f"}
}
\end{lstlisting}

\paragraph{Actualizar un usuario - PUT /api/virtshell/v1/users/:id} ~\\

\begin{lstlisting}[style=json]
curl -sv -X PUT \
  -H 'accept: application/json' \
  -H 'X-VirtShell-Authorization: UserId:Signature' \
  -H "Content-Type: multipart/form-data" \
  -d '{"type": "system",
       "groups": [{"uuid": "a146cae4-8c90-11e5-8994-feff819cdc9f"},
                  {"uuid": "a146d00c-8c90-11e5-8994-feff819cdc9f"}]}' \
   'http://localhost:8080/api/virtshell/v1/file?id=8de7b824-d7d1-4265-a3a6-5b46cc9b8ed5'
\end{lstlisting}

\vspace{1cm}
Respuesta:
\vspace{1cm}

\begin{lstlisting}[style=json]
HTTP/1.1 200 OK
Content-Type: application/json

{ "update": "success" }
\end{lstlisting}


\paragraph{Eliminar un usuario - DELETE /api/virtshell/v1/users/:id} ~\\

\begin{lstlisting}[style=json]
curl -sv -X DELETE \
   -H 'accept: application/json' \
   -H 'X-VirtShell-Authorization: UserId:Signature' \
   'http://localhost:8080/api/virtshell/v1/fles?id=ab8076c0-db91-11e2-82ce-0002a5d5c51b'
\end{lstlisting}

\vspace{1cm}
Respuesta:
\vspace{1cm}

\begin{lstlisting}[style=json]
HTTP/1.1 200 OK
Content-Type: application/json
```
```json
{ "delete": "success" }
\end{lstlisting}


% VirtShell is a multi-user framework that is based on the Unix permissions concepts to provide security.

% VirtShell provides mechanisms to control access by  limiting the types of
% resource access that can be made. Access is permitted or denied depending on
% several factors, one of which is the type of access requested. Several different
% types of operations may be controlled:

% Read. Read from the resouce.
% Write. Write or rewrite of resoures.
% Execute. Load the resource into host and execute it.

% Here is a quick breakdown of the access that the three basic permission types grant a user.

% Read
% ----
% Read permission allows a user to view the contents of any resource in VirtShell.

% Write
% -----
% Write permission allows a user to create, modify and delete whatever resources.

% Execute
% -------
% Execute permission allows a user to execute virtual machines or containers, for example: start, stop, pause, snapshot. (the user must also have read permission). 

\subsection{Partitions}
Las particones permiten organizar las máquinas que albergaran recursos virtuales en partes aisladas de las demás.Los métodos soportados son:

\begin{center}
 \captionof{table}{Métodos HTTP para partitions}
 \begin{tabular}{| l | l | l | l |}
 \hline
  \rowcolor{blueapi}
  \textbf{Acci'on} & \textbf{Método HTTP} & \textbf{Solicitud HTTP} & \textbf{Descripci'on} \\ [0.5ex] 
  \hline\hline
  get & GET & /partitions/:name & Gets one partition by name. \\
  \hline
  list & GET & /partitions & Retrieves the list of partitions. \\
  \hline  
  create & POST & /partitions/ & Inserts a new partition configuration. \\
  \hline
  delete & DELETE & /partitions/:name & Deletes an existing partition. \\
  \hline  
  update & PUT & /partitions/:name/host/:hostname & Add a host to partition. \\ [1ex] 
  \hline
\end{tabular}
\end{center}

Representaci'on del recurso de una partición:

\medskip
\begin{lstlisting}[style=json]
{
  "uuid": string,
  "name": string,
  "description": string, 
  "hosts": [ ... list of hosts associated with the Partitions ...],
  "created": {"at": number, "by": string},
  "modified": {"at": number, "by": string}
}
\end{lstlisting}

Ejemplo:

\medskip
\begin{lstlisting}[style=json]
{
  "uuid": "ab8076c0-db91-11e2-82ce-0002a5d5c51b",
  "name": "development_co",
  "description": "Collection of servers oriented to development team in Colombia.", 
  "hosts": [ ... list of hosts associated with the partition ...],
  "created": {"at":"1429207233", "by":"92d30f0c-8c9c-11e5-8994-feff819cdc9f"},
  "modified": {"at":"1529207233", "by":"92d31132-8c9c-11e5-8994-feff819cdc9f"}
}
\end{lstlisting}

\subsubsection{Ejemplos de peticiones HTTP}

\paragraph{Crear una nueva partición - POST /api/virtshell/v1/partitions} ~\\

\begin{lstlisting}[style=json]
curl -sv -X POST \
  -H 'accept: application/json' \
  -H 'X-VirtShell-Authorization: UserId:Signature' \
  -d '{
       "name": "development_co",
       "description": "Collection of servers oriented to development team in colombia."
      }' \
   'http://localhost:8080/api/virtshell/v1/partitions'
\end{lstlisting}

Response:

\begin{lstlisting}[style=json]
HTTP/1.1 200 OK
Content-Type: application/json
{ "create": "success" }
\end{lstlisting}

\paragraph{Obtener una partición- GET /api/virtshell/v1/partitions/:name} ~\\

\begin{lstlisting}[style=json]
curl -sv -H 'accept: application/json' 
     -H 'X-VirtShell-Authorization: UserId:Signature' \ 
     'http://<host>:<port>/api/virtshell/v1/partitions/development_co'
\end{lstlisting}

Response:

\begin{lstlisting}[style=json]
HTTP/1.1 200 OK
Content-Type: application/json
{
  "uuid": "efa1777c-cad7-11e5-9956-625662870761",
  "name": "backend_development_04",
  "description": "Servers for backend of the company", 
  "hosts": [ ... list of hosts associated with the section ...],  
  "created": {"at":"1429207233", "by":"1a900cdc-cad8-11e5-9956-625662870761"},
  "modified": {"at":"1529207233", "by":"2163b554-cad8-11e5-9956-625662870761"}
}
\end{lstlisting}

\paragraph{Obtener todas las particiones - GET /api/virtshell/v1/partitions} ~\\

\begin{lstlisting}[style=json]
curl -sv -H 'accept: application/json' 
     -H 'X-VirtShell-Authorization: UserId:Signature' \ 
     'http://localhost:8080/api/virtshell/v1/partitions'
\end{lstlisting}

Response:

\begin{lstlisting}[style=json]
HTTP/1.1 200 OK
Content-Type: application/json
{
  "partitions": [
    {
      "uuid": "ab8076c0-db91-11e2-82ce-0002a5d5c51b",
      "name": "development_co",
      "description": "Collection of servers oriented to development team in colombia.",
      "hosts": [ ... list of hosts associated with the section ...],
      "created": {"at":"1429207233", "by":"92d30f0c-8c9c-11e5-8994-feff819cdc9f"},
      "modified": {"at":"1529207233", "by":"92d31132-8c9c-11e5-8994-feff819cdc9f"}
    },
    { 
      "uuid": "efa1777c-cad7-11e5-9956-625662870761",
      "name": "production_us_miami",
      "description": "Collection of servers oriented to production in us.",
      "hosts": [ ... list of hosts associated with the section ...],      
      "created": {"at":"1429207233", "by":"1a900cdc-cad8-11e5-9956-625662870761"},
      "modified": {"at":"1529207233", "by":"2163b554-cad8-11e5-9956-625662870761"}
    }    
  ]
}  
\end{lstlisting}

\paragraph{Eliminar una partición - DELETE /api/virtshell/v1/partitions/:name} ~\\

\begin{lstlisting}[style=json]
curl -sv -X DELETE \
   -H 'accept: application/json' \
   -H 'X-VirtShell-Authorization: UserId:Signature' \
   'http://<host>:<port>/api/virtshell/v1/partitions/backend_development_04'
\end{lstlisting}

Response:

\begin{lstlisting}[style=json]
HTTP/1.1 200 OK
Content-Type: application/json
```
```json
{ "delete": "success" }
\end{lstlisting}

\paragraph{Agregar un host a una partición - PUT /api/virtshell/v1/partitions/:name/host/:hostname} ~\\

\begin{lstlisting}[style=json]
curl -sv -X PUT \
  -H 'accept: application/json' \
  -H 'X-VirtShell-Authorization: UserId:Signature' \
  'http://localhost:8080/virtshell/api/v1/partitions/:name/host/:hostname'
\end{lstlisting}

Response:

\begin{lstlisting}[style=json]
HTTP/1.1 200 OK
Content-Type: application/json

{ "add_host": "success" }
\end{lstlisting}
\subsection{Enviroments}
Representan subredes de trabajo más pequeñas asociadas a una partición. Los métodos soportados son:

\begin{center}
 \captionof{table}{Métodos HTTP para enviroments}
 \begin{tabular}{| l | l | l | l |}
 \hline
  \rowcolor{blueapi}
  \textbf{Acción} & \textbf{Método HTTP} & \textbf{Solicitud HTTP} & \textbf{Descripción} \\ [0.5ex] 
  \hline\hline
  get & GET & /enviroments/:name & Gets one enviroment by name. \\
  \hline
  list & GET & /enviroments & \pbox{5cm}{\vspace{0.2cm} Retrieves the list of \\ enviroments. \vspace{0.2cm}} \\
  \hline  
  create & POST & /enviroments/ & Inserts a new enviroment. \\
  \hline
  delete & DELETE & /enviroments/:name & Deletes an existing enviroment. \\
  \hline
\end{tabular}
\end{center}

Representaci'on del recurso de un ambiente:

\medskip
\begin{lstlisting}[style=json]
{
  "uuid": string,
  "name": string,
  "description": string, 
  "users": [ user_resource],
  "partition": string,
  "created": {"at": number, "by": string},
  "modified": {"at": number, "by": string}
}
\end{lstlisting}

Ejemplo:

\medskip
\begin{lstlisting}[style=json]
{
  "uuid": "ab8076c0-db91-11e2-82ce-0002a5d5c51b",
  "name": "bigdata_test_01",
  "description": "Collection of servers oriented to big data.", 
  "users": [ ... list of users allowed to use the enviroment ...],
  "partition": "partition associated with the enviroment",
  "created": {"at":"1429207233", "by":"92d30f0c-8c9c-11e5-8994-feff819cdc9f"},
  "modified": {"at":"1529207233", "by":"92d31132-8c9c-11e5-8994-feff819cdc9f"}
}
\end{lstlisting}

\subsubsection{Ejemplos de peticiones HTTP}

\paragraph{Crear un nuevo ambiente - POST /api/virtshell/v1/enviroments} ~\\

\begin{lstlisting}[style=json]
curl -sv -X POST \
  -H 'accept: application/json' \
  -H 'X-VirtShell-Authorization: UserId:Signature' \
  -d '{
       "name": "bigdata_test_01",
       "description": "Collection of servers oriented to big data.", 
       "users": [ ... list of users allowed to use the enviroment ...],
       "partition": "partition associated with the enviroment"
      }' \
   'http://localhost:8080/api/virtshell/v1/enviroments'
\end{lstlisting}

Response:

\begin{lstlisting}[style=json]
HTTP/1.1 200 OK
Content-Type: application/json
{ "create": "success" }
\end{lstlisting}

\paragraph{Obtener un ambiente- GET \\ /api/virtshell/v1/enviroments/:name} ~\\

\begin{lstlisting}[style=json]
curl -sv -H 'accept: application/json' 
     -H 'X-VirtShell-Authorization: UserId:Signature' \ 
     'http://<host>:<port>/api/virtshell/v1/enviroments/backend_development'
\end{lstlisting}

Response:

\begin{lstlisting}[style=json]
HTTP/1.1 200 OK
Content-Type: application/json
{
  "uuid": "efa1777c-cad7-11e5-9956-625662870761",
  "name": "backend_development",
  "description": "All backend of the company", 
  "users": [ ... list of users allowed to use the enviroment ...],
  "partition": "partition associated with the enviroment",
  "created": {"at":"1429207233", "by":"1a900cdc-cad8-11e5-9956-625662870761"},
  "modified": {"at":"1529207233", "by":"2163b554-cad8-11e5-9956-625662870761"}
}
\end{lstlisting}

\paragraph{Obtener todos los ambientes - GET \\ /api/virtshell/v1/enviroments} ~\\

\begin{lstlisting}[style=json]
curl -sv -H 'accept: application/json' 
     -H 'X-VirtShell-Authorization: UserId:Signature' \ 
     'http://localhost:8080/api/virtshell/v1/enviroments'
\end{lstlisting}

Response:

\begin{lstlisting}[style=json]
HTTP/1.1 200 OK
Content-Type: application/json
{
  "enviroments": [
    {
      "uuid": "ab8076c0-db91-11e2-82ce-0002a5d5c51b",
      "name": "bigdata_test_01",
      "description": "Collection of servers oriented to big data.", 
      "users": [ ... list of users allowed to use the enviroment ...],
      "partition": "partition associated with the enviroment",
      "created": {"at":"1429207233", "by":"92d30f0c-8c9c-11e5-8994-feff819cdc9f"},
      "modified": {"at":"1529207233", "by":"92d31132-8c9c-11e5-8994-feff819cdc9f"}
    },
    { 
      "uuid": "efa1777c-cad7-11e5-9956-625662870761",
      "name": "backend_development",
      "description": "All backend of the company", 
      "users": [ ... list of users allowed to use the enviroment ...],
      "partition": "partition associated with the enviroment",      
      "created": {"at":"1429207233", "by":"1a900cdc-cad8-11e5-9956-625662870761"},
      "modified": {"at":"1529207233", "by":"2163b554-cad8-11e5-9956-625662870761"}
    }    
  ]
}   
\end{lstlisting}

\paragraph{Eliminar un ambiente - DELETE \\ /api/virtshell/v1/enviroments/:name} ~\\

\begin{lstlisting}[style=json]
curl -sv -X DELETE \
   -H 'accept: application/json' \
   -H 'X-VirtShell-Authorization: UserId:Signature' \
   'http://<host>:<port>/api/virtshell/v1/enviroments/backend_development'
\end{lstlisting}

Response:

\begin{lstlisting}[style=json]
HTTP/1.1 200 OK
Content-Type: application/json
```
```json
{ "delete": "success" }
\end{lstlisting}
\subsection{Hosts}
Representan las m'aquinas f'sicas; un host es un anfitrion de maquinas virtuales o contenedores. Los metodos soportados son:

\begin{center}
 \begin{tabular}{| l | l | l | l |}
 \hline
  \rowcolor{blueapi}
  \textbf{Acci'on} & \textbf{Metodo HTTP} & \textbf{Solicitud HTTP} & \textbf{Descripci'on} \\ [0.5ex] 
  \hline\hline
  get & GET & /hosts/id & Gets one host by ID. \\
  \hline
  list & GET & /hosts & Retrieves the list of hosts. \\
  \hline  
  create & POST & /hosts/ & Inserts a new host configuration. \\
  \hline
  delete & DELETE & /hosts/id & Deletes an existing host. \\
  \hline  
  update & PUT & /hosts/id & Updates an existing host. \\ [1ex] 
  \hline
\end{tabular}
\end{center}

Representaci'on del recurso de un host:

\medskip
\begin{lstlisting}[style=json]
{
  "uuid": string,
  "name": string,
  "os": string,
  "memory": string,
  "capacity": string,
  "enabled": string,
  "type":string,
  "local_ipv4": string,
  "local_ipv6": string,
  "public_ipv4": string,
  "public_ipv6": string,
  "instances": [ instance_resource],
  "created":["at": number, "by": number]
}
\end{lstlisting}

Ejemplo:

\medskip
\begin{lstlisting}[style=json]
{
  "uuid": "ab8076c0-db91-11e2-82ce-0002a5d5c51b",
  "name": "host-01-pdn",
  "os": "Ubuntu_12.04_3.5.0-23.x86_64",
  "memory": "16GB",
  "capacity": "120GB",
  "enabled": "true|false",
  "type":"StorageOptimized|GeneralPurpose|HighPerformance",
  "local_ipv4": "15.54.88.19",
  "local_ipv6": "ff06:0:0:0:0:0:0:c3",
  "public_ipv4": "10.54.88.19",
  "public_ipv6": "yt06:0:0:0:0:0:0:c3",
  "instances": [
    ... instances resource is here
  ],
  "created":["at":"timestamp", "by":1234]
}
\end{lstlisting}

\subsubsection{Ejemplos de peticiones HTTP}

\paragraph{Crear un nuevo host - POST /virtshell/api/v1/hosts} ~\\

\begin{lstlisting}[style=json]
curl -sv -X POST \
  -H 'accept: application/json' \
    -H 'X-VirtShell-Authorization: UserId:Signature' \
  -d '{"name": "host-01-pdn",
       "os": "Ubuntu_12.04_3.5.0-23.x86_64",
       "memory": "16GB",
       "capacity": "120GB",
       "enabled": "true",
       "type" : "GeneralPurpose",
       "local_ipv4": "15.54.88.19",
         "local_ipv6": "ff06:0:0:0:0:0:0:c3",
       "public_ipv4": "10.54.88.19",
       "public_ipv6": "yt06:0:0:0:0:0:0:c3"}' \
   'http://localhost:8080/virtshell/api/v1/hosts'
\end{lstlisting}

Response:

\begin{lstlisting}[style=json]
HTTP/1.1 200 OK
Content-Type: application/json
{ "create": "success" }
\end{lstlisting}

\paragraph{Obtener un host- GET /virtshell/api/v1/hosts/:id} ~\\

\begin{lstlisting}[style=json]
curl -sv -H 'accept: application/json' 
     -H 'X-VirtShell-Authorization: UserId:Signature' \ 
     'http://localhost:8080/api/virtshell/v1/hosts?id=ab8076c0-db91-11e2-82ce-0002a5d5c51b'
\end{lstlisting}

Response:

\begin{lstlisting}[style=json]
HTTP/1.1 200 OK
Content-Type: application/json
{
  "uuid": "ab8076c0-db91-11e2-82ce-0002a5d5c51b",
  "name": "host-01-pdn",
  "os": "Ubuntu_12.04_3.5.0-23.x86_64",
  "memory": "16GB",
  "capacity": "120GB",
  "enabled": "true",
  "type" : "StorageOptimized",
  "local_ipv4": "15.54.88.19",
  "local_ipv6": "ff06:0:0:0:0:0:0:c3",
  "public_ipv4": "10.54.88.19",
  "public_ipv6": "yt06:0:0:0:0:0:0:c3",
  "instances": [
    {
      "name": "name1",
      "id": "72C05559-0590-4DA6-BE56-28AB36CB669C"
    },
    {
      "name": "name2",
      "id": "17173587-C4E9-4369-9C43-FCBF5E075973"
    }
  ],
  "created":["at":"20130625105211", "by":10]
}
\end{lstlisting}

\paragraph{Obtener todos los host - GET /virtshell/api/v1/hosts} ~\\

\begin{lstlisting}[style=json]
curl -sv -H 'accept: application/json' 
     -H 'X-VirtShell-Authorization: UserId:Signature' \ 
     'http://localhost:8080/api/virtshell/v1/hosts'
\end{lstlisting}

Response:

\begin{lstlisting}[style=json]
HTTP/1.1 200 OK
Content-Type: application/json
{
  "hosts": [
    {
      "uuid": "ab8076c0-db91-11e2-82ce-0002a5d5c51b",
      "name": "host-01-pdn",
      "os": "Ubuntu_12.04_3.5.0-23.x86_64",
      "memory": "16GB",
      "capacity": "120GB",
      "enabled": "true",
      "type" : "StorageOptimized",
      "local_ipv4": "15.54.88.19",
      "local_ipv6": "ff06:0:0:0:0:0:0:c3",
      "public_ipv4": "10.54.88.19",
      "public_ipv6": "yt06:0:0:0:0:0:0:c3",
      "instances": [
        {
          "name": "name1",
          "id": "72C05559-0590-4DA6-BE56-28AB36CB669C"
        },
        {
          "name": "name2",
          "id": "17173587-C4E9-4369-9C43-FCBF5E075973"
        }
      ],
      "created":["at":"20130625105211", "by":10]
    },
    {
      "uuid": "ab8076c0-db91-11e2-82ce-0002a5d5c51b",
      "name": "host-01-pdn",
      "os": "Ubuntu_12.04_3.5.0-23.x86_64",
      "memory": "16GB",
      "capacity": "120GB",
      "enabled": "true",
      "type" : "GeneralPurpose",
      "local_ipv4": "15.54.88.19",
      "local_ipv6": "ff06:0:0:0:0:0:0:c3",
      "public_ipv4": "10.54.88.19",
      "public_ipv6": "yt06:0:0:0:0:0:0:c3",
      "instances": [
        {
          "name": "name3",
          "id": "DE11CC9A-482F-4033-A7F8-503EE449DD0A"
        },
        {
          "name": "name4",
          "id": "17173587-C4E9-4369-9C43-FCBF5E075973"
        },    
      ],
      "created":["at":"20130625105211", "by":10]
    }
  ]
}   
\end{lstlisting}

\paragraph{Actualizar un host - PUT /virtshell/api/v1/hosts/:id} ~\\

\begin{lstlisting}[style=json]
curl -sv -X PUT \
  -H 'accept: application/json' \
    -H 'X-VirtShell-Authorization: UserId:Signature' \
  -d '{"memory": "24GB",
     "capacity": "750GB"}' \
   'http://localhost:8080/api/virtshell/v1/hosts?id=ab8076c0-db91-11e2-82ce-0002a5d5c51b'
\end{lstlisting}

Response:

\begin{lstlisting}[style=json]
HTTP/1.1 200 OK
Content-Type: application/json

{ "update": "success" }
\end{lstlisting}

\paragraph{Eliminar un host - DELETE /virtshell/api/v1/hosts/:id} ~\\

\begin{lstlisting}[style=json]
curl -sv -X DELETE \
   -H 'accept: application/json' \
   -H 'X-VirtShell-Authorization: UserId:Signature' \
   'http://localhost:8080/api/virtshell/v1/hosts?id=ab8076c0-db91-11e2-82ce-0002a5d5c51b'
\end{lstlisting}

Response:

\begin{lstlisting}[style=json]
HTTP/1.1 200 OK
Content-Type: application/json
```
```json
{ "delete": "success" }
\end{lstlisting}
\subsection{Instances}
Representan las instancias de las m'aquinas virtuales o los contenedores. Los métodos soportados son:

\begin{center}
 \captionof{table}{Métodos HTTP para instances}
 \begin{tabular}{| l | l | l | l |}
 \hline
  \rowcolor{blueapi}
  \textbf{Acci'on} & \textbf{Método HTTP} & \textbf{Solicitud HTTP} & \textbf{Descripci'on} \\ [0.5ex] 
  \hline\hline
  get & GET & /provisioners/:name & Gets one provisioner by ID. \\
  \hline
  list & GET & /provisioners & Retrieves the list of provisioners. \\
  \hline  
  create & POST & /provisioners/ & Creates a new provisioner. \\
  \hline
  delete & DELETE & /provisioners/:name & Deletes an existing host. \\ [1ex] 
  \hline
\end{tabular}
\end{center}

Representaci'on del recurso de un provisioner:

\medskip
\begin{lstlisting}[style=json]
{
  "uuid": string,
  "name": string,
  "description": string, 
  "enviroment": string,
  "provisioner": string,
  "host_type": string,
  "ipv4": string,
  "ipv6": string,
  "driver": string,
  "permissions": string,
  "created": {"at": timestamp, "by": string},
  "modified": {"at": timestamp, "by": string}
}
\end{lstlisting}

Ejemplo:

\medskip
\begin{lstlisting}[style=json]
{
  "uuid": "ab8076c0-db91-11e2-82ce-0002a5d5c51b",
  "name": "transactional_log",
  "description": "Server transactional only for store logs", 
  "enviroment": "Enviroment name to which it belongs",
  "provisioner": "all_backend",
  "host_type": "GeneralPurpose | ComputeOptimized | MemoryOptimized | StorageOptimized",
  "ipv4": "172.16.56.104",
  "ipv6": "FE80:0000:0000:0000:0202:B3FF:FE1E:8329",
  "driver": "lxc | virtualbox | vmware | ec2 | kvm | docker",
  "permissions": "xwrxwrxwr",
  "created": {"at":"1429207233", "by":"92d30f0c-8c9c-11e5-8994-feff819cdc9f"},
  "modified": {"at":"1529207233", "by":"92d31132-8c9c-11e5-8994-feff819cdc9f"}
}
\end{lstlisting}

\subsubsection{Ejemplos de peticiones HTTP}

\paragraph{Crear una nueva instance - POST /api/virtshell/v1/instances} ~\\


\begin{lstlisting}[style=json]
curl -sv -X POST \
  -H 'accept: application/json' \
  -H 'X-VirtShell-Authorization: UserId:Signature' \
  -d '{ "name": "transactional_log",
        "enviroment": "development_co",
        "description": "Server transactional only for store logs", 
        "provisioner": "all_backend",
        "host_type": "GeneralPurpose",
        "driver": "lxc"
      }' \
   'http://localhost:8080/virtshell/api/v1/instances'
\end{lstlisting}

Response:

\begin{lstlisting}[style=json]
HTTP/1.1 200 OK
Content-Type: application/json
{ "create": "in progress" }
\end{lstlisting}

\paragraph{Obtener un instance- GET /api/virtshell/v1/instances/:name} ~\\

\begin{lstlisting}[style=json]
curl -sv -H 'accept: application/json' 
     -H 'X-VirtShell-Authorization: UserId:Signature' \ 
     'http://<host>:<port>/api/virtshell/v1/instances/orders_colombia'
\end{lstlisting}

Response:

\begin{lstlisting}[style=json]
HTTP/1.1 200 OK
Content-Type: application/json
{
  "uuid": "ab8076c0-db91-11e2-82ce-0002a5d5c51b",
  "name": "transactional_log",
  "enviroment": "development_co",
  "description": "Server transactional only for store logs", 
  "provisioner": "all_backend",
  "host_type": "GeneralPurpose",
  "drive": "lxc",
  "created": {"at":"1429207233", "by":"92d30f0c-8c9c-11e5-8994-feff819cdc9f"},
  "modified": {"at":"1529207233", "by":"cf744732-8f12-11e5-8994-feff819cdc9f"}
  }
\end{lstlisting}

\paragraph{Obtener todos las instances - GET /api/virtshell/v1/instances} ~\\

\begin{lstlisting}[style=json]
curl -sv -H 'accept: application/json' 
     -H 'X-VirtShell-Authorization: UserId:Signature' \ 
     'http://localhost:8080/api/virtshell/v1/instances'
\end{lstlisting}

Response:

\begin{lstlisting}[style=json]
HTTP/1.1 200 OK
Content-Type: application/json
{
  "instances": [
    {
      "uuid": "ab8076c0-db91-11e2-82ce-0002a5d5c51b",
      "name": "transactional_log",
      "enviroment": "development_co",
      "description": "Server transactional only for store logs", 
      "provisioner": "all_backend",
      "host_type": "GeneralPurpose",
      "drive": "lxc",
      "permissions": "xwrxwrxwr",
      "created": {"at":"1429207233", "by":"92d30f0c-8c9c-11e5-8994-feff819cdc9f"},
      "modified": {"at":"1529207233", "by":"cf744732-8f12-11e5-8994-feff819cdc9f"}
    },
    { 
      "uuid": "cf744476-8f12-11e5-8994-feff819cdc9f",
      "name": "orders_colombia",
      "description": "Server transactional dedicated to receive orders", 
      "enviroment": "development_mx",
      "provisioner": "all_backend",
      "host_type": "StorageOptimized",
      "drive": "docker",
      "permissions": "xwrxwrxwr",
      "created": {"at":"1429207233", "by":"92d30f0c-8c9c-11e5-8994-feff819cdc9f"},
      "modified": {"at":"1529207233", "by":"92d31132-8c9c-11e5-8994-feff819cdc9f"}
    }    
  ]
} 
\end{lstlisting}

\paragraph{Eliminar una instance - DELETE /api/virtshell/v1/instances/:nae} ~\\

\begin{lstlisting}[style=json]
curl -sv -X DELETE \
   -H 'accept: application/json' \
   -H 'X-VirtShell-Authorization: UserId:Signature' \
   'http://<host>:<port>/api/virtshell/v1/instances/orders_colombia'
\end{lstlisting}

Response:

\begin{lstlisting}[style=json]
HTTP/1.1 200 OK
Content-Type: application/json
```
```json
{ "delete": "in progress" }
\end{lstlisting}
\subsection{Tasks}
Representan una tarea en VirtShell. Los métodos soportados son:

\begin{center}
 \captionof{table}{Métodos HTTP para tasks}
 \begin{tabular}{| l | l | l | l |}
 \hline
  \rowcolor{blueapi}
  \textbf{Acci'on} & \textbf{Metodo HTTP} & \textbf{Solicitud HTTP} & \textbf{Descripci'on} \\ [0.5ex] 
  \hline\hline
  get & GET & /tasks/:id & Gets one task by ID. \\
  \hline
  list & GET & /tasks & Retrieves the list of tasks. \\
  \hline
  get & GET & /tasks/status & Gets all task by status name. \\
  \hline 
  create & POST & /tasks/ & Creates a new task \\
  \hline  
  update & PUT & /tasks/:id & Updates an existing task. \\ [1ex] 
  \hline
\end{tabular}
\end{center}

Representaci'on del recurso de un task:

\medskip
\begin{lstlisting}[style=json]
{
  "uuid": string,
  "description": string,
  "status" : string,
  "type": string,
  "object_uuid": string,
  "created":["at":"timestamp", "by":string],
  "last_update": "timestamp",
  "log": string
}
\end{lstlisting}

Ejemplo:

\medskip
\begin{lstlisting}[style=json]
{
  "uuid": "ab8076c0-db91-11e2-82ce-0002a5d5c51b",
  "description": "clone virtual machine database_01",
  "status" : "pending|in progress|sucess|failed",
  "type": "create_instance|delete_instance|restart_instance|...",
  "object_uuid": "uuid of the object (instance, host, property, ...)",
  "created":["at":"timestamp", "by":user_id],
  "last_update": "timestamp",
  "log": "summary of the task"
}
\end{lstlisting}

\subsubsection{Ejemplos de peticiones HTTP}

\paragraph{Crear una nueva tarea - POST /api/virtshell/v1/tasks} ~\\

\begin{lstlisting}[style=json]
curl -sv -X POST \
  -H 'accept: application/json' \
    -H 'X-VirtShell-Authorization: UserId:Signature' \
  -d '{ "description": "clone virtual machine database_01",
        "status" : "in progress"}' \
   'http://localhost:8080/api/virtshell/v1/tasks'
\end{lstlisting}

Response:

\begin{lstlisting}[style=json]
HTTP/1.1 200 OK
Content-Type: application/json
{ "create": "success" }
\end{lstlisting}

\paragraph{Obtener una tarea- GET /api/virtshell/v1/tasks/:id} ~\\

\begin{lstlisting}[style=json]
curl -sv -H 'accept: application/json' 
     -H 'X-VirtShell-Authorization: UserId:Signature' \ 
     'http://<host>:<port>/api/virtshell/v1/tasks/ab8076c0-db91-11e2-82ce-0002a5d5c51b'
\end{lstlisting}

Response:

\begin{lstlisting}[style=json]
HTTP/1.1 200 OK
Content-Type: application/json
{
  "description": "clone virtual machine database_01",
  "status" : "in progress",
  "created": {"at":"1429207233", "by":"92d30f0c-8c9c-11e5-8994-feff819cdc9f"},
  "last_update": "1429207435",
  "log": "summary of the task"
}
\end{lstlisting}

\paragraph{Obtener una tarea de acuerdo a su status- GET /api/virtshell/v1/tasks/:status} ~\\

\begin{lstlisting}[style=json]
curl -sv -H 'accept: application/json' 
     -H 'X-VirtShell-Authorization: UserId:Signature' \ 
     'http://<host>:<port>/api/virtshell/v1/tasks/sucess'
\end{lstlisting}

Response:

\begin{lstlisting}[style=json]
HTTP/1.1 200 OK
Content-Type: application/json
{
  "tasks": [
    {
      "uuid": "a62ad146-ccf4-11e5-9956-625662870761",
      "description": "create container webserver_09",
      "status" : "sucess",
      "created": {"at":"1454433171", "by":"cc7f8e2c-ccf4-11e5-9956-625662870761"},
      "last_update": "1454436771",
      "log": "summary of the task"
    }
  ]
}
\end{lstlisting}

\paragraph{Obtener todas las tareas - GET /api/virtshell/v1/tasks} ~\\

\begin{lstlisting}[style=json]
curl -sv -H 'accept: application/json' 
     -H 'X-VirtShell-Authorization: UserId:Signature' \ 
     'http://<host>:<port>/api/virtshell/v1/tasks/'
\end{lstlisting}

Response:

\begin{lstlisting}[style=json]
HTTP/1.1 200 OK
Content-Type: application/json
{
  "tasks": [
    {
      "uuid": "ab8076c0-db91-11e2-82ce-0002a5d5c51b",
      "description": "clone virtual machine database_01",
      "status" : "in progress",
      "created": {"at":"1429207233", "by":"92d30f0c-8c9c-11e5-8994-feff819cdc9f"},
      "last_update": "1429207435",
      "log": "summary of the task"
    },
    {
      "uuid": "a62ad146-ccf4-11e5-9956-625662870761",
      "description": "create container webserver_09",
      "status" : "sucess",
      "created": {"at":"1454433171", "by":"cc7f8e2c-ccf4-11e5-9956-625662870761"},
      "last_update": "1454436771",
      "log": "summary of the task"
    }
  ]
}  
\end{lstlisting}

\paragraph{Actualizar una tarea - PUT /api/virtshell/v1/tasks/:id} ~\\

\begin{lstlisting}[style=json]
curl -sv -X PUT \
  -H 'accept: application/json' \
    -H 'X-VirtShell-Authorization: UserId:Signature' \
  -d '{"status": "sucess",
     "log": "....."}' \
   'http://localhost:8080/api/virtshell/v1/hosts/a62ad146-ccf4-11e5-9956-625662870761'
\end{lstlisting}

Response:

\begin{lstlisting}[style=json]
HTTP/1.1 200 OK
Content-Type: application/json

{ "update": "success" }
\end{lstlisting}

\subsection{Properties}
Representan propiedades de configuraci'on de las m'aquinas virtuales o contenedores. Los metodos soportados son:

\begin{center}
 \captionof{table}{Métodos HTTP para properties}
 \begin{tabular}{| l | l | l | l |}
 \hline
  \rowcolor{blueapi}
  \textbf{Acci'on} & \textbf{Metodo HTTP} & \textbf{Solicitud HTTP} & \textbf{Descripci'on} \\ [0.5ex] 
  \hline\hline
  get & GET & /properties/ & Install one or more packages. \\ [1ex] 
  \hline
\end{tabular}
\end{center}

\vspace{1cm}
Representaci'on del recurso de un paquete:
\vspace{1cm}

\begin{lstlisting}[style=json]
{
  "properties": [
      {"name": "propertie_name1"},
      {"name": "propertie_name2"}
  ],
  "hosts": [ 
      {"name": "Host_", "range": "[1-3]"}, 
      {"name": "database_001"}
  ],
  "tags": [
    {"name": "db"},
    {"name": "web"}
  ]
}
\end{lstlisting}

Ejemplo:

\medskip
\begin{lstlisting}[style=json]
{
  "properties": [
      {"name": "memory"},
      {"name": "cpu"}
  ],
  "hosts": [ 
      {"name": "Host_", "range": "[1-3]"}
  ]
}
\end{lstlisting}

\subsubsection{Ejemplos de peticiones HTTP}

\paragraph{Obtener una o mas propiedades de una unica instancia - POST /api/virtshell/v1/properties} ~\\

\begin{lstlisting}[style=json]
curl -sv -X GET \
  -H 'accept: application/json' \
  -H "Content-Type: text/plain" \
  -H 'X-VirtShell-Authorization: UserId:Signature' \
  -d '{ "properties": [{"name": "memory"}, {"name": "cpu"}],
        "hosts": [{"name": "WebServer"}]}' \
   'http://localhost:8080/api/virtshell/v1/properties'
\end{lstlisting}

\vspace{1cm}
Respuesta:
\vspace{1cm}

\begin{lstlisting}[style=json]
HTTP/1.1 202 OK
Content-Type: application/json
{
  "id": "kj5436c0-dc94-13tg-82ce-9992b5d5c51b",
  "name": "Database001",
  "memory": 1024
}
\end{lstlisting}

\paragraph{Obtener una o mas propiedades de una o mas instancias por tag - POST /api/virtshell/v1/properties} ~\\

\begin{lstlisting}[style=json]
curl -sv -X GET \
  -H 'accept: application/json' \
  -H "Content-Type: text/plain" \
  -H 'X-VirtShell-Authorization: UserId:Signature' \
  -d '{ "properties": [{"name": "memory"}, {"name": "cpu"}],
        "tag": [{"name": "web"}]}' \
   'http://localhost:8080/api/virtshell/v1/properties'
\end{lstlisting}

\vspace{1cm}
Respuesta:
\vspace{1cm}

\begin{lstlisting}[style=json]
HTTP/1.1 202 OK
Content-Type: application/json
{
  properties: [
    {
     "id": "kj5436c0-dc94-13tg-82ce-9992b5d5c51b",
     "name": "WebServerPhp001",
     "memory": 1024,
     "cpu": 2
    },
    {
     "id": "591b3828-7aaf-4833-a94c-ad0df44d59a4",
     "name": "WebServerPhp002",
     "memory": 1024,
     "cpu": 1  
    }
  ]
}
\end{lstlisting}

\paragraph{Obtener una o mas propiedades de una o mas instancias usando como prefijo un rango - POST /api/virtshell/v1/properties} ~\\

\begin{lstlisting}[style=json]
curl -sv -X GET \
  -H 'accept: application/json' \
  -H "Content-Type: text/plain" \
  -H 'X-VirtShell-Authorization: UserId:Signature' \
  -d '{ "properties": [{"name": "memory"}, {"name": "cpu"}],
        {"name": "Database00", "range": "[1-3]"}]}' \
   'http://localhost:8080/api/virtshell/v1/properties'
\end{lstlisting}

\vspace{1cm}
Respuesta:
\vspace{1cm}

\begin{lstlisting}[style=json]
HTTP/1.1 202 OK
Content-Type: application/json
{
  properties: [
    {
     "id": "kj5436c0-dc94-13tg-82ce-9992b5d5c51b",
     "name": "Database001",
     "memory": 4024,
     "cpu": 2
    },
    {
     "id": "591b3828-7aaf-4833-a94c-ad0df44d59a4",
     "name": "Database002",
     "memory": 4024,
     "cpu": 1  
    },
    {
     "id": "f7c81039-5c88-423b-8b0d-c124483d586b",
     "name": "Database003",
     "memory": 4024,
     "cpu": 3  
    }
  ]  
}
\end{lstlisting}

\subsection{Provisioners}
Representan los scripts que aprovisionan las m'aquinas virtuales o los contenedores. Los métodos soportados son:

\begin{center}
 \captionof{table}{Métodos HTTP para provisioners}
 \begin{tabular}{| l | l | l | l |}
 \hline
  \rowcolor{blueapi}
  \textbf{Acci'on} & \textbf{Metodo HTTP} & \textbf{Solicitud HTTP} & \textbf{Descripci'on} \\ [0.5ex] 
  \hline\hline
  get & GET & /provisioners/:name & Gets one provisioner by ID. \\
  \hline
  list & GET & /provisioners & Retrieves the list of provisioners. \\
  \hline  
  create & POST & /provisioners/ & Creates a new provisioner. \\
  \hline
  delete & DELETE & /provisioners/:name & Deletes an existing provisioner. \\
  \hline  
  update & PUT & /provisioners/:name & Updates an existing provisioner. \\ [1ex] 
  \hline
\end{tabular}
\end{center}

Representaci'on del recurso de un provisioner:

\medskip
\begin{lstlisting}[style=json]
{
  "uuid": string,
  "name": string,
  "description": string,
  "launch": number,
  "memory": number,
  "cpus": number,
  "hdsize": number,
  "image": string,
  "builder": string,
  "executor": string,
  "tag": string,
  "permissions": string,
  "depends": [ ... list of dependencies necessary for the builder ... ],
  "created": {"at":timestamp, "by":string},
  "modified": {"at":timestamp, "by":string}
}

\end{lstlisting}

Ejemplo:

\medskip
\begin{lstlisting}[style=json]
{
  "uuid": "ab8076c0-db91-11e2-82ce-0002a5d5c51b",
  "name": "backend-services-provisioner",
  "description": "Installs/Configures a backend server",
  "launch": 1,
  "memory": 4,
  "cpus": 2,
  "hdsize": 20,
  "image": "ubuntu_server_14.04.2_amd64",
  "builder": "https://github.com/janutechnology/VirtShell_Provisioners_Examples.git",
  "executor": "sh run1.sh",
  "tag": "backend",
  "permissions": "xwrxwrxwr",
  "depends": [ ... list of dependencies necessary for the builder ... ],
  "created": {"at":"1429207233", "by":"92d30f0c-8c9c-11e5-8994-feff819cdc9f"},
  "modified": {"at":"1529207233", "by":"92d31132-8c9c-11e5-8994-feff819cdc9f"}
}
\end{lstlisting}

\subsubsection{Ejemplos de peticiones HTTP}

\paragraph{Crear un nuevo provisioner - POST /api/virtshell/v1/provisioners} ~\\


\begin{lstlisting}[style=json]
curl -sv -X POST \
  -H 'accept: application/json' \
  -H 'X-VirtShell-Authorization: UserId:Signature' \
  -d '{"name": "backend-services-provisioner",
       "launch": 1,
       "memory": 4,
       "cpus": 2,
       "hdsize": 20,
       "image": "ubuntu_server_14.04.2_amd64",
       "driver": "docker",
       "builder": "https://github.com/janutechnology/VirtShell_Provisioners_Examples.git",
       "executor": "sh run1.sh",
       "tag": "backend",
       "permissions": "xwrxwrxwr",
       "depends": [
            {"provisioner_name": "db-users", "version": "2.0.0"},
            {"provisioner_name": "db-transactional"}
        ]
      }' \
   'http://localhost:8080/virtshell/api/v1/provisioners'
\end{lstlisting}

Response:

\begin{lstlisting}[style=json]
HTTP/1.1 200 OK
Content-Type: application/json
{ "create": "success" }
\end{lstlisting}

\paragraph{Obtener un provisioner- GET /api/virtshell/v1/provisioners/:name} ~\\

\begin{lstlisting}[style=json]
curl -sv -H 'accept: application/json' 
     -H 'X-VirtShell-Authorization: UserId:Signature' \ 
     'http://localhost:8080/api/virtshell/v1/provisioners/backend-services-provisioner'
\end{lstlisting}

Response:

\begin{lstlisting}[style=json]
HTTP/1.1 200 OK
Content-Type: application/json
  {
    "name": "backend-services-provisioner",
    "launch": 1,
    "memory": 4,
    "cpus": 2,
    "hdsize": 20,
    "image": "ubuntu_server_14.04.2_amd64",
    "driver": "docker",
    "permissions": "xwrxwrxwr",
    "builder": "https://github.com/janutechnology/VirtShell_Provisioners_Examples.git",
    "executor": "sh run1.sh",
    "tag": "backend",
    "depends": [
        {"provisioner_name": "db-users", "version": "2.0.0"},
        {"provisioner_name": "db-transactional"}
    ],
    "created": {"at":"1429207233", "by":"420aa2c4-8d96-11e5-8994-feff819cdc9f"},
    "modified": {"at":"1529207233", "by":"92d31132-8c9c-11e5-8994-feff819cdc9f"}    
  }
\end{lstlisting}

\paragraph{Obtener todos los provisioners - GET /api/virtshell/v1/provisioners} ~\\

\begin{lstlisting}[style=json]
curl -sv -H 'accept: application/json' 
     -H 'X-VirtShell-Authorization: UserId:Signature' \ 
     'http://localhost:8080/api/virtshell/v1/provisioners'
\end{lstlisting}

Response:

\begin{lstlisting}[style=json]
HTTP/1.1 200 OK
Content-Type: application/json
{
  "provisioners": [
    {
      "name": "backend-services-provisioner",
      "launch": 1,
      "memory": 4,
      "cpus": 2,
      "hdsize": 20,
      "image": "ubuntu_server_14.04.2_amd64",
      "driver": "docker",
      "builder": "https://github.com/janutechnology/VirtShell_Provisioners_Examples.git",
      "executor": "sh run1.sh",
      "tag": "backend",
      "permissions": "xwrxwrxwr",
      "depends": [
          {"provisioner_name": "db-users", "version": "2.0.0"},
          {"provisioner_name": "db-transactional"}
      ]
    },
    {
      "name": "db-transactional",
      "launch": 2,
      "memory": 8,
      "cpus": 2,
      "hdsize": 40,
      "image": "ubuntu_server_14.04.2_amd64",
      "driver": "docker",
      "builder": "https://github.com/janutechnology/VirtShell_Provisioners_Examples.git",
      "executor": "sh run_db.sh",
      "tag": "db",
      "permissions": "xwrxwrxwr"
    }
  ]
}
\end{lstlisting}

\paragraph{Actualizar un provisioner - PUT /api/virtshell/v1/provisioners/:name} ~\\

\begin{lstlisting}[style=json]
curl -sv -X PUT \
  -H 'accept: application/json' \
  -H 'X-VirtShell-Authorization: UserId:Signature' \
  -d '{ "executor": "run_backend.sh" }' \
   'http://localhost:8080/api/virtshell/v1/provisioners/backend-services-provisioner
\end{lstlisting}

Response:

\begin{lstlisting}[style=json]
HTTP/1.1 200 OK
Content-Type: application/json

{ "update": "success" }
\end{lstlisting}

\paragraph{Eliminar un provisioner - DELETE /api/virtshell/v1/provisioners/:name} ~\\

\begin{lstlisting}[style=json]
curl -sv -X DELETE \
   -H 'accept: application/json' \
   -H 'X-VirtShell-Authorization: UserId:Signature' \
   'http://localhost:8080/api/virtshell/v1/provisioners/backend-services-provisioner'
\end{lstlisting}

Response:

\begin{lstlisting}[style=json]
HTTP/1.1 200 OK
Content-Type: application/json
```
```json
{ "delete": "success" }
\end{lstlisting}
\subsection{Images}
Representan imagenes de m'aquinas virtuales o contenedores. Los métodos soportados son:

\begin{center}
 \captionof{table}{Métodos HTTP para images}
 \begin{tabular}{| l | l | l | l |}
 \hline
  \rowcolor{blueapi}
  \textbf{Acción} & \textbf{Método HTTP} & \textbf{Solicitud HTTP} & \textbf{Descripción} \\ [0.5ex] 
  \hline\hline
  get & GET & /images/:name & Gets one image by name. \\
  \hline
  list & GET & /images & Retrieves the list of images. \\
  \hline  
  create & POST & /images/ & Inserts a new image. \\
  \hline
  delete & DELETE & /images/:name & Deletes an existing image. \\ [1ex] 
  \hline
\end{tabular}
\end{center}

\vspace{1cm}
Representación del recurso de una imagen:
\vspace{1cm}

\begin{lstlisting}[style=json]
{
  "id": string,
  "name": string,
  "type": string,
  "os": string,
  "timezone": "America/Bogota", 
  "key": string,
  "preseed_url": url,
  "download_url": url,
  "permissions" : string,
  "created":["at": timestamp,"by": string],
  "details": string
}
\end{lstlisting}

Ejemplo:

\medskip
\begin{lstlisting}[style=json]
{
  "id": "kj5436c0-dc94-13tg-82ce-9992b5d5c51b",
  "name": "ubuntu_server_14.04.2_amd64",
  "type": "iso",
  "os": "ubuntu",
  "timezone": "America/Bogota",
  "preseed_url": "https://<host>:<port>/api/virtshell/v1/files/seeds/seed_ubuntu14-04.txt",
  "download_url": "http://releases.ubuntu.com/raring/ubuntu-14.04-server-amd64.iso",
  "permissions" : "rwxrw----",
  "details": "ubuntu-trusty, version: 14.04.2, amd64-server"
  "created":["at":"20150625105211","by":10]
}
\end{lstlisting}

\subsubsection{Ejemplos de peticiones HTTP}

\paragraph{Crear una nueva imagen - POST /virtshell/api/v1/images} ~\\

\begin{lstlisting}[style=json]
curl -sv -X PUT \
  -H 'accept: application/json' \
  -H "Content-Type: text/plain" \
  -H 'X-VirtShell-Authorization: UserId:Signature' \
  -d '{"name": "ubuntu_server_14.04.2_amd64",
     "type": "iso",
     "os": "ubuntu",
     "timezone": "America/Bogota", 
     "key": "/home/callanor/.ssh/id_rsa.pub",
     "permissions" : "rwxrwxr--",
     "preseed_url": "https://<host>:<port>/api/virtshell/v1/files/seeds/seed_ubuntu14-04.txt",
     "download_url": "http://releases.ubuntu.com/raring/ubuntu-14.04-server-amd64.iso"}' \
   'http://localhost:8080/api/virtshell/v1/image'
\end{lstlisting}

\vspace{1cm}
Respuesta:
\vspace{1cm}

\begin{lstlisting}[style=json]
HTTP/1.1 201 OK
Content-Type: application/json
{ "create": "success" }
\end{lstlisting}

\paragraph{Obtener una imagen - GET /virtshell/api/v1/images/:name} ~\\

\begin{lstlisting}[style=json]
curl -sv -H 'accept: application/json' 
     -H 'X-VirtShell-Authorization: UserId:Signature' \ 
     'http://localhost:8080/api/virtshell/v1/images/ubuntu_server_14.04.2_amd64'
\end{lstlisting}

\vspace{1cm}
Respuesta:
\vspace{1cm}

\begin{lstlisting}[style=json]
HTTP/1.1 200 OK
Content-Type: application/json
{
  "id": "kj5436c0-dc94-13tg-82ce-9992b5d5c51b",
  "name": "ubuntu_server_14.04.2_amd64",
  "type": "iso",
  "os": "ubuntu", 
  "timezone": "America/Bogota", 
  "preseed_url": "https://<host>:<port>/api/virtshell/v1/files/seeds/seed_ubuntu_14_04.txt",
  "download_url": "http://releases.ubuntu.com/raring/ubuntu-14.04-server-amd64.iso",
  "permissions" : "rwxrwxrwx",
  "created":["at":"20130625105211","by":10]
}
\end{lstlisting}

\paragraph{Obtener todas las imagenes - GET /virtshell/api/v1/images} ~\\

\begin{lstlisting}[style=json]
curl -sv -H 'accept: application/json' 
     -H 'X-VirtShell-Authorization: UserId:Signature' \ 
     'http://localhost:8080/api/virtshell/v1/images'
\end{lstlisting}

\vspace{1cm}
Respuesta:
\vspace{1cm}

\begin{lstlisting}[style=json]
HTTP/1.1 200 OK
Content-Type: application/json
{
  "images": [
    {
      "id": "b180ef2c-e798-4a8f-b23f-aaac2fb8f7e8",
      "name": "ubuntu_server_14.04.2_amd64",
      "type": "iso",
      "os": "ubuntu",  
      "timezone": "America/Bogota", 
      "preseed_file": "https://<host>:<port>/api/virtshell/v1/files/seeds/seed_file.txt",
      "download_url": "http://releases.ubuntu.com/raring/ubuntu-14.04-server-amd64.iso",
      "permissions" : "rwxrw----",
      "created":["at":"20130625105211","by":10]
    },
    {
      "id": "ca326181-bc84-4edb-bfc5-843037e7195e",
      "name": "centos:centos6",
      "type": "docker-container",
      "os": "centos", 
      "permissions" : "rwxrwxr--",
      "created":["at":"20140625105211","by":12]
    }
  ]
}  
\end{lstlisting}

\paragraph{Eliminar una imagen - DELETE \\ /virtshell/api/v1/images/:name} ~\\

\begin{lstlisting}[style=json]
curl -sv -X DELETE \
   -H 'accept: application/json' \
   -H 'X-VirtShell-Authorization: UserId:Signature' \
   'http://<host>:<port>/api/virtshell/v1/images/ubuntu_server_14.04.2_amd64'
\end{lstlisting}

\vspace{1cm}
Respuesta:
\vspace{1cm}

\begin{lstlisting}[style=json]
HTTP/1.1 200 OK
Content-Type: application/json
```
```json
{ "delete": "success" }
\end{lstlisting}

\subsection{Packages}
Representan paquetes de software que se ejecutan en las m'aquinas virtuales o contenedores. Los metodos soportados son:

\begin{center}
 \begin{tabular}{| l | l | l | l |}
 \hline
  \rowcolor{blueapi}
  \textbf{Acci'on} & \textbf{Metodo HTTP} & \textbf{Solicitud HTTP} & \textbf{Descripci'on} \\ [0.5ex] 
  \hline\hline
  install & POST & /install\_packages/ & Install one or more packages. \\
  \hline
  upgrade & POST & /upgrade\_packages/ & Upgrade one or more packages. \\
  \hline
  remove & POST & /remove\_packages/ & Remove one or more packages. \\ [1ex] 
  \hline
\end{tabular}
\end{center}

\vspace{1cm}
Representaci'on del recurso de un paquete:
\vspace{1cm}

\begin{lstlisting}[style=json]
{
  "packages": [
      {"name": "package_name1"},
      {"name": "package_name2"}
  ],
  "hosts": [ 
      {"name": "Host_", "range": "[1-3]"}, 
      {"name": "database_001"}
  ],
  "tags": [
    {"name": "db"},
    {"name": "web"}
  ]
}
\end{lstlisting}

Ejemplo:

\medskip
\begin{lstlisting}[style=json]
{
  "packages": [
      {"name": "git"},
      {"name": "nginx"}
  ],
  "hosts": [ 
      {"name": "Host_", "range": "[1-3]"}
  ]
}
\end{lstlisting}

\subsection{Ejemplos de peticiones HTTP}

\subsubsection{Instalar uno o mas paquetes - POST /virtshell/api/v1/install\_packages}

\begin{lstlisting}[style=json]
curl -sv -X PUT \
  -H 'accept: application/json' \
  -H "Content-Type: text/plain" \
  -H 'X-VirtShell-Authorization: UserId:Signature' \
  -d '{ "packages": [{"name": "git"}, {"name": "nginx"}],
        "hosts": [{"name": "WebServer_", "range": "[1-3]"}]}' \
   'http://localhost:8080/api/virtshell/v1/install_packages'
\end{lstlisting}

\vspace{1cm}
Respuesta:
\vspace{1cm}

\begin{lstlisting}[style=json]
HTTP/1.1 202 Accepted
Content-Type: application/json
{ "install_package": "accepted" }
\end{lstlisting}

\subsubsection{Actualizar uno o mas paquetes - POST /virtshell/api/v1/upgrade\_packages}

\begin{lstlisting}[style=json]
curl -sv -X PUT \
  -H 'accept: application/json' \
  -H "Content-Type: text/plain" \
  -H 'X-VirtShell-Authorization: UserId:Signature' \
  -d '{ "packages": [{"name": "git"}, {"name": "nginx"}, {"name": "mc"}],
        "hosts": [{"name": "WebServer_", "range": "[1-3]"}]}' \
   'http://localhost:8080/api/virtshell/v1/upgrade_packages'
\end{lstlisting}

\vspace{1cm}
Respuesta:
\vspace{1cm}

\begin{lstlisting}[style=json]
HTTP/1.1 202 Accepted
Content-Type: application/json
{ "install_package": "accepted" }
\end{lstlisting}

\subsubsection{Remover uno o mas paquetes - POST /virtshell/api/v1/remove\_packages}

\begin{lstlisting}[style=json]
curl -sv -X PUT \
  -H 'accept: application/json' \
  -H "Content-Type: text/plain" \
  -H 'X-VirtShell-Authorization: UserId:Signature' \
  -d '{ "packages": [{"name": "apache2"}],
        "hosts": [{"name": "WebServer_", "range": "[1-3]"}]}' \
   'http://localhost:8080/api/virtshell/v1/remove_packages'
\end{lstlisting}

\vspace{1cm}
Respuesta:
\vspace{1cm}

\begin{lstlisting}[style=json]
HTTP/1.1 202 Accepted
Content-Type: application/json
{ "install_package": "accepted" }
\end{lstlisting}
\subsection{Files}
Representan toda clase de archivos que se requieran para crear o aprovisionar m'aquinas virtuales o contenedores. Los metodos soportados son:

\begin{center}
 \begin{tabular}{| l | l | l | l |}
 \hline
  \rowcolor{blueapi}
  \textbf{Acci'on} & \textbf{Metodo HTTP} & \textbf{Solicitud HTTP} & \textbf{Descripci'on} \\ [0.5ex] 
  \hline\hline
  get & GET & /files/id & Gets one file by ID. \\
  \hline
  create & POST & /files/ & upload a new file. \\
  \hline
  delete & DELETE & /files/id & Deletes an existing file. \\
  \hline  
  update & PUT & /files/id & Updates an existing file. \\ [1ex]  
  \hline
\end{tabular}
\end{center}

\vspace{1cm}
Representaci'on del recurso de un archivo:
\vspace{1cm}

\begin{lstlisting}[style=json]
{
  "uuid": "ab8076c0-db91-11e2-82ce-0002a5d5c51b",
  "name": "file_name.extension",
  "folder_name" : "folder_name",
  "download_url": "https://<host>:<port>/api/virtshell/v1/files/folder_name/file.txt",
  "created":["at":"timestamp", "by":user_id]
}
\end{lstlisting}

Ejemplo:

\medskip
\begin{lstlisting}[style=json]
{
  "uuid": "ab8076c0-db91-11e2-82ce-0002a5d5c51b",
  "name": "ubuntu_seed_14-04.tex",
  "folder_name" : "ubuntu_seeds",
  "download_url": "https://<host>:<port>/api/virtshell/v1/files/ubuntu_seeds/ubuntu_seed_14-04.tex",
  "created": ["at":"20130625105211", "by":10]
}
\end{lstlisting}

\subsubsection{Ejemplos de peticiones HTTP}

\paragraph{Subir un nuevo archivo - POST /virtshell/api/v1/images} ~\\

\begin{lstlisting}[style=json]
curl -X POST \
  -H 'accept: application/json' \
  -H 'X-VirtShell-Authorization: UserId:Signature' \
  -H "Content-Type: multipart/form-data" \
  -F "file_data=@/path/to/file/seed_file.txt;filename=seed_file_ubuntu-14_04.txt" \
  -F "folder_name=ubuntu_seeds" \
  'http://<host>:<port>/api/virtshell/v1/files'
\end{lstlisting}

\vspace{1cm}
Respuesta:
\vspace{1cm}

\begin{lstlisting}[style=json]
HTTP/1.1 200 OK
Content-Type: application/json
{ 
  "create": "success",
  "location": "http://<host>:<port>/api/virtshell/v1/files/ubuntu_seeds/seed_file_ubuntu-14_04.txt" 
}
\end{lstlisting}

\paragraph{Obtener un archivo - GET /virtshell/api/v1/files/:id} ~\\

Para descargar un archivo, primero recibira la url apropiada que viene en la metadata provista por la url. Luego podra descargarlo usando la url.

\begin{lstlisting}[style=json]
curl -sv -H 'accept: application/json' 
     -H 'X-VirtShell-Authorization: UserId:Signature' \ 
     'http://<host>:<port>/api/virtshell/v1/files/?id=ab8076c0-db91-11e2-82ce-0002a5d5c51b'
\end{lstlisting}

\vspace{1cm}
Respuesta:
\vspace{1cm}

\begin{lstlisting}[style=json]
HTTP/1.1 200 OK
Content-Type: application/json
{
  "uuid": "ab8076c0-db91-11e2-82ce-0002a5d5c51b",
  "name": "file_name.extension",
  "folder_name" : "folder_name",
  "download_url": "http://<host>:<port>/api/virtshell/v1/files/ubuntu_seeds/seed_file_ubuntu-14_04.txt",
  "created":["at":"timestamp", "by":user_id] 
}
\end{lstlisting}

\paragraph{Actualizar un archivo - PUT /virtshell/api/v1/files/:id} ~\\

\begin{lstlisting}[style=json]
curl -sv -X PUT \
  -H 'accept: application/json' \
  -H 'X-VirtShell-Authorization: UserId:Signature' \
  -H "Content-Type: multipart/form-data" \
  -F "file_data=@/path/to/file/seed_file.txt;filename=seed_file_ubuntu-14_04_v2.txt" \
   'http://localhost:8080/api/virtshell/v1/file?id=8de7b824-d7d1-4265-a3a6-5b46cc9b8ed5'
\end{lstlisting}

\vspace{1cm}
Respuesta:
\vspace{1cm}

\begin{lstlisting}[style=json]
HTTP/1.1 200 OK
Content-Type: application/json

{ "update": "success" }
\end{lstlisting}


\paragraph{Eliminar un archivo - DELETE /virtshell/api/v1/files/:id} ~\\

\begin{lstlisting}[style=json]
curl -sv -X DELETE \
   -H 'accept: application/json' \
   -H 'X-VirtShell-Authorization: UserId:Signature' \
   'http://localhost:8080/api/virtshell/v1/fles?id=ab8076c0-db91-11e2-82ce-0002a5d5c51b'
\end{lstlisting}

\vspace{1cm}
Respuesta:
\vspace{1cm}

\begin{lstlisting}[style=json]
HTTP/1.1 200 OK
Content-Type: application/json
```
```json
{ "delete": "success" }
\end{lstlisting}



\section{API Calls}

\subsection{Start Instance}

Permite iniciar una instancia.

\paragraph{Iniciar una instance - \\ POST /virtshell/api/v1/instances/start\_instance/:id} ~\\


\begin{lstlisting}[style=json]
curl -sv -X POST \
  -H 'accept: application/json' \
  -H 'X-VirtShell-Authorization: UserId:Signature' \
   'http://localhost:8080/virtshell/api/v1/instances/start\_instance/420aa3f0-8d96-11e5-8994-feff819cdc9f'
\end{lstlisting}

Response:

\begin{lstlisting}[style=json]
HTTP/1.1 200 OK
Content-Type: application/json
{ "start": "success" }
\end{lstlisting}


\subsection{Stop Instance}

Permite detener una instancia.

\paragraph{Detener una instancia - \\ POST /virtshell/api/v1/instances/stop\_instance/:id} ~\\

\begin{lstlisting}[style=json]
curl -sv -X POST \
  -H 'accept: application/json' \
  -H 'X-VirtShell-Authorization: UserId:Signature' \
   'http://localhost:8080/virtshell/api/v1/instances/stop\_instance/420aa3f0-8d96-11e5-8994-feff819cdc9f'
\end{lstlisting}

Response:

\begin{lstlisting}[style=json]
HTTP/1.1 200 OK
Content-Type: application/json
{ "stop": "success" }
\end{lstlisting}


\subsection{Restart Instance}

Permite reiniciar una instancia.

\paragraph{Reiniciar una instancia - \\ POST /virtshell/api/v1/instances/restart\_instance/:id} ~\\

\begin{lstlisting}[style=json]
curl -sv -X POST \
  -H 'accept: application/json' \
  -H 'X-VirtShell-Authorization: UserId:Signature' \
   'http://localhost:8080/virtshell/api/v1/instances/restart\_instance/420aa3f0-8d96-11e5-8994-feff819cdc9f'
\end{lstlisting}

Response:

\begin{lstlisting}[style=json]
HTTP/1.1 200 OK
Content-Type: application/json
{ "restart": "success" }
\end{lstlisting}


\subsection{Clone Instance}

Permite clonar una instancia.

\paragraph{Clonar una instancia - \\ POST /virtshell/api/v1/instances/clone\_instance/:id} ~\\

\begin{lstlisting}[style=json]
curl -sv -X POST \
  -H 'accept: application/json' \
  -H 'X-VirtShell-Authorization: UserId:Signature' \
   'http://localhost:8080/virtshell/api/v1/instances/clone\_instance/420aa3f0-8d96-11e5-8994-feff819cdc9f'
\end{lstlisting}

Response:

\begin{lstlisting}[style=json]
HTTP/1.1 200 OK
Content-Type: application/json
{ "clone": "success" }
\end{lstlisting}

\subsection{Execute command}

Permite ejecutar un comando en una o mas instancias.

Representaci'on del recurso para ejecutar un comando:

\medskip
\begin{lstlisting}[style=json]
{
  "instances": [ ... list of instances names, patterns(*|[numeric:numeric]) or tags ...],
  "command": string,
  "created": {"at": timestamp, "by": string}
}
\end{lstlisting}

Ejemplo:

\medskip
\begin{lstlisting}[style=json]
{
  "instances": [
      {"name": "database\_server\_01"},
      {"name": "transactional\_server\_co"},      
      {"pattern": "web\_server*"},
      {"pattern": "grid\_[1:5]"},
      {"tag": "web"}
  ],
  "command": "apt-get upgrade",
  "created": {"at": timestamp, "by": string}
}
\end{lstlisting}

\paragraph{Ejecutar un comando en una o mas instancias - \\ POST /virtshell/api/v1/instances/execute\_command/} ~\\

\begin{lstlisting}[style=json]
curl -sv -X POST \
  -H 'accept: application/json' \
  -H 'X-VirtShell-Authorization: UserId:Signature' \
  -d '{ "instances": [
          {"name": "database\_server\_01"},
          {"name": "transactional\_server\_co"},          
          {"pattern": "web\_server*"},
          {"pattern": "grid\_server\_[1:5]"},
          {"tag": "web"}
        ],
        "command": "apt-get upgrade" }' \
  'http://localhost:8080/virtshell/api/v1/instances/execute\_command/'
\end{lstlisting}

Response:

\begin{lstlisting}[style=json]
HTTP/1.1 200 OK
Content-Type: application/json
{ "execute_command": "success" }
\end{lstlisting}


\subsection{Copy files}

Permite ejecutar copiar uno archivo en una o mas instancias.

Representaci'on del recurso para ejecutar un comando:

\medskip
\begin{lstlisting}[style=json]
{
  "path": string,
  "destination": string,
  "instances": [ ... list of instances names, patterns(*|[numeric:numeric]) or tags ...],
  "created": {"at": timestamp, "by": string}
}
\end{lstlisting}

Ejemplo:

\medskip
\begin{lstlisting}[style=json]
{
  "uuid_file": "0d832c60-7066-4d37-bd72-ce6ac4f61bcc",
  "destination": "$MYSQL_HOME/my.cnf"
  "instances": [
      {"name": "database\_server\_01"},
      {"name": "web\_server*"},
      {"name": "grid\_[1:5]"},
      {"name": "transactional\_server\_co"},
      {"tag": "web"}
  ]
}
\end{lstlisting}

\paragraph{Copiar un archivo en una o mas instancias - \\ POST /virtshell/api/v1/instances/copy\_files/} ~\\

\begin{lstlisting}[style=json]
curl -sv -X POST \
  -H 'accept: application/json' \
  -H 'X-VirtShell-Authorization: UserId:Signature' \
  -d '{ "uuid_file": "0d832c60-7066-4d37-bd72-ce6ac4f61bcc",
        "destination": "$MYSQL_HOME/my.cnf"
        "instances": [
            {"name": "database\_server\_01"},
            {"name": "web\_server*"},
            {"name": "grid\_[1:5]"},
            {"name": "transactional\_server\_co"},
            {"tag": "web"}
        ] }' \
  'http://localhost:8080/virtshell/api/v1/instances/copy\_files/'
\end{lstlisting}

Response:

\begin{lstlisting}[style=json]
HTTP/1.1 200 OK
Content-Type: application/json
{ "copy_files": "success" }
\end{lstlisting}

