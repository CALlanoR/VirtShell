\subsection{Hosts}
Representan las maquinas físicas; un host es un anfitrión de maquinas virtuales o contenedores. Los métodos soportados son:\\
\\

\begin{center}
 \captionof{table}{Métodos HTTP para hosts}
 \begin{tabular}{| l | l | l | l |}
 \hline
  \rowcolor{blueapi}
  \textbf{Acción} & \textbf{Método HTTP} & \textbf{Solicitud HTTP} & \textbf{Descripción} \\ [0.5ex] 
  \hline\hline
  get & GET & /hosts/:name & Gets one host by name. \\
  \hline
  list & GET & /hosts & Retrieves the list of hosts. \\
  \hline  
  create & POST & /hosts/ & Inserts a new host. \\
  \hline
  delete & DELETE & /hosts/:name & Deletes an existing host. \\
  \hline  
  update & PUT & /hosts/:name & Updates an existing host. \\ [1ex] 
  \hline
\end{tabular}
\end{center}

Representación del recurso de un host:

\medskip
\begin{lstlisting}[style=json]
{
  "uuid": string,
  "name": string,
  "os": string,
  "memory": string,
  "capacity": string,
  "enabled": string,
  "type":string,
  "local_ipv4": string,
  "local_ipv6": string,
  "public_ipv4": string,
  "public_ipv6": string,
  "instances": [ instance_resource ],
  "partition": string,
  "created":["at": number, "by": number]
}
\end{lstlisting}

Ejemplo:

\medskip
\begin{lstlisting}[style=json]
{
  "uuid": "ab8076c0-db91-11e2-82ce-0002a5d5c51b",
  "name": "host-01-pdn",
  "os": "Ubuntu_12.04_3.5.0-23.x86_64",
  "memory": "16GB",
  "capacity": "120GB",
  "enabled": "true|false",
  "type":"StorageOptimized|GeneralPurpose|HighPerformance",
  "local_ipv4": "15.54.88.19",
  "local_ipv6": "ff06:0:0:0:0:0:0:c3",
  "public_ipv4": "10.54.88.19",
  "public_ipv6": "yt06:0:0:0:0:0:0:c3",
  "instances": [
    ... instances resource is here
  ],
  "partition": "development_co",
  "created":["at":"timestamp", "by":1234]
}
\end{lstlisting}

\subsubsection{Ejemplos de peticiones HTTP}

\paragraph{Crear un nuevo host - POST /api/virtshell/v1/hosts} ~\\

\begin{lstlisting}[style=json]
curl -sv -X POST \
  -H 'accept: application/json' \
    -H 'X-VirtShell-Authorization: UserId:Signature' \
  -d '{"name": "host-01-pdn",
       "os": "Ubuntu_12.04_3.5.0-23.x86_64",
       "memory": "16GB",
       "capacity": "120GB",
       "enabled": "true",
       "type" : "GeneralPurpose",
       "local_ipv4": "15.54.88.19",
       "local_ipv6": "ff06:0:0:0:0:0:0:c3",
       "public_ipv4": "10.54.88.19",
       "public_ipv6": "yt06:0:0:0:0:0:0:c3",
       "partition": "development_co"}' \
   'http://localhost:8080/virtshell/api/v1/hosts'
\end{lstlisting}

Response:

\begin{lstlisting}[style=json]
HTTP/1.1 200 OK
Content-Type: application/json
{ "create": "success" }
\end{lstlisting}

\paragraph{Obtener un host- GET /api/virtshell/v1/hosts/:name} ~\\

\begin{lstlisting}[style=json]
curl -sv -H 'accept: application/json' 
     -H 'X-VirtShell-Authorization: UserId:Signature' \ 
     'http://localhost:8080/api/virtshell/v1/hosts/host-01-pdn'
\end{lstlisting}

Response:

\begin{lstlisting}[style=json]
HTTP/1.1 200 OK
Content-Type: application/json
{
  "uuid": "ab8076c0-db91-11e2-82ce-0002a5d5c51b",
  "name": "host-01-pdn",
  "os": "Ubuntu_12.04_3.5.0-23.x86_64",
  "memory": "16GB",
  "capacity": "120GB",
  "enabled": "true",
  "type" : "StorageOptimized",
  "local_ipv4": "15.54.88.19",
  "local_ipv6": "ff06:0:0:0:0:0:0:c3",
  "public_ipv4": "10.54.88.19",
  "public_ipv6": "yt06:0:0:0:0:0:0:c3",
  "instances": [
    {
      "name": "name1",
      "id": "72C05559-0590-4DA6-BE56-28AB36CB669C"
    },
    {
      "name": "name2",
      "id": "17173587-C4E9-4369-9C43-FCBF5E075973"
    }
  ],
  "partition": "development_co",
  "created":["at":"20130625105211", "by":10]
}
\end{lstlisting}

\paragraph{Obtener todos los host - GET /api/virtshell/v1/hosts} ~\\

\begin{lstlisting}[style=json]
curl -sv -H 'accept: application/json' 
     -H 'X-VirtShell-Authorization: UserId:Signature' \ 
     'http://localhost:8080/api/virtshell/v1/hosts'
\end{lstlisting}

Response:

\begin{lstlisting}[style=json]
HTTP/1.1 200 OK
Content-Type: application/json
{
  "hosts": [
    {
      "uuid": "ab8076c0-db91-11e2-82ce-0002a5d5c51b",
      "name": "host-01-pdn",
      "os": "Ubuntu_12.04_3.5.0-23.x86_64",
      "memory": "16GB",
      "capacity": "120GB",
      "enabled": "true",
      "type" : "StorageOptimized",
      "local_ipv4": "15.54.88.19",
      "local_ipv6": "ff06:0:0:0:0:0:0:c3",
      "public_ipv4": "10.54.88.19",
      "public_ipv6": "yt06:0:0:0:0:0:0:c3",
      "instances": [
        {
          "name": "name1",
          "id": "72C05559-0590-4DA6-BE56-28AB36CB669C"
        },
        {
          "name": "name2",
          "id": "17173587-C4E9-4369-9C43-FCBF5E075973"
        }
      ],
      "partition": "development_co",
      "created":["at":"20130625105211", "by":10]
    },
    {
      "uuid": "ab8076c0-db91-11e2-82ce-0002a5d5c51b",
      "name": "host-01-pdn",
      "os": "Ubuntu_12.04_3.5.0-23.x86_64",
      "memory": "16GB",
      "capacity": "120GB",
      "enabled": "true",
      "type" : "GeneralPurpose",
      "local_ipv4": "15.54.88.19",
      "local_ipv6": "ff06:0:0:0:0:0:0:c3",
      "public_ipv4": "10.54.88.19",
      "public_ipv6": "yt06:0:0:0:0:0:0:c3",
      "instances": [
        {
          "name": "name3",
          "id": "DE11CC9A-482F-4033-A7F8-503EE449DD0A"
        },
        {
          "name": "name4",
          "id": "17173587-C4E9-4369-9C43-FCBF5E075973"
        },    
      ],
      "partition": "development_mx",
      "created":["at":"20130625105211", "by":10]
    }
  ]
}   
\end{lstlisting}

\paragraph{Actualizar un host - PUT /api/virtshell/v1/hosts/:name} ~\\

\begin{lstlisting}[style=json]
curl -sv -X PUT \
  -H 'accept: application/json' \
    -H 'X-VirtShell-Authorization: UserId:Signature' \
  -d '{"memory": "24GB",
     "capacity": "750GB"}' \
   'http://localhost:8080/api/virtshell/v1/hosts/host-01-pdn'
\end{lstlisting}

Response:

\begin{lstlisting}[style=json]
HTTP/1.1 200 OK
Content-Type: application/json

{ "update": "success" }
\end{lstlisting}

\paragraph{Eliminar un host - DELETE /api/virtshell/v1/hosts/:name} ~\\

\begin{lstlisting}[style=json]
curl -sv -X DELETE \
   -H 'accept: application/json' \
   -H 'X-VirtShell-Authorization: UserId:Signature' \
   'http://localhost:8080/api/virtshell/v1/hosts/host-01-pdn'
\end{lstlisting}

Response:

\begin{lstlisting}[style=json]
HTTP/1.1 200 OK
Content-Type: application/json
```
```json
{ "delete": "success" }
\end{lstlisting}