\section{Files}
Representan toda clase de archivos que se requieran para crear o aprovisionar m'aquinas virtuales o contenedores. Los metodos soportados son:

\begin{center}
 \begin{tabular}{| l | l | l | l |}
 \hline
  \rowcolor{blueapi}
  \textbf{Acci'on} & \textbf{Metodo HTTP} & \textbf{Solicitud HTTP} & \textbf{Descripci'on} \\ [0.5ex] 
  \hline\hline
  get & GET & /files/id & Gets one file by ID. \\
  \hline
  create & POST & /files/ & upload a new file. \\
  \hline
  delete & DELETE & /files/id & Deletes an existing file. \\
  \hline  
  update & PUT & /files/id & Updates an existing file. \\ [1ex]  
  \hline
\end{tabular}
\end{center}

\vspace{1cm}
Representaci'on del recurso de un archivo:
\vspace{1cm}

\begin{lstlisting}[style=json]
{
  "uuid": "ab8076c0-db91-11e2-82ce-0002a5d5c51b",
  "name": "file_name.extension",
  "folder_name" : "folder_name",
  "download_url": "https://<host>:<port>/api/virtshell/v1/files/folder_name/file.txt",
  "created":["at":"timestamp", "by":user_id]
}
\end{lstlisting}

Ejemplo:

\medskip
\begin{lstlisting}[style=json]
{
  "uuid": "ab8076c0-db91-11e2-82ce-0002a5d5c51b",
  "name": "ubuntu_seed_14-04.tex",
  "folder_name" : "ubuntu_seeds",
  "download_url": "https://<host>:<port>/api/virtshell/v1/files/ubuntu_seeds/ubuntu_seed_14-04.tex",
  "created": ["at":"20130625105211", "by":10]
}
\end{lstlisting}

\subsection{Ejemplos de peticiones HTTP}

\subsubsection{Subir un nuevo archivo - POST /virtshell/api/v1/images}

\begin{lstlisting}[style=json]
curl -X POST \
  -H 'accept: application/json' \
  -H 'X-VirtShell-Authorization: UserId:Signature' \
  -H "Content-Type: multipart/form-data" \
  -F "file_data=@/path/to/file/seed_file.txt;filename=seed_file_ubuntu-14_04.txt" \
  -F "folder_name=ubuntu_seeds" \
  'http://<host>:<port>/api/virtshell/v1/files'
\end{lstlisting}

\vspace{1cm}
Respuesta:
\vspace{1cm}

\begin{lstlisting}[style=json]
HTTP/1.1 200 OK
Content-Type: application/json
{ 
  "create": "success",
  "location": "http://<host>:<port>/api/virtshell/v1/files/ubuntu_seeds/seed_file_ubuntu-14_04.txt" 
}
\end{lstlisting}

\subsubsection{Obtener un archivo - GET /virtshell/api/v1/files/:id}

Para descargar un archivo, primero recibira la url apropiada que viene en la metadata provista por la url. Luego podra descargarlo usando la url.

\begin{lstlisting}[style=json]
curl -sv -H 'accept: application/json' 
     -H 'X-VirtShell-Authorization: UserId:Signature' \ 
     'http://<host>:<port>/api/virtshell/v1/files/?id=ab8076c0-db91-11e2-82ce-0002a5d5c51b'
\end{lstlisting}

\vspace{1cm}
Respuesta:
\vspace{1cm}

\begin{lstlisting}[style=json]
HTTP/1.1 200 OK
Content-Type: application/json
{
  "uuid": "ab8076c0-db91-11e2-82ce-0002a5d5c51b",
  "name": "file_name.extension",
  "folder_name" : "folder_name",
  "download_url": "http://<host>:<port>/api/virtshell/v1/files/ubuntu_seeds/seed_file_ubuntu-14_04.txt",
  "created":["at":"timestamp", "by":user_id] 
}
\end{lstlisting}

\subsubsection{Actualizar un archivo - PUT /virtshell/api/v1/files/:id}

\begin{lstlisting}[style=json]
curl -sv -X PUT \
  -H 'accept: application/json' \
  -H 'X-VirtShell-Authorization: UserId:Signature' \
  -H "Content-Type: multipart/form-data" \
  -F "file_data=@/path/to/file/seed_file.txt;filename=seed_file_ubuntu-14_04_v2.txt" \
   'http://localhost:8080/api/virtshell/v1/file?id=8de7b824-d7d1-4265-a3a6-5b46cc9b8ed5'
\end{lstlisting}

\vspace{1cm}
Respuesta:
\vspace{1cm}

\begin{lstlisting}[style=json]
HTTP/1.1 200 OK
Content-Type: application/json

{ "update": "success" }
\end{lstlisting}


\subsubsection{Eliminar un archivo - DELETE /virtshell/api/v1/files/:id}

\begin{lstlisting}[style=json]
curl -sv -X DELETE \
   -H 'accept: application/json' \
   -H 'X-VirtShell-Authorization: UserId:Signature' \
   'http://localhost:8080/api/virtshell/v1/fles?id=ab8076c0-db91-11e2-82ce-0002a5d5c51b'
\end{lstlisting}

\vspace{1cm}
Respuesta:
\vspace{1cm}

\begin{lstlisting}[style=json]
HTTP/1.1 200 OK
Content-Type: application/json
```
```json
{ "delete": "success" }
\end{lstlisting}
