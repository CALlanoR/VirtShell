\chapter{Planteamiento del problema}
\label{capproblema}

\section{Problema}
En la actualidad, con la creciente adopción de modelos de computación como el \emph{Cloud Computing} \footnote{conocida también como servicios en la nube, es un paradigma que permite ofrecer servicios de computación a través de una red, que usualmente es Internet.} y el \emph{Grid Computing} \footnote{Un grid es un sistema de computación distribuido que permite coordinar computadoras de diferente hardware y software y cuyo fin es procesar una tarea que demanda una gran cantidad de recursos y poder de procesamiento.}, los ambientes computacionales se han tornado cada vez más sofisticados y complejos, requiriendo de soluciones computacionales que traten de manera integral el aprovisionamiento de diferentes servicios sobre ambientes virtuales capaces de atender la variable demanda computacional a través del despliegue de infraestructuras elásticas de computación.\\
\\
Hoy en día, se encuentran diversas soluciones que abordan el problema de aprovisionamiento usando diferentes enfoques para el despliegue y orquestación de plataformas y servicios. Sin embargo los enfoques actuales carecen de mecanismos de comunicación que permitan interoperar entre diferentes aplicaciones; En general, las soluciones actuales presentan dificultades para ser accedidas en una red, como internet y ejecutadas de manera remota.\\
\\
En consecuencia, para lograr la interoperabilidad con los actuales modelos de computación, este proyecto propone como objetivo principal, plantear el diseño de un framework orientado al web, cuyas partes soporten una arquitectura que permita un eficiente aprovisionamiento de software de manera automática, para ambientes virtualizados. De igual modo, se propone realizar una ejemplificación del framework en un ambiente virtualizado.\\
\\
Para lograrlo sera necesario evaluar diferentes estilos arquitectónicos, realizar un modelo del framework, definir la plataforma en la que se realizara una ejemplificación y definir el mecanismo de aprovisionamiento que se utilizara para aprovisionar ambientes en maquinas virtuales.\\ 
\\

\section{Objetivo General}
Diseñar un framework web, que permita el aprovisionamiento de software automático, para ambientes virtualizados.

\section{Objetivos Específicos}
\begin{itemize}
\item Evaluar diferentes herramientas de aprovisionamiento que se utilizan en la actualidad.
\item Evaluar diferentes estilos arquitecturales que soporten el framewok de aprovisionamiento.
\item Evaluar diferentes mecanismos de aprovisionamiento.
\item Realizar una ejemplificación del framework.
\end{itemize}
