\documentclass[oneside,numbers,spanish]{ezthesis}
%% # Opciones disponibles para el documento #
%%
%% Las opciones con un (*) son las opciones predeterminadas.
%%
%% Modo de compilar:
%%   draft            - borrador con marcas de fecha y sin im'agenes
%%   draftmarks       - borrador con marcas de fecha y con im'agenes
%%   final (*)        - version final de la tesis
%%
%% Tama'no de papel:
%%   letterpaper (*)  - tama'no carta (Am'erica)
%%   a4paper          - tama'no A4    (Europa)
%%
%% Formato de impresi'on:
%%   oneside          - hojas impresas por un solo lado
%%   twoside (*)      - hijas impresas por ambos lados
%%
%% Tama'no de letra:
%%   10pt, 11pt, o 12pt (*)
%%
%% Espaciado entre renglones:
%%   singlespace      - espacio sencillo
%%   onehalfspace (*) - espacio de 1.5
%%   doublespace      - a doble espacio
%%
%% Formato de las referencias bibliogr'aficas:
%%   numbers          - numeradas, p.e. [1]
%%   authoryear (*)   - por autor y a'no, p.e. (Newton, 1997)
%%
%% Opciones adicionales:
%%   spanish         - tesis escrita en espa'nol
%%
%% Desactivar opciones especiales:
%%   nobibtoc   - no incluir la bibiolgraf'ia en el 'Indice general
%%   nofancyhdr - no incluir "fancyhdr" para producir los encabezados
%%   nocolors   - no incluir "xcolor" para producir ligas con colores
%%   nographicx - no incluir "graphicx" para insertar gr'aficos
%%   nonatbib   - no incluir "natbib" para administrar la bibliograf'ia

%% Paquetes adicionales requeridos se pueden agregar tambi'en aqu'i.
%% Por ejemplo:
%\usepackage{subfig}
%\usepackage{multirow}

%% # Datos del documento #
%% Nota que los acentos se deben escribir: \'a, \'e, \'i, etc.
%% La letra n con tilde es: \~n.


\usepackage{titlesec}
\usepackage{fancyhdr}
\usepackage[Sonny]{fncychap} %Options: Sonny, Lenny, Glenn, Conny, Rejne
                             %         Bjarne, Bjornstrup
\usepackage{lmodern}
\usepackage{listings}
\usepackage{color}
\usepackage{array}
\usepackage{colortbl}
\usepackage{url}
\usepackage{xcolor}
\usepackage[utf8]{inputenc}


\setcounter{secnumdepth}{4}

\author{Carlos Alberto Llano Rodríguez}
\title{VirtShell - Framework para aprovisionamiento de soluciones virtuales}
\degree{Maestría en Ingeniería con énfasis en Ingeniería de Sistemas}
\supervisor{Jhon Alexander Sanabria}
\institution{Universidad del Valle}
\faculty{Escuela de Ingeniería de Sistemas y Computación}
\department{Facultad de Ingeniería}

%% # M'argenes del documento #
%% 
%% Quitar el comentario en la siguiente linea para austar los m'argenes del
%% documento. Leer la documentaci'on de "geometry" para m'as informaci'on.

%\geometry{top=40mm,bottom=33mm,inner=40mm,outer=25mm}

%% El siguiente comando agrega ligas activas en el documento para las
%% referencias cruzadas y citas bibliogr'aficas. Tiene que ser *la 'ultima*
%% instrucci'on antes de \begin{document}.
\hyperlinking


% Colors in http://latexcolor.com/
\definecolor{lightgray}{rgb}{.9,.9,.9}
\definecolor{darkgray}{rgb}{.4,.4,.4}
\definecolor{purple}{rgb}{0.65, 0.12, 0.82}
\definecolor{cornflowerblue}{rgb}{0,0.4,0.8}
\definecolor{blueapi}{rgb}{0.74, 0.83, 0.9}
\definecolor{magnolia}{rgb}{0.97, 0.96, 1.0}
\colorlet{punct}{red!60!black}
\definecolor{background}{HTML}{EEEEEE}
\definecolor{delim}{RGB}{20,105,176}
\colorlet{numb}{magenta!60!black}
\definecolor{navyblue}{rgb}{0.0, 0.0, 0.5}

\newcommand\JSONnumbervaluestyle{\color{navyblue}}
\newcommand\JSONstringvaluestyle{\color{red}}

% switch used as state variable
\newif\ifcolonfoundonthisline

\makeatletter

\lstdefinestyle{json}
{
  showstringspaces    = false,
  numbers             = left,
  numberstyle         = \scriptsize,
  stepnumber          = 1,
  numbersep           = 8pt,
  breaklines          = true,
  frame               = lines,
  alsoletter          = 0123456789.,
  morestring          = [s]{"}{"},
  literate            = {:}{{{\color{punct}{:}}}}{1}
                        {,}{{{\color{punct}{,}}}}{1}
                        {\{}{{{\color{delim}{\{}}}}{1}
                        {\}}{{{\color{delim}{\}}}}}{1}
                        {[}{{{\color{delim}{[}}}}{1}
                        {]}{{{\color{delim}{]}}}}{1}
                        {|}{{{\color{delim}{|}}}}{1},
  stringstyle         = \ifcolonfoundonthisline\JSONstringvaluestyle\fi,
  MoreSelectCharTable =%
    \lst@DefSaveDef{`:}\colon@json{\processColon@json},
  basicstyle          = \ttfamily,
  backgroundcolor     = \color{magnolia},
  keywordstyle        = \ttfamily\bfseries,
}

% flip the switch if a colon is found in Pmode
\newcommand\processColon@json{%
  \colon@json%
  \ifnum\lst@mode=\lst@Pmode%
    \global\colonfoundonthislinetrue%
  \fi
}

\lst@AddToHook{Output}{%
  \ifcolonfoundonthisline%
    \ifnum\lst@mode=\lst@Pmode%
      \def\lst@thestyle{\JSONnumbervaluestyle}%
    \fi
  \fi
  %override by keyword style if a keyword is detected!
  \lsthk@DetectKeywords% 
}

% reset the switch at the end of line
\lst@AddToHook{EOL}%
  {\global\colonfoundonthislinefalse}

\makeatother

\begin{document}

%% En esta secci'on se describe la estructura del documento de la tesis.
%% Consulta los reglamentos de tu universidad para determinar el orden
%% y la cantidad de secciones que debes de incluir.

%% # Portada de la tesis #
%% Mirar el archivo "titlepage.tex" para los detalles.
%% ## Construye tu propia portada ##
%% 
%% Una portada se conforma por una secuencia de "Blocks" que incluyen
%% piezas individuales de informaci'on. Un "Block" puede incluir, por
%% ejemplo, el t'itulo del documento, una im'agen (logotipo de la universidad),
%% el nombre del autor, nombre del supervisor, u cualquier otra pieza de
%% informaci'on.
%%
%% Cada "Block" aparece centrado horizontalmente en la p'agina y,
%% verticalmente, todos los "Blocks" se distruyen de manera uniforme 
%% a lo largo de p'agina.
%%
%% Nota tambi'en que, dentro de un mismo "Block" se pueden cortar
%% lineas usando el comando \\
%%
%% El tama'no del texto dentro de un "Block" se puede modificar usando uno de
%% los comandos:
%%   \small      \LARGE
%%   \large      \huge
%%   \Large      \Huge
%%
%% Y el tipo de letra se puede modificar usando:
%%   \bfseries - negritas
%%   \itshape  - it'alicas
%%   \scshape  - small caps
%%   \slshape  - slanted
%%   \sffamily - sans serif
%%
%% Para producir plantillas generales, la informaci'on que ha sido inclu'ida
%% en el archivo principal "tesis.tex" se puede accesar aqu'i usando:
%%   \insertauthor
%%   \inserttitle
%%   \insertsupervisor
%%   \insertinstitution
%%   \insertdegree
%%   \insertfaculty
%%   \insertdepartment
%%   \insertsubmitdate

\begin{titlepage}
  \TitleBlock{\scshape\insertinstitution}
  \TitleBlock[\bigskip]{\scshape\insertfaculty}
  \TitleBlock{\Huge\scshape\inserttitle}
  \TitleBlock{\scshape
    Tesis presentada por \insertauthor \\
    para obtener el grado de \insertdegree}
  \TitleBlock{\insertsubmitdate}
  \TitleBlock[\bigskip]{\insertdepartment}
\end{titlepage}

%% Nota 1:
%% Se puede agregar un escudo o logotipo en un "Block" como:
%%   \TitleBlock{\includegraphics[height=4cm]{escudo_uni}}
%% y teniendo un archivo "escudo_uni.pdf", "escudo_uni.png" o "escudo_uni.jpg"
%% en alg'un lugar donde LaTeX lo pueda encontrar.

%% Nota 2:
%% Normalmente, el espacio entre "Blocks" se extiende de modo que el
%% contenido se reparte uniformemente sobre toda la p'agina. Este
%% comportamiento se puede modificar para mantener fijo, por ejemplo, el
%% espacio entre un par de "Blocks". Escribiendo:
%%   \TitleBlock{Bloque 1}
%%   \TitleBlock[\bigskip]{Bloque2}
%% se deja un espacio "grande" y de tama~no fijo entre el bloque 1 y 2.
%% Adem'as de \bigskip est'an tambi'en \smallskip y \medskip. Si necesitas
%% aun m'as control puedes usar tambi'en, por ejemplo, \vspace*{2cm}.




%% # Prefacios #
%% Por cada prefacio (p.e. agradecimientos, resumen, etc.) crear
%% un nuevo archivo e incluirlo aqu'i.
%% Para m'as detalles y un ejemplo mirar el archivo "gracias.tex".
%% Las secciones del "prefacio" inician con el comando \prefacesection{T'itulo}
%% Este tipo de secciones *no* van numeradas, pero s'i aparecen en el 'indice.
\prefacesection{Agradecimientos}

\begin{itemize}
\item A Dios, por iluminarme en momentos dificiles.
\item A mi familia sin ella, no hubiera podido alcanzar esta meta.
\item A Jhon Alexander Sanabria, profesor de la Facultad de Ingenier'ia de la Universidad del Valle, por su inmensa paciencia y apoyo a lo largo de todo el proyecto.
\item A mis amigos colaboradores de la Universidad del Valle, por todo el apoyo y por creer siempre en este gran esfuerzo.
\item A mis compañeros y amigos por sus palabras de aliento y constante apoyo.
\item A todas aquellas personas que de una u otra forma colaboraron en la realizaci'on del presente trabajo.
\end{itemize}



%% # 'Indices y listas de contenido #
%% Quitar los comentarios en las lineas siguientes para obtener listas de
%% figuras y cuadros/tablas.
\tableofcontents
\listoffigures
\listoftables

%% # Cap'itulos #
%% Por cada cap'itulo hay que crear un nuevo archivo e incluirlo aqu'i.
%% Mirar el archivo "intro.tex" para un ejemplo y recomendaciones para
%% escribir.
\chapter{Introducci'on}

Ideas para la introduccion:

En los últimos años se han producido importantes avances tecnológicos que han dado lugar a la demanda de nuevas aplicaciones relacionadas con la automatización de .........\\
\\
Por un lado ha aumentado el número de aplicaciones de ...... que requieren estrategicas ....... capaces de ...... . Por otro lado, las aplicaciones  de ultima generacion. En esta tesis se propone una estrategia distinta de aprovisionamiento y administracion, que cumplen los requisitos de velocidad y facilidad de uso demandados por los usuarios.\\
\\
Se propone un sistema web rapido y eficaz, capas de ....... para aprovisionar ...... 

basada en el popular metodo de ....., capaz de ......

La estrategia propuesta mejora muy notablemente la ......





\chapter{Estado del arte}

\label{aprmaqvir}
La computación en la nube ha sido un punto importante de investigación en la industria recientemente. Esta puede ser descrita como una nueva clase de computación en la cual recursos dinámicos y escalables pueden ser provistos sobre internet. Para los usuarios esto es transparente y ellos solo pagan lo que usan de acuerdo a niveles de servicio establecidos con los proveedores de nubes.\\
\\
En ese contexto, una de las principales características de la computación en la nube es la virtualización, la cual consiste en crear una versión virtual de un recurso tecnológico en lugar de usar una versión física. La virtualización se puede aplicar a computadoras, sistemas operativos, dispositivos de almacenamiento de información, aplicaciones o redes permitiendo que las empresas ejecuten mas de un sistema virtual, ademas de múltiples sistemas operativos y aplicaciones, en un único servidor, de esta manera se logra economía de escala y una mayor eficiencia.\\
\\
\section{Técnicas de Virtualización}
Actualmente predominan dos técnicas de virtualización, la primera técnica se denomina virtualización de hardware y consiste en crear un hardware sintético el cual usan las maquinas virtuales como propio, la idea es virtualizar el sistema operativo completo el cual se ejecuta sobre un software llamado el hypervisor, su función es interactuar directamente con la CPU en el servidor físico, ofreciendo a cada uno de los servidores virtuales una total autonomía e independencia. Incluso pueden coexistir en una misma maquina distintos servidores virtuales funcionando con distintos sistemas operativos. Esta técnica es la mas desarrollada y hay diferentes clases que cada fabricante ha ido desarrollando y adaptando, como por ejemplo Xen, KVM, VMWare y VirtualBox.\\
\\
La segunda técnica es conocida como virtualización del sistema operativo. En esta técnica lo que se virtualiza es el sistema operativo completo el cual corre directamente virtual sobre la maquina física. En esta técnica las maquinas virtuales son llamadas contenedores, los cuales acceden por igual a todos los recursos del sistema. La ventaja es a su vez una desventaja: Todas las maquinas virtuales usan el mismo Kernel que el sistema operativo lo que reduce mucho los errores y multiplica el rendimiento, pero a su vez solo puede haber un mismo tipo de sistema operativo en los contenedores, no se puede mezclar Windows-Linux-Etc. Este sistema, también se acerca mucho a lo que seria una virtualización nativa.\\
\\
De hecho, sin importar la técnica de virtualización que se use, la instalación de una maquina virtual (o de un contenedor) requiere normalmente de la generación e instalación de una imagen y a su vez de la instalación y configuración de paquetes de software. Estas tareas generalmente son realizadas por técnicos de los proveedores de la nube. Cuando un usuario de la nube solicita un nuevo servicio o mas capacidad de computo, el administrador selecciona la apropiada imagen para clonar e instalar en los nodos de la nube. Si no hay una imagen apropiada para los requerimientos del cliente se crea y configura una nueva que cumpla con la solicitud. Esta creación de una nueva imagen puede ser realizada modificando la imagen mas cercada de las ya existentes. En el momento de la creación optima de la imagen un administrador puede tener dificultades y preguntas como, cual es la mejor configuración?, cuales paquetes y sus dependencias deberían ser instaladas? y como encontrar una imagen que mejor llene las expectativas?.\\
\\
Por lo tanto, los proveedores de la nube desean cada vez mas automatizar y simplificar este proceso porque la dependencia entre paquetes de software y la dificultad de mantenimiento agrega tiempo a la creación de las maquinas virtuales. En otras palabras los proveedores de nube quieren dar mas flexibilidad y agilidad a la hora de satisfacer los requerimientos de los usuarios finales.\\
\\
\section{Soluciones de aprovisionamiento}
Ciertamente existen muchas soluciones que permiten la interacción con diferentes ambientes de visualización. Estas soluciones usan diferentes enfoques para realizar despliegues de software en las maquinas virtuales de manera rápida, controlada y automática, en maquinas físicas o virtuales. Sin embargo la mayoría las soluciones no tienen la capacidad de manejar de manera simultanea las dos técnicas de virtualización antes mencionadas, algunas se centran solo en manejar maquinas virtuales y otras pocas solo hacen aprovisionamiento sobre contenedores.\\
\\
Así mismo, hay soluciones de aprovisionamiento que han incorporado su propio lenguaje de aprovisionamiento buscando mayor flexibilidad y fácil configuración de las tareas pero que incorporan una curva de aprendizaje bastante alta lo que se traduce en un gran esfuerzo inicial para contar con toda la infraestructura automatizada. En ese mismo orden de ideas, existen a su vez, soluciones cuya curva de aprendizaje es mucho menor lo que las hace mas atractivas para muchos ingenieros de TI.\\
\\
Adicionalmente, así como se encuentran soluciones o herramientas de aprovisionamiento de desarrollo propietario que cobran por sus funcionalidades mas importantes o por el numero de maquinas que pueden aprovisionar, también se encuentran herramientas de código abierto, o por lo menos de uso libre que permiten trabajar con un gran numero de maquinas virtuales.\\
\\
Al revisar al rededor de 40 diferentes herramientas de aprovisionamiento se logro identificar dos características que no se encuentran en las soluciones actuales, la primera trata de la ausencia de una interfaz o API web para realizar aprovisionamiento remoto, y la segunda se refiere a que las herramientas se limitan a aprovisionar las maquinas virtuales pero no ofrecen mecanismos de administración y monitoreo de la red aprovisionada ni de los hosts que albergan los recursos virtuales.\\
\\
A continuación se describirán algunas de las herramientas mas significativas que existen indicando sus características principales.

\begin{description}
\item [Fabric]
 es una herramienta de automatización que usa SSH para hacer despliegues de aplicaciones y administración de tareas. Fabric es una librería gratuita hecha en python y su forma de interactuar es por medio de linea de comandos por otra parte permite cargar y descargar archivos que pueden ser ejecutados por su conjunto de funciones \cite{fabfile16}.

\item [Chef]
es una de las herramientas más conocidas de automatización de infraestructura de nube, esta escrita en Ruby y Erlang. Utiliza un lenguaje de dominio especifico escrito también en Ruby para la escritura y configuración de "recetas". Estas recetas contienen los recursos que deben ser creados. Chef se puede integrar con plataformas basadas en la nube, como Rackspace, Internap, Amazon EC2, Cloud Platform Google, OpenStack, SoftLayer y Microsoft Azure. Adicionalmente puede aprovisionar sobre contenedores si se instala la librería indicada. Chef contiene soluciones para sistemas de peque~na y gran escala. \cite{Chef15}\\
\\
Es uno de los cuatro principales sistemas de gestión de la configuración en Linux, junto con Cfengine, Bcfg2 y Puppet. Para un cierto numero de nodos se puede usar su versión gratuita, pero para contar con todo conjunto de características, administración y soporte no es gratuita.

\item [Puppet]
es una herramienta diseñada para administrar la configuración de sistemas similares a Unix y a Microsoft Windows de forma declarativa. El usuario describe los recursos del sistema y sus estados utilizando el lenguaje declarativo que proporciona Puppet. Esta información es almacenada en archivos denominados manifiestos Puppet. Puppet descubre la información del sistema a través de una utilidad llamada Facter, y compila los manifiestos en un catalogo especifico del sistema que contiene los recursos y la dependencia de dichos recursos, estos catálogos son ejecutados en los sistemas de destino \cite{Pupet15}.\\
\\
Puppet es de uso gratuito para redes muy pequeñas de hasta solo 10 nodos.

\item [Juju]
 es una herramienta de configuración y administración de servicios en nubes publicas. Permite crear ambientes completos con unos pocos comandos, cuenta con cientos de servicios pre-configurados y disponibles en la tienda de juju. Se puede usar a través de una interfaz gráfica o de linea de comandos. Juju permite re-crear un ambiente de producción en portátiles usando contenedores enfocado a pruebas. El uso de juju es gratuito pero se debe pagar por el uso de la nube publica \cite{juju16}.

\item [CFEngine]
es un sistema basado en el lenguaje escrito por Mark Burgess, dise~nado específicamente para probar y configurar software. CFEngine es como un lenguaje de muy alto nivel. La idea de CFEngine es crear un único archivo o conjunto de archivos de configuración que describen la configuración de cada host de la red. CFEngine se ejecuta en cada host, y analiza cada archivo (o archivos), que especifica una política para la configuración del sistema; la configuración del host es verificada contra el modelo y, si es necesario, cualquier desviación de la configuración es corregida. \cite{cfengine15}\\
\\
CFEngine cuenta con una versión gratuita y la versión empresarial que cuenta con interfaz gráfica, soporte y reportes.

\item [Ansible]
es una herramienta de código libre desarrollada en python y comercialmente ofrecida por AnsibleWorks los cuales la definen como un motor de orquestación muy simple que automatiza las tareas de despliegue. Ansible no usa agentes, solo necesita tener instalado Python en las maquinas hosts y las tareas las realiza por medio de ssh. Ansible podría trabajar mediante un solo archivo de configuración que contendría todo o por medio de varios archivos organizados en una estructura de directorios \cite{ans16}. 

\item [Bcfg2]
esta escrito en Python y permite gestionar la configuración de un gran numero de ordenadores mediante un modelo de configuración central. Bcfg2 funciona con un modelo simple de configuración del sistema, modelando elementos intuitivos como paquetes, servicios y archivos de configuración (así como las dependencias entre ellos). Este modelo de configuración del sistema se utiliza para la verificación  y validación, permitiendo una auditoria robusta de los sistemas desplegados. La especificación de la configuración de Bcfg2 está escrita utilizando un modelo XML declarativo. Toda la especificacion puede ser validada utilizando los validadores de esquema XML ampliamente disponibles. Bcfg2 no tiene soporte para contenedores. Es gratuito y cuenta con una lista limitada de plataformas en las cuales trabaja bien.\cite{bdfg215}

\item [Cobbler]
 es una plataforma que busca el rápido despliegue de servidores y en general computadores en una infra-estructura de red por medio de linea de comandos, se basa en el modelo de scripts y cuenta con una completa base de simples comandos, que permite hacer despliegues de manera rápida y con poca intervención humana. Cobbler es capaz de instalar maquinas físicas y maquinas virtuales. Cobbler, es una pequena y ligera aplicacion, que es extremadamente facil de usar para pequeños o muy grandes despliegues. Es de uso gratuito y no cuenta con soporte para contenedores. \cite{Cobbler15}

\item [SmartFrog]
 es un framework para servicios de configuración, descripción, despliegue y administración del ciclo de vida. Consiste de un lenguaje declarativo, un motor que corre en los nodos remotos y ejecuta plantillas escritas en el lenguaje de SmartFrog y un modelo de componentes. El lenguaje soporta encapsulación (que es similar a las clases de python), herencia y composición que permite personalizar y combinar configuraciones. SmartFrog, permite enlaces estáticos y dinámicos entre componentes, que ayudan a soportar diferentes formas de conexión en tiempo de despliegue.\\
\\
El modelo de componentes, administra el ciclo de vida a través de cinco estados: instalado, iniciado, terminado y fallido. Esto permite al motor del SmartFrog detectar fallas y reiniciar automáticamente re-despliegues de los componentes \cite{Smart09}.\\
\\
SmartFrog es desarrollado y mantenido por un equipo de investigación en los laboratorios de Hewlett-Packard en Bristol, Inglaterra, así como por el laboratorio Europeo de Hewlett-Packard y adicional con contribuciones de otros usuarios de SmartFrog y desarrolladores externos a HP. Se utiliza en la investigación de HP específicamente en la automatización de la infraestructura y automatización de servicios, ademas de ser solo utilizado en determinados productos de HP.

\item [Amazon EC2]
es un API propietario de Amazon y maneja un enfoque manual, que permite desplegar imágenes de maquinas virtuales conocidas como AMI (Amazon Machine Images) \cite{Amazon16}, que son las imágenes que se utilizan en Amazon para arrancar instancias. El concepto de las amis es similar a las maquinas virtuales de otros sistemas. Básicamente estan compuestas de una serie ficheros de datos que conforman la imagen y luego un xml que especifica ciertos valores necesarios para que sea una imagen valida para Amazon que es el image.manifest.xml. 

\item [Docker composer]
permite describir un conjunto de contenedores que se relacionan entre ellos. Docker composer permite definir una aplicación multicontenedor en un archivo con las mismas propiedades que se indicarían en un archivo individual de docker. Docker composer usa archivos en formato yaml para describir las características de los servicios en cada contenedor \cite{doccom16}. Es completamente gratuito. 

\item [Vagrant]
 es una herramienta de linea de comando que permite la creación y configuración de entornos de desarrollo virtualizados. Originalmente se desarrolló para VirtualBox y sistemas de configuración tales como Chef, Salt y Puppet. Sin embargo desde la versión 1.1 Vagrant es capaz de trabajar con múltiples proveedores, como VMware, Amazon EC2, LXC, DigitalOcean \cite{Vag15}.
\\
Vagrant ofrece múltiples opciones para realizar aprovisionamientos, desde scritps en shell hasta complejos sistemas de configuración.

\item [SaltStack]
 es un sistema de manejo de configuración cuyo objetivo es garantizar que un servicio este corriendo o que una aplicación haya sido instalada o desplegada. Salt está construido en Python y al igual que Chef, CFEngine y Puppet utiliza un esquema de cliente (salt minions) - servidor (salt master), cuyo método de conexión con los minions se realiza a través de un broker messages llamado ZeroMQ (0MQ), que no solo garantiza una conexión segura sino que la hace confiable y rápida. Salt tiene una versión gratuita llamada Salt Open sin embargo para obtener todos los beneficios se debe pagar la versión empresarial. Salt cuenta con soporte para contenedores. \cite{Salt15}

\end{description}

\chapter{VirtShell: Framework para aprovisionamiento de soluciones virtuales}
\label{capdisevirtshell}

\section {Dise�o Inicial de VirtShell}


\section{Autenticaci'on}

La autenticaci'on es el proceso de demostrar la identidad al sistema. La identidad es un factor importante en las decisiones de control de acceso. Las solicitudes se conceden o deniegan en parte sobre la base de la identidad del solicitante.\\
\\
El VirtShell, el API REST utiliza un esquema HTTP personalizado basado en una llave-HMAC (Hash Message Authentication Code) para la autenticaci'on. Para autenticar una solicitud, primero se concatenan los elementos seleccionados de la solicitud para formar una cadena. A continuaci'on, utiliza una clave secreta de acceso para calcular el HMAC de esa cadena. Informalmente, se le denomina a este proceso \"la firma de la solicitud\", y se denomina a la salida del algoritmo HMAC la "firma", ya que simula las propiedades de seguridad de una firma real. Por 'ultimo, se agrega esta firma como un par'ametro de la petici'on, con la sintaxis descrita en esta secci'on.\\
\\
Cuando el sistema recibe una solicitud fehaciente, se obtiene la clave secreta de acceso que dicen tener, y lo utiliza de la misma manera que se calcula una "firma" del mensaje que recibi'o. A continuaci'on, compara la firma que se calcula con la firma presentada por el solicitante. Si las dos firmas coinciden, el sistema llega a la conclusi'on de que el solicitante debe tener acceso a la clave secreta de acceso, y por lo tanto act'ua con la autoridad del principal al que se emiti'o la clave. Si las dos firmas no coinciden, la solicitud se descarta y el sistema responde con un mensaje de error.\\
\\
Ejemplo de una petici'on autenticada:

\medskip
\begin{lstlisting}
  GET /api/virtshell/packages/{packageId} HTTP/1.1
  Host: host1.edu.co
  Date: Fri, 01 Jul 2011 19:37:58 +0000

  Authorization: 0PN5J17HBGZHT7JJ3X82:frJIUN8DYpKDtOLCwo//yllqDzg= 
\end{lstlisting}

\subsection{Authentication Header}

El API REST utiliza el encabezado de autorizaci'on HTTP para pasar informaci'on de autenticaci'on. Bajo el esquema de autenticaci'on de VirtShell, el encabezado de autorizaci'on tiene la siguiente forma.

\medskip
\begin{lstlisting}
  Authorization: UserId:Signature
\end{lstlisting}
\medskip

Los usuarios tendr'an un ID de clave de acceso (VirtShell Access Key ID) y una clave secreta de acceso (VirtShell Secret Access Key) cuando se registran. Para la petici'on de autenticaci'on, el elemento de VirtShell Access Key Id identifica la clave secreta que se utiliz'o para calcular la firma, y (indirectamente) el usuario que realiza la solicitud.\\
\\
Para la firma de los elementos de la petici'on se usa el RFC 2104HMAC-SHA1, por lo que la parte de la firma de la cabecera autorizaci'on variar'a de una petici'on a otra. Si la solicitud de la firma calculada por el sistema coincide con la firma incluida en la solicitud, el solicitante habr'a demostrado la posesi'on de la clave secreta de acceso. La solicitud ser'a procesada bajo la identidad, y con la autoridad, de la promotora que se emiti'o la clave.\\
\\
A continuaci'on se muestra la pseudo-gramática que ilustra la construcci'on de la cabecera de la solicitud de autorizaci'on (
\textbackslash{}n significa el punto de c'odigo Unicode U +000 A com'unmente llamado salto de l'inea).

\medskip
\begin{lstlisting}
  Authorization = VirtShellUserId + ":" + Signature;

  Signature = Base64( HMAC-SHA1( UTF-8-Encoding-Of( YourSecretAccessKeyID, StringToSign ) ) );

  StringToSign = HTTP-Verb + "\n" +
  Host + "\n" +
  Content-MD5 + "\n" +
  Content-Type + "\n" +
  Date + "\n" +
  CanonicalizedResource;

  CanonicalizedResource = <HTTP-Request-URI, from the protocol name up to the query string (resource path)>
\end{lstlisting}

HMAC-SHA1 es un algoritmo definido por la RFC 2104 (ver la RFC 2104 con llave Hashing para la autenticaci'on de mensajes).\\
\\
El algoritmo toma como entrada dos cadenas de bytes: una clave y un mensaje. Para la solicitud de autenticaci'on, se utiliza la clave secreta (YourSecretAccessKeyID) como la clave, y la codificaci'on UTF-8 del StringToSign como el mensaje. La salida de HMAC-SHA1 es tambi'en una cadena de bytes, llamado el resumen. El par'ametro de la petici'on de la Firma se construye codificada en Base64.

\subsection{Solicitud can'onica para firmar}

Cuando el sistema recibe una solicitud autenticada, compara la solicitud de firma calculada con la firma proporcionada en la solicitud de StringToSign. Por esta raz'on, se debe calcular la firma con el mismo m'etodo utilizado por VirtShell. A este proceso de poner una solicitud en una forma acordada para la firma se denomino "canonizaci'on".

\subsection{Tiempo de sello}

Un sello de tiempo v'alido (utilizando el HTTP header Date) es obligatorio para solicitudes autenticadas. Por otra parte, el tiempo del sello enviado por un usuario que se encuentra incluido en una solicitud autenticada debe estar dentro de los 15 minutos de la hora del sistema cuando se recibe la solicitud. En caso contrario, la solicitud fallar'a con el c'odigo de estado de error RequestTimeTooSkewed. La intenci'on de estas restricciones es limitar la posibilidad de que solicitudes interceptadas pueden ser reproducidos por un adversario.Para una mayor protecci'on contra las escuchas, se debe utilizar el transporte HTTPS para solicitudes autenticadas.

\subsection{Ejemplos de autenticaci'on}

\begin{tabular}{|l|l|} \hline
\textbf{Parametro} & \textbf{Valor} \\ \hline
VirtShellUserId  & 13010f3e-3f46-4889-b989-592ce8fb30c6 \\ \hline
\multicolumn{1}{|m{3.7cm}|}{
VirtShellSecretAccessKey} & \multicolumn{1}{m{12.5cm}|}%
{\raggedright c991f519-bed0-4dab-9165-6d9f722dc845 \\
\textbf{Base64:} \\ Yzk5MWY1MTktYmVkMC00ZGFiLTkxNjUtNmQ5ZjcyMmRjODQ1} \tabularnewline \hline
\end{tabular}

\textbf{Ejemplo de un objeto con GET}

Este es un ejemplo que consulta por un host dado su identificador.

\begin{tabular}{|l|l|} \hline
\textbf{Request} & \textbf{StringToSign} \\ \hline
\multicolumn{1}{|m{10.5cm}|}%
{\raggedright GET /api/virtshell/hosts/45 HTTP/1.1 \\
 Host: host1.edu.co \\
 Date: Tue, 27 Mar 2007 19:36:42 +0000 \\
 Authorization: 13010f3e-3f46-4889-b989-592ce8fb30c6: Yzk5MWY1MmVkMC00ZGFiLTtNmQ5ZjcyMmRjODQ1 } & \multicolumn{1}{m{6.5cm}|}%
{\raggedright GET\textbackslash{}n \\
 host1.edu.co\textbackslash{}n \\
 \textbackslash{}n \\
 \textbackslash{}n \\
 Tue, 27 Mar 2007 19:36:42 +0000\textbackslash{}n \\ /api/virtshell/hosts/45} \tabularnewline \hline
\end{tabular}

\textbf{Ejemplo de un objeto con DELETE}

Este ejemplo remueve un usuario.

\begin{tabular}{|l|l|} \hline
\textbf{Request} & \textbf{StringToSign} \\ \hline
\multicolumn{1}{|m{10.5cm}|}%
{\raggedright DELETE /api/virtshell/users/9876 HTTP/1.1 \\
 Host: host1.edu.co \\
 Date: Tue, 27 Mar 2007 21:20:27 +0000 \\
 Authorization: 13010f3e-3f46-4889-b989-592ce8fb30c6: Yzk5MWY1MmVkMC00ZGFiLTtNmQ5ZjcyMmRjODQ1 } & \multicolumn{1}{m{6.5cm}|}%
{\raggedright DELETE\textbackslash{}n \\
 host1.edu.co\textbackslash{}n \\
 \textbackslash{}n \\
 \textbackslash{}n \\
 Tue, 27 Mar 2007 21:20:27 +0000\textbackslash{}n \\ /api/virtshell/users/9876} \tabularnewline \hline
\end{tabular}

\chapter{API}
\label{capapi}

En este capítulo se detalla el API REST de VirtShell, en el se describen cada uno de los métodos HTTP que soportan los módulos, sus recursos en formato JSON y adicionalmente se ofrecen ejemplos detallados de cada una de las peticiones HTTP.\\

\section{Definición de API}
API significa ``Application Programming Interface", y como término, especifica cómo debe interactuar el software.\\
\\
En términos generales, cuando nos referimos a las API de hoy, nos referimos más concretamente a las API web, que son manejadas a través del protocolo de transferencia de hipertexto (HTTP). Para este caso específico, entonces, una API especifica cómo un consumidor puede consumir el servicio que el API expone: cuales URI están disponibles, qué métodos HTTP puede utilizarse con cada URI, que parámetros de consulta se acepta, lo que los datos que pueden ser enviados en el cuerpo de la petición, y lo que el consumidor puede esperar como respuesta.

\subsection{VirtShell API REST}
En el VirtShell API REST un usuario enviá una solicitud al servidor para realizar una acción determinada (como la creación, recuperación, actualización o eliminación de un recurso virtual), y el servidor realiza la acción y enviá una respuesta, a menudo en la forma de una representación del recurso especificado.\\
\\
En el VirtShell API, el usuario especifica una acción con un verbo HTTP como POST, GET, PUT o DELETE. Especificando un recurso por un URI único global de la siguiente forma: \\
\\
https://[host]:[port]/virtshell/api/v1/resourcePath?parameters\\
\\
Debido a que todos los recursos del API tienen una única URI HTTP accesible, REST permite el almacenamiento en cache de datos y esta optimizado para trabajar con una infraestructura distribuida de la web.\\
\\
En esta sección se detalla los recursos y operaciones que puede realizar un usuario del API para realizar aprovisionamientos automáticos desde cualquier plataforma de desarrollo. El VirtShell API provee acceso a los objetos en el VirtShell Server, esto incluye los hosts, imágenes, archivos, templates, aprovisionadores, instancias, grupos y usuarios. Por medio del API podrá crear ambientes, maquinas virtuales y contenedores personalizados, realizar configuraciones y administrar los recursos físicos y virtuales de manera programática. \\

\section{Formato de entrada y salida}
JSON (JavaScript Object Notation) es un formato de datos común, independiente del lenguaje que proporciona una representación de texto simple de estructuras de datos arbitrarias. Para obtener mas información, ver json.org.\\
\\
El VirtShell API solo soporta el formato json para intercambio de información. Cualquier solicitud que no se encuentre en formato json resultara en un error con código 406 (Content Not Acceptable Error).

\section{Codigos de error}
Aqui se presenta una lista de codigos de error que pueden resultar de una petici'on al API en cualquier recurso.

\begin{itemize}
\item \textbf{400 Bad Request} La solicitud no pudo ser procesada con 'exito porque el URI no era v'alido. El cuerpo de la respuesta contendr'a una raz'on del fracaso de la petici'on. Esta respuesta indica error permanente.

\item \textbf{403 Forbidden} La solicitud no pudo ser procesada con 'exito porque la identidad del usuario no tiene acceso suficiente para procesar la solicitud. Esta respuesta indica error permanente.

\item \textbf{406 Content Not Acceptable} Un recurso genera este error de acuerdo al tipo de cabeceras enviadas en la petici'on. Esta respuesta indica un error permanete e indica un formato de salida no soportado. La respuesta de este tipo de error no contiene un contenido debido a la inhabilidad del servidor para generar una respuesta en el formato solicitado.

\item \textbf{404 Not Found} La solicitud no pudo ser procesada con 'exito porque la solicitud no era v'alida. Lo m'as probable es que no se encontró la url. Esta respuesta indica error permanente.

\item \textbf{500 Server Error} La solicitud no pudo ser procesada debido a que el servidor encontr'o una condici'on inesperada que le impidi'o cumplir con la petici'on.

\item \textbf{501 Not Implemented} La solicitud no se pudo completar porque el servidor o bien no reconoce el m'etodo de petici'on o el recurso solicitado no existe.

\end{itemize}

Los errores que no sean de codigo 406 (Content Not Acceptable) contienen una respuesta en formato json, que contiene un breve mensaje explicado el error con m'as detalle. Por ejemplo, una consulta POST /virtshell/api/v1/hosts, con un cuerpo vacio, dar'ia lugar a la siguiente respuesta:

\vspace{1cm}
\begin{lstlisting}[style=json]
HTTP/1.1 400 Bad Request
Content-Type: application/json

{"error": "Missing input for create instance"}
\end{lstlisting}

\section{API Resources}

\subsection{Groups}
Representan los grupos registrados en VirtShell. Los metodos soportados son:

\begin{center}
 \begin{tabular}{| l | l | l | l |}
 \hline
  \rowcolor{blueapi}
  \textbf{Acci'on} & \textbf{Metodo HTTP} & \textbf{Solicitud HTTP} & \textbf{Descripci'on} \\ [0.5ex] 
  \hline\hline
  get & GET & /users/id & Gets one group by ID. \\
  \hline
  list & GET & /hosts & Retrieves the list of groups. \\  
  \hline
  create & POST & /users/ & creates a new group. \\
  \hline
  delete & DELETE & /users/id & Deletes an existing group. \\
  \hline
\end{tabular}
\end{center}

\vspace{1cm}
Representaci'on del recurso de un grupo:
\vspace{1cm}

\begin{lstlisting}[style=json]
{
  "uuid": "ab8076c0-db91-11e2-82ce-0002a5d5c51b",
  "name": "web_development_team",
  "users": [ ... list of members of the group ...],  
  "created":[ {"at":"timestamp"}, {"by":user_id}]
}
\end{lstlisting}

Ejemplo:

\medskip
\begin{lstlisting}[style=json]
{
  "uuid": "ab8076c0-db91-11e2-82ce-0002a5d5c51b",
  "name": "web_development_team",
  "users": [ 
      {"username": "user1", "id": "a146cae4-8c90-11e5-8994-feff819cdc9f"},
      {"username": "user2", "id": "a146d00c-8c90-11e5-8994-feff819cdc9f"}
  ]
  "created":[{"at":"1447696674"}, {"by":"a379e8e6-8c8b-11e5-8994-feff819cdc9f"}]
}
\end{lstlisting}

\subsubsection{Ejemplos de peticiones HTTP}

\paragraph{Crear un nuevo grupo - POST /virtshell/api/v1/grupos} ~\\

\begin{lstlisting}[style=json]
curl -X POST \
  -H 'accept: application/json' \
  -H 'X-VirtShell-Authorization: UserId:Signature' \
  -H "Content-Type: multipart/form-data" \
  -d '{"name": "database_team"}' \
  'http://<host>:<port>/api/virtshell/v1/groups'
\end{lstlisting}

\vspace{1cm}
Respuesta:
\vspace{1cm}

\begin{lstlisting}[style=json]
HTTP/1.1 200 OK
Content-Type: application/json
{ "create": "success" }
\end{lstlisting}

\paragraph{Obtener un grupo - GET /virtshell/api/v1/groups/:id} ~\\

\begin{lstlisting}[style=json]
curl -sv -H 'accept: application/json' 
     -H 'X-VirtShell-Authorization: UserId:Signature' \ 
     'http://<host>:<port>/api/virtshell/v1/groups/?id=ab8076c0-db91-11e2-82ce-0002a5d5c51b'
\end{lstlisting}

\vspace{1cm}
Respuesta:
\vspace{1cm}

\begin{lstlisting}[style=json]
HTTP/1.1 200 OK
Content-Type: application/json
{
  "uuid": "ab8076c0-db91-11e2-82ce-0002a5d5c51b",
  "name": "web_development_team",
  "users": [ 
      {"username": "user1", "id": "a146cae4-8c90-11e5-8994-feff819cdc9f"},
      {"username": "user2", "id": "a146d00c-8c90-11e5-8994-feff819cdc9f"}
  ]
  "created":[{"at":"1447696674"}, {"by":"a379e8e6-8c8b-11e5-8994-feff819cdc9f"}]
}
\end{lstlisting}

\paragraph{Obtener todos los grupos - GET /virtshell/api/v1/groups} ~\\

\begin{lstlisting}[style=json]
curl -sv -H 'accept: application/json' 
     -H 'X-VirtShell-Authorization: UserId:Signature' \ 
     'http://localhost:8080/api/virtshell/v1/groups'
\end{lstlisting}

\vspace{1cm}
Respuesta:
\vspace{1cm}

\begin{lstlisting}[style=json]
HTTP/1.1 200 OK
Content-Type: application/json
{
  "groups": [
    {
      "uuid": "ab8076c0-db91-11e2-82ce-0002a5d5c51b",
      "name": "web_development_team",
      "users": [ 
          {"username": "user1", "id": "a146cae4-8c90-11e5-8994-feff819cdc9f"},
          {"username": "user2", "id": "a146d00c-8c90-11e5-8994-feff819cdc9f"}
      ],     
      "created":[{"at":"1447696833"}, {"by":"d2372efa-8c8b-11e5-8994-feff819cdc9f"}]
    },
    {
      "uuid": "a379f19c-8c8b-11e5-8994-feff819cdc9f",
      "name": "math_team",
      "users": [ 
          {"username": "user3", "id": "a146cae4-8c90-11e5-8994-feff819cdc9f"}
      ],     
      "created":[{"at":"1421431233"}, {"by":"18489280-8c91-11e5-8994-feff819cdc9f"}]
    },
    {
      "uuid": "a379f3d6-8c8b-11e5-8994-feff819cdc9f",
      "name": "chemical_team",
      "users": [ 
          {"username": "user4", "id": "F8489280-8c91-11e5-8994-feff819cdc9f"},
          {"username": "user5", "id": "18489780-8c91-11e5-8994-feff819cdc9f"}
      ],       
      "created":[{"at":"1424109633"}, {"by":"d2373576-8c8b-11e5-8994-feff819cdc9f"}]
    },        
}  
\end{lstlisting}

\paragraph{Eliminar un grupo - DELETE /virtshell/api/v1/groups/:id} ~\\

Para eliminar un grupo se debe tener en cuenta que no debe tener usuarios asociados a el.

\begin{lstlisting}[style=json]
curl -sv -X DELETE \
   -H 'accept: application/json' \
   -H 'X-VirtShell-Authorization: UserId:Signature' \
   'http://localhost:8080/api/virtshell/v1/groups?id=73cff0b0-8c8e-11e5-8994-feff819cdc9f'
\end{lstlisting}

\vspace{1cm}
Respuesta:
\vspace{1cm}

\begin{lstlisting}[style=json]
HTTP/1.1 200 OK
Content-Type: application/json
```
```json
{ "delete": "success" }
\end{lstlisting}

\subsection{Users}
Representan los usuarios registrados en VirtShell. Los metodos soportados son:

\begin{center}
 \begin{tabular}{| l | l | l | l |}
 \hline
  \rowcolor{blueapi}
  \textbf{Acci'on} & \textbf{Metodo HTTP} & \textbf{Solicitud HTTP} & \textbf{Descripci'on} \\ [0.5ex] 
  \hline\hline
  get & GET & /users/id & Gets one user by ID. \\
  \hline
  create & POST & /users/ & creates a new user. \\
  \hline
  list & GET & /users & Retrieves the list of users. \\  
  \hline
  delete & DELETE & /users/id & Deletes an existing user. \\
  \hline  
  update & PUT & /users/id & Updates an existing user. \\ [1ex]  
  \hline
\end{tabular}
\end{center}

\vspace{1cm}
Representaci'on del recurso de un usuario:
\vspace{1cm}

\begin{lstlisting}[style=json]
{
  "uuid": "ab8076c0-db91-11e2-82ce-0002a5d5c51b",
  "username": "virtshell",
  "type": "system/regular",
  "login": "user@mail.com",
  "groups": [ ... list of users ...],
  "created": {"at": timestamp, "by": user_uuid},
  "modified": {"at": timestamp, "by": user_uuid}
}
\end{lstlisting}

Ejemplo:

\medskip
\begin{lstlisting}[style=json]
{
  "uuid": "ab8076c0-db91-11e2-82ce-0002a5d5c51b",
  "username": "virtshell",
  "type": "system/regular",
  "login": "user@mail.com",
  "groups": [ {"uuid": "a146cae4-8c90-11e5-8994-feff819cdc9f"},
              {"uuid": "a146d00c-8c90-11e5-8994-feff819cdc9f"}
  ],
  "created": {"at":"1429207233", "by":"92d30f0c-8c9c-11e5-8994-feff819cdc9f"},
  "modified": {"at":"1529207233", "by":"92d31132-8c9c-11e5-8994-feff819cdc9f"}
}
\end{lstlisting}

\subsubsection{Ejemplos de peticiones HTTP}

\paragraph{Crear un nuevo usuario - POST /api/virtshell/v1/users} ~\\

\begin{lstlisting}[style=json]
curl -X POST \
  -H 'accept: application/json' \
  -H 'X-VirtShell-Authorization: UserId:Signature' \
  -H "Content-Type: multipart/form-data" \
  -d {"uuid": "ab8076c0-db91-11e2-82ce-0002a5d5c51b",
       "username": "virtshell", 
       "type": "system/regular",
       "login": "user@mail.com",
       "groups": [ {"uuid": "a146cae4-8c90-11e5-8994-feff819cdc9f"},
                   {"uuid": "a146d00c-8c90-11e5-8994-feff819cdc9f"}
        ],
       "created": {"at":"1429207233", "by":"92d30f0c-8c9c-11e5-8994-feff819cdc9f"},
       "modified": {"at":"1529207233", "by":"92d31132-8c9c-11e5-8994-feff819cdc9f"}
      } \
  'http://<host>:<port>/api/virtshell/v1/users'
\end{lstlisting}

\vspace{1cm}
Respuesta:
\vspace{1cm}

\begin{lstlisting}[style=json]
HTTP/1.1 200 OK
Content-Type: application/json
{ "create": "success" }
\end{lstlisting}

\paragraph{Obtener un usuario - GET /api/virtshell/v1/users/:id} ~\\

\begin{lstlisting}[style=json]
curl -sv -H 'accept: application/json' 
     -H 'X-VirtShell-Authorization: UserId:Signature' \ 
     'http://<host>:<port>/api/virtshell/v1/users/?id=ab8076c0-db91-11e2-82ce-0002a5d5c51b'
\end{lstlisting}

\vspace{1cm}
Respuesta:
\vspace{1cm}

\begin{lstlisting}[style=json]
HTTP/1.1 200 OK
Content-Type: application/json
{
  "uuid": "ab8076c0-db91-11e2-82ce-0002a5d5c51b",
  "username": "virtshell",
  "type": "system/regular",
  "login": "user@mail.com",
  "groups": [ {"uuid": "a146cae4-8c90-11e5-8994-feff819cdc9f"}],
  "created": {"at":"1429207233", "by":"92d30f0c-8c9c-11e5-8994-feff819cdc9f"},
  "modified": {"at":"1529207233", "by":"92d31132-8c9c-11e5-8994-feff819cdc9f"}
}
\end{lstlisting}

\paragraph{Actualizar un usuario - PUT /api/virtshell/v1/users/:id} ~\\

\begin{lstlisting}[style=json]
curl -sv -X PUT \
  -H 'accept: application/json' \
  -H 'X-VirtShell-Authorization: UserId:Signature' \
  -H "Content-Type: multipart/form-data" \
  -d '{"type": "system",
       "groups": [{"uuid": "a146cae4-8c90-11e5-8994-feff819cdc9f"},
                  {"uuid": "a146d00c-8c90-11e5-8994-feff819cdc9f"}]}' \
   'http://localhost:8080/api/virtshell/v1/file?id=8de7b824-d7d1-4265-a3a6-5b46cc9b8ed5'
\end{lstlisting}

\vspace{1cm}
Respuesta:
\vspace{1cm}

\begin{lstlisting}[style=json]
HTTP/1.1 200 OK
Content-Type: application/json

{ "update": "success" }
\end{lstlisting}


\paragraph{Eliminar un usuario - DELETE /api/virtshell/v1/users/:id} ~\\

\begin{lstlisting}[style=json]
curl -sv -X DELETE \
   -H 'accept: application/json' \
   -H 'X-VirtShell-Authorization: UserId:Signature' \
   'http://localhost:8080/api/virtshell/v1/fles?id=ab8076c0-db91-11e2-82ce-0002a5d5c51b'
\end{lstlisting}

\vspace{1cm}
Respuesta:
\vspace{1cm}

\begin{lstlisting}[style=json]
HTTP/1.1 200 OK
Content-Type: application/json
```
```json
{ "delete": "success" }
\end{lstlisting}


% VirtShell is a multi-user framework that is based on the Unix permissions concepts to provide security.

% VirtShell provides mechanisms to control access by  limiting the types of
% resource access that can be made. Access is permitted or denied depending on
% several factors, one of which is the type of access requested. Several different
% types of operations may be controlled:

% Read. Read from the resouce.
% Write. Write or rewrite of resoures.
% Execute. Load the resource into host and execute it.

% Here is a quick breakdown of the access that the three basic permission types grant a user.

% Read
% ----
% Read permission allows a user to view the contents of any resource in VirtShell.

% Write
% -----
% Write permission allows a user to create, modify and delete whatever resources.

% Execute
% -------
% Execute permission allows a user to execute virtual machines or containers, for example: start, stop, pause, snapshot. (the user must also have read permission). 

\subsection{Partitions}
Las particones permiten organizar las máquinas que albergaran recursos virtuales en partes aisladas de las demás.Los métodos soportados son:

\begin{center}
 \captionof{table}{Métodos HTTP para partitions}
 \begin{tabular}{| l | l | l | l |}
 \hline
  \rowcolor{blueapi}
  \textbf{Acci'on} & \textbf{Método HTTP} & \textbf{Solicitud HTTP} & \textbf{Descripci'on} \\ [0.5ex] 
  \hline\hline
  get & GET & /partitions/:name & Gets one partition by name. \\
  \hline
  list & GET & /partitions & Retrieves the list of partitions. \\
  \hline  
  create & POST & /partitions/ & Inserts a new partition configuration. \\
  \hline
  delete & DELETE & /partitions/:name & Deletes an existing partition. \\
  \hline  
  update & PUT & /partitions/:name/host/:hostname & Add a host to partition. \\ [1ex] 
  \hline
\end{tabular}
\end{center}

Representaci'on del recurso de una partición:

\medskip
\begin{lstlisting}[style=json]
{
  "uuid": string,
  "name": string,
  "description": string, 
  "hosts": [ ... list of hosts associated with the Partitions ...],
  "created": {"at": number, "by": string},
  "modified": {"at": number, "by": string}
}
\end{lstlisting}

Ejemplo:

\medskip
\begin{lstlisting}[style=json]
{
  "uuid": "ab8076c0-db91-11e2-82ce-0002a5d5c51b",
  "name": "development_co",
  "description": "Collection of servers oriented to development team in Colombia.", 
  "hosts": [ ... list of hosts associated with the partition ...],
  "created": {"at":"1429207233", "by":"92d30f0c-8c9c-11e5-8994-feff819cdc9f"},
  "modified": {"at":"1529207233", "by":"92d31132-8c9c-11e5-8994-feff819cdc9f"}
}
\end{lstlisting}

\subsubsection{Ejemplos de peticiones HTTP}

\paragraph{Crear una nueva partición - POST /api/virtshell/v1/partitions} ~\\

\begin{lstlisting}[style=json]
curl -sv -X POST \
  -H 'accept: application/json' \
  -H 'X-VirtShell-Authorization: UserId:Signature' \
  -d '{
       "name": "development_co",
       "description": "Collection of servers oriented to development team in colombia."
      }' \
   'http://localhost:8080/api/virtshell/v1/partitions'
\end{lstlisting}

Response:

\begin{lstlisting}[style=json]
HTTP/1.1 200 OK
Content-Type: application/json
{ "create": "success" }
\end{lstlisting}

\paragraph{Obtener una partición- GET /api/virtshell/v1/partitions/:name} ~\\

\begin{lstlisting}[style=json]
curl -sv -H 'accept: application/json' 
     -H 'X-VirtShell-Authorization: UserId:Signature' \ 
     'http://<host>:<port>/api/virtshell/v1/partitions/development_co'
\end{lstlisting}

Response:

\begin{lstlisting}[style=json]
HTTP/1.1 200 OK
Content-Type: application/json
{
  "uuid": "efa1777c-cad7-11e5-9956-625662870761",
  "name": "backend_development_04",
  "description": "Servers for backend of the company", 
  "hosts": [ ... list of hosts associated with the section ...],  
  "created": {"at":"1429207233", "by":"1a900cdc-cad8-11e5-9956-625662870761"},
  "modified": {"at":"1529207233", "by":"2163b554-cad8-11e5-9956-625662870761"}
}
\end{lstlisting}

\paragraph{Obtener todas las particiones - GET /api/virtshell/v1/partitions} ~\\

\begin{lstlisting}[style=json]
curl -sv -H 'accept: application/json' 
     -H 'X-VirtShell-Authorization: UserId:Signature' \ 
     'http://localhost:8080/api/virtshell/v1/partitions'
\end{lstlisting}

Response:

\begin{lstlisting}[style=json]
HTTP/1.1 200 OK
Content-Type: application/json
{
  "partitions": [
    {
      "uuid": "ab8076c0-db91-11e2-82ce-0002a5d5c51b",
      "name": "development_co",
      "description": "Collection of servers oriented to development team in colombia.",
      "hosts": [ ... list of hosts associated with the section ...],
      "created": {"at":"1429207233", "by":"92d30f0c-8c9c-11e5-8994-feff819cdc9f"},
      "modified": {"at":"1529207233", "by":"92d31132-8c9c-11e5-8994-feff819cdc9f"}
    },
    { 
      "uuid": "efa1777c-cad7-11e5-9956-625662870761",
      "name": "production_us_miami",
      "description": "Collection of servers oriented to production in us.",
      "hosts": [ ... list of hosts associated with the section ...],      
      "created": {"at":"1429207233", "by":"1a900cdc-cad8-11e5-9956-625662870761"},
      "modified": {"at":"1529207233", "by":"2163b554-cad8-11e5-9956-625662870761"}
    }    
  ]
}  
\end{lstlisting}

\paragraph{Eliminar una partición - DELETE /api/virtshell/v1/partitions/:name} ~\\

\begin{lstlisting}[style=json]
curl -sv -X DELETE \
   -H 'accept: application/json' \
   -H 'X-VirtShell-Authorization: UserId:Signature' \
   'http://<host>:<port>/api/virtshell/v1/partitions/backend_development_04'
\end{lstlisting}

Response:

\begin{lstlisting}[style=json]
HTTP/1.1 200 OK
Content-Type: application/json
```
```json
{ "delete": "success" }
\end{lstlisting}

\paragraph{Agregar un host a una partición - PUT /api/virtshell/v1/partitions/:name/host/:hostname} ~\\

\begin{lstlisting}[style=json]
curl -sv -X PUT \
  -H 'accept: application/json' \
  -H 'X-VirtShell-Authorization: UserId:Signature' \
  'http://localhost:8080/virtshell/api/v1/partitions/:name/host/:hostname'
\end{lstlisting}

Response:

\begin{lstlisting}[style=json]
HTTP/1.1 200 OK
Content-Type: application/json

{ "add_host": "success" }
\end{lstlisting}
\subsection{Enviroments}
Representan subredes de trabajo más pequeñas asociadas a una partición. Los métodos soportados son:

\begin{center}
 \captionof{table}{Métodos HTTP para enviroments}
 \begin{tabular}{| l | l | l | l |}
 \hline
  \rowcolor{blueapi}
  \textbf{Acción} & \textbf{Método HTTP} & \textbf{Solicitud HTTP} & \textbf{Descripción} \\ [0.5ex] 
  \hline\hline
  get & GET & /enviroments/:name & Gets one enviroment by name. \\
  \hline
  list & GET & /enviroments & \pbox{5cm}{\vspace{0.2cm} Retrieves the list of \\ enviroments. \vspace{0.2cm}} \\
  \hline  
  create & POST & /enviroments/ & Inserts a new enviroment. \\
  \hline
  delete & DELETE & /enviroments/:name & Deletes an existing enviroment. \\
  \hline
\end{tabular}
\end{center}

Representaci'on del recurso de un ambiente:

\medskip
\begin{lstlisting}[style=json]
{
  "uuid": string,
  "name": string,
  "description": string, 
  "users": [ user_resource],
  "partition": string,
  "created": {"at": number, "by": string},
  "modified": {"at": number, "by": string}
}
\end{lstlisting}

Ejemplo:

\medskip
\begin{lstlisting}[style=json]
{
  "uuid": "ab8076c0-db91-11e2-82ce-0002a5d5c51b",
  "name": "bigdata_test_01",
  "description": "Collection of servers oriented to big data.", 
  "users": [ ... list of users allowed to use the enviroment ...],
  "partition": "partition associated with the enviroment",
  "created": {"at":"1429207233", "by":"92d30f0c-8c9c-11e5-8994-feff819cdc9f"},
  "modified": {"at":"1529207233", "by":"92d31132-8c9c-11e5-8994-feff819cdc9f"}
}
\end{lstlisting}

\subsubsection{Ejemplos de peticiones HTTP}

\paragraph{Crear un nuevo ambiente - POST /api/virtshell/v1/enviroments} ~\\

\begin{lstlisting}[style=json]
curl -sv -X POST \
  -H 'accept: application/json' \
  -H 'X-VirtShell-Authorization: UserId:Signature' \
  -d '{
       "name": "bigdata_test_01",
       "description": "Collection of servers oriented to big data.", 
       "users": [ ... list of users allowed to use the enviroment ...],
       "partition": "partition associated with the enviroment"
      }' \
   'http://localhost:8080/api/virtshell/v1/enviroments'
\end{lstlisting}

Response:

\begin{lstlisting}[style=json]
HTTP/1.1 200 OK
Content-Type: application/json
{ "create": "success" }
\end{lstlisting}

\paragraph{Obtener un ambiente- GET \\ /api/virtshell/v1/enviroments/:name} ~\\

\begin{lstlisting}[style=json]
curl -sv -H 'accept: application/json' 
     -H 'X-VirtShell-Authorization: UserId:Signature' \ 
     'http://<host>:<port>/api/virtshell/v1/enviroments/backend_development'
\end{lstlisting}

Response:

\begin{lstlisting}[style=json]
HTTP/1.1 200 OK
Content-Type: application/json
{
  "uuid": "efa1777c-cad7-11e5-9956-625662870761",
  "name": "backend_development",
  "description": "All backend of the company", 
  "users": [ ... list of users allowed to use the enviroment ...],
  "partition": "partition associated with the enviroment",
  "created": {"at":"1429207233", "by":"1a900cdc-cad8-11e5-9956-625662870761"},
  "modified": {"at":"1529207233", "by":"2163b554-cad8-11e5-9956-625662870761"}
}
\end{lstlisting}

\paragraph{Obtener todos los ambientes - GET \\ /api/virtshell/v1/enviroments} ~\\

\begin{lstlisting}[style=json]
curl -sv -H 'accept: application/json' 
     -H 'X-VirtShell-Authorization: UserId:Signature' \ 
     'http://localhost:8080/api/virtshell/v1/enviroments'
\end{lstlisting}

Response:

\begin{lstlisting}[style=json]
HTTP/1.1 200 OK
Content-Type: application/json
{
  "enviroments": [
    {
      "uuid": "ab8076c0-db91-11e2-82ce-0002a5d5c51b",
      "name": "bigdata_test_01",
      "description": "Collection of servers oriented to big data.", 
      "users": [ ... list of users allowed to use the enviroment ...],
      "partition": "partition associated with the enviroment",
      "created": {"at":"1429207233", "by":"92d30f0c-8c9c-11e5-8994-feff819cdc9f"},
      "modified": {"at":"1529207233", "by":"92d31132-8c9c-11e5-8994-feff819cdc9f"}
    },
    { 
      "uuid": "efa1777c-cad7-11e5-9956-625662870761",
      "name": "backend_development",
      "description": "All backend of the company", 
      "users": [ ... list of users allowed to use the enviroment ...],
      "partition": "partition associated with the enviroment",      
      "created": {"at":"1429207233", "by":"1a900cdc-cad8-11e5-9956-625662870761"},
      "modified": {"at":"1529207233", "by":"2163b554-cad8-11e5-9956-625662870761"}
    }    
  ]
}   
\end{lstlisting}

\paragraph{Eliminar un ambiente - DELETE \\ /api/virtshell/v1/enviroments/:name} ~\\

\begin{lstlisting}[style=json]
curl -sv -X DELETE \
   -H 'accept: application/json' \
   -H 'X-VirtShell-Authorization: UserId:Signature' \
   'http://<host>:<port>/api/virtshell/v1/enviroments/backend_development'
\end{lstlisting}

Response:

\begin{lstlisting}[style=json]
HTTP/1.1 200 OK
Content-Type: application/json
```
```json
{ "delete": "success" }
\end{lstlisting}
\subsection{Hosts}
Representan las m'aquinas f'sicas; un host es un anfitrion de maquinas virtuales o contenedores. Los metodos soportados son:

\begin{center}
 \begin{tabular}{| l | l | l | l |}
 \hline
  \rowcolor{blueapi}
  \textbf{Acci'on} & \textbf{Metodo HTTP} & \textbf{Solicitud HTTP} & \textbf{Descripci'on} \\ [0.5ex] 
  \hline\hline
  get & GET & /hosts/id & Gets one host by ID. \\
  \hline
  list & GET & /hosts & Retrieves the list of hosts. \\
  \hline  
  create & POST & /hosts/ & Inserts a new host configuration. \\
  \hline
  delete & DELETE & /hosts/id & Deletes an existing host. \\
  \hline  
  update & PUT & /hosts/id & Updates an existing host. \\ [1ex] 
  \hline
\end{tabular}
\end{center}

Representaci'on del recurso de un host:

\medskip
\begin{lstlisting}[style=json]
{
  "uuid": string,
  "name": string,
  "os": string,
  "memory": string,
  "capacity": string,
  "enabled": string,
  "type":string,
  "local_ipv4": string,
  "local_ipv6": string,
  "public_ipv4": string,
  "public_ipv6": string,
  "instances": [ instance_resource],
  "created":["at": number, "by": number]
}
\end{lstlisting}

Ejemplo:

\medskip
\begin{lstlisting}[style=json]
{
  "uuid": "ab8076c0-db91-11e2-82ce-0002a5d5c51b",
  "name": "host-01-pdn",
  "os": "Ubuntu_12.04_3.5.0-23.x86_64",
  "memory": "16GB",
  "capacity": "120GB",
  "enabled": "true|false",
  "type":"StorageOptimized|GeneralPurpose|HighPerformance",
  "local_ipv4": "15.54.88.19",
  "local_ipv6": "ff06:0:0:0:0:0:0:c3",
  "public_ipv4": "10.54.88.19",
  "public_ipv6": "yt06:0:0:0:0:0:0:c3",
  "instances": [
    ... instances resource is here
  ],
  "created":["at":"timestamp", "by":1234]
}
\end{lstlisting}

\subsubsection{Ejemplos de peticiones HTTP}

\paragraph{Crear un nuevo host - POST /virtshell/api/v1/hosts} ~\\

\begin{lstlisting}[style=json]
curl -sv -X POST \
  -H 'accept: application/json' \
    -H 'X-VirtShell-Authorization: UserId:Signature' \
  -d '{"name": "host-01-pdn",
       "os": "Ubuntu_12.04_3.5.0-23.x86_64",
       "memory": "16GB",
       "capacity": "120GB",
       "enabled": "true",
       "type" : "GeneralPurpose",
       "local_ipv4": "15.54.88.19",
         "local_ipv6": "ff06:0:0:0:0:0:0:c3",
       "public_ipv4": "10.54.88.19",
       "public_ipv6": "yt06:0:0:0:0:0:0:c3"}' \
   'http://localhost:8080/virtshell/api/v1/hosts'
\end{lstlisting}

Response:

\begin{lstlisting}[style=json]
HTTP/1.1 200 OK
Content-Type: application/json
{ "create": "success" }
\end{lstlisting}

\paragraph{Obtener un host- GET /virtshell/api/v1/hosts/:id} ~\\

\begin{lstlisting}[style=json]
curl -sv -H 'accept: application/json' 
     -H 'X-VirtShell-Authorization: UserId:Signature' \ 
     'http://localhost:8080/api/virtshell/v1/hosts?id=ab8076c0-db91-11e2-82ce-0002a5d5c51b'
\end{lstlisting}

Response:

\begin{lstlisting}[style=json]
HTTP/1.1 200 OK
Content-Type: application/json
{
  "uuid": "ab8076c0-db91-11e2-82ce-0002a5d5c51b",
  "name": "host-01-pdn",
  "os": "Ubuntu_12.04_3.5.0-23.x86_64",
  "memory": "16GB",
  "capacity": "120GB",
  "enabled": "true",
  "type" : "StorageOptimized",
  "local_ipv4": "15.54.88.19",
  "local_ipv6": "ff06:0:0:0:0:0:0:c3",
  "public_ipv4": "10.54.88.19",
  "public_ipv6": "yt06:0:0:0:0:0:0:c3",
  "instances": [
    {
      "name": "name1",
      "id": "72C05559-0590-4DA6-BE56-28AB36CB669C"
    },
    {
      "name": "name2",
      "id": "17173587-C4E9-4369-9C43-FCBF5E075973"
    }
  ],
  "created":["at":"20130625105211", "by":10]
}
\end{lstlisting}

\paragraph{Obtener todos los host - GET /virtshell/api/v1/hosts} ~\\

\begin{lstlisting}[style=json]
curl -sv -H 'accept: application/json' 
     -H 'X-VirtShell-Authorization: UserId:Signature' \ 
     'http://localhost:8080/api/virtshell/v1/hosts'
\end{lstlisting}

Response:

\begin{lstlisting}[style=json]
HTTP/1.1 200 OK
Content-Type: application/json
{
  "hosts": [
    {
      "uuid": "ab8076c0-db91-11e2-82ce-0002a5d5c51b",
      "name": "host-01-pdn",
      "os": "Ubuntu_12.04_3.5.0-23.x86_64",
      "memory": "16GB",
      "capacity": "120GB",
      "enabled": "true",
      "type" : "StorageOptimized",
      "local_ipv4": "15.54.88.19",
      "local_ipv6": "ff06:0:0:0:0:0:0:c3",
      "public_ipv4": "10.54.88.19",
      "public_ipv6": "yt06:0:0:0:0:0:0:c3",
      "instances": [
        {
          "name": "name1",
          "id": "72C05559-0590-4DA6-BE56-28AB36CB669C"
        },
        {
          "name": "name2",
          "id": "17173587-C4E9-4369-9C43-FCBF5E075973"
        }
      ],
      "created":["at":"20130625105211", "by":10]
    },
    {
      "uuid": "ab8076c0-db91-11e2-82ce-0002a5d5c51b",
      "name": "host-01-pdn",
      "os": "Ubuntu_12.04_3.5.0-23.x86_64",
      "memory": "16GB",
      "capacity": "120GB",
      "enabled": "true",
      "type" : "GeneralPurpose",
      "local_ipv4": "15.54.88.19",
      "local_ipv6": "ff06:0:0:0:0:0:0:c3",
      "public_ipv4": "10.54.88.19",
      "public_ipv6": "yt06:0:0:0:0:0:0:c3",
      "instances": [
        {
          "name": "name3",
          "id": "DE11CC9A-482F-4033-A7F8-503EE449DD0A"
        },
        {
          "name": "name4",
          "id": "17173587-C4E9-4369-9C43-FCBF5E075973"
        },    
      ],
      "created":["at":"20130625105211", "by":10]
    }
  ]
}   
\end{lstlisting}

\paragraph{Actualizar un host - PUT /virtshell/api/v1/hosts/:id} ~\\

\begin{lstlisting}[style=json]
curl -sv -X PUT \
  -H 'accept: application/json' \
    -H 'X-VirtShell-Authorization: UserId:Signature' \
  -d '{"memory": "24GB",
     "capacity": "750GB"}' \
   'http://localhost:8080/api/virtshell/v1/hosts?id=ab8076c0-db91-11e2-82ce-0002a5d5c51b'
\end{lstlisting}

Response:

\begin{lstlisting}[style=json]
HTTP/1.1 200 OK
Content-Type: application/json

{ "update": "success" }
\end{lstlisting}

\paragraph{Eliminar un host - DELETE /virtshell/api/v1/hosts/:id} ~\\

\begin{lstlisting}[style=json]
curl -sv -X DELETE \
   -H 'accept: application/json' \
   -H 'X-VirtShell-Authorization: UserId:Signature' \
   'http://localhost:8080/api/virtshell/v1/hosts?id=ab8076c0-db91-11e2-82ce-0002a5d5c51b'
\end{lstlisting}

Response:

\begin{lstlisting}[style=json]
HTTP/1.1 200 OK
Content-Type: application/json
```
```json
{ "delete": "success" }
\end{lstlisting}
\subsection{Instances}
Representan las instancias de las m'aquinas virtuales o los contenedores. Los métodos soportados son:

\begin{center}
 \captionof{table}{Métodos HTTP para instances}
 \begin{tabular}{| l | l | l | l |}
 \hline
  \rowcolor{blueapi}
  \textbf{Acci'on} & \textbf{Método HTTP} & \textbf{Solicitud HTTP} & \textbf{Descripci'on} \\ [0.5ex] 
  \hline\hline
  get & GET & /provisioners/:name & Gets one provisioner by ID. \\
  \hline
  list & GET & /provisioners & Retrieves the list of provisioners. \\
  \hline  
  create & POST & /provisioners/ & Creates a new provisioner. \\
  \hline
  delete & DELETE & /provisioners/:name & Deletes an existing host. \\ [1ex] 
  \hline
\end{tabular}
\end{center}

Representaci'on del recurso de un provisioner:

\medskip
\begin{lstlisting}[style=json]
{
  "uuid": string,
  "name": string,
  "description": string, 
  "enviroment": string,
  "provisioner": string,
  "host_type": string,
  "ipv4": string,
  "ipv6": string,
  "driver": string,
  "permissions": string,
  "created": {"at": timestamp, "by": string},
  "modified": {"at": timestamp, "by": string}
}
\end{lstlisting}

Ejemplo:

\medskip
\begin{lstlisting}[style=json]
{
  "uuid": "ab8076c0-db91-11e2-82ce-0002a5d5c51b",
  "name": "transactional_log",
  "description": "Server transactional only for store logs", 
  "enviroment": "Enviroment name to which it belongs",
  "provisioner": "all_backend",
  "host_type": "GeneralPurpose | ComputeOptimized | MemoryOptimized | StorageOptimized",
  "ipv4": "172.16.56.104",
  "ipv6": "FE80:0000:0000:0000:0202:B3FF:FE1E:8329",
  "driver": "lxc | virtualbox | vmware | ec2 | kvm | docker",
  "permissions": "xwrxwrxwr",
  "created": {"at":"1429207233", "by":"92d30f0c-8c9c-11e5-8994-feff819cdc9f"},
  "modified": {"at":"1529207233", "by":"92d31132-8c9c-11e5-8994-feff819cdc9f"}
}
\end{lstlisting}

\subsubsection{Ejemplos de peticiones HTTP}

\paragraph{Crear una nueva instance - POST /api/virtshell/v1/instances} ~\\


\begin{lstlisting}[style=json]
curl -sv -X POST \
  -H 'accept: application/json' \
  -H 'X-VirtShell-Authorization: UserId:Signature' \
  -d '{ "name": "transactional_log",
        "enviroment": "development_co",
        "description": "Server transactional only for store logs", 
        "provisioner": "all_backend",
        "host_type": "GeneralPurpose",
        "driver": "lxc"
      }' \
   'http://localhost:8080/virtshell/api/v1/instances'
\end{lstlisting}

Response:

\begin{lstlisting}[style=json]
HTTP/1.1 200 OK
Content-Type: application/json
{ "create": "in progress" }
\end{lstlisting}

\paragraph{Obtener un instance- GET /api/virtshell/v1/instances/:name} ~\\

\begin{lstlisting}[style=json]
curl -sv -H 'accept: application/json' 
     -H 'X-VirtShell-Authorization: UserId:Signature' \ 
     'http://<host>:<port>/api/virtshell/v1/instances/orders_colombia'
\end{lstlisting}

Response:

\begin{lstlisting}[style=json]
HTTP/1.1 200 OK
Content-Type: application/json
{
  "uuid": "ab8076c0-db91-11e2-82ce-0002a5d5c51b",
  "name": "transactional_log",
  "enviroment": "development_co",
  "description": "Server transactional only for store logs", 
  "provisioner": "all_backend",
  "host_type": "GeneralPurpose",
  "drive": "lxc",
  "created": {"at":"1429207233", "by":"92d30f0c-8c9c-11e5-8994-feff819cdc9f"},
  "modified": {"at":"1529207233", "by":"cf744732-8f12-11e5-8994-feff819cdc9f"}
  }
\end{lstlisting}

\paragraph{Obtener todos las instances - GET /api/virtshell/v1/instances} ~\\

\begin{lstlisting}[style=json]
curl -sv -H 'accept: application/json' 
     -H 'X-VirtShell-Authorization: UserId:Signature' \ 
     'http://localhost:8080/api/virtshell/v1/instances'
\end{lstlisting}

Response:

\begin{lstlisting}[style=json]
HTTP/1.1 200 OK
Content-Type: application/json
{
  "instances": [
    {
      "uuid": "ab8076c0-db91-11e2-82ce-0002a5d5c51b",
      "name": "transactional_log",
      "enviroment": "development_co",
      "description": "Server transactional only for store logs", 
      "provisioner": "all_backend",
      "host_type": "GeneralPurpose",
      "drive": "lxc",
      "permissions": "xwrxwrxwr",
      "created": {"at":"1429207233", "by":"92d30f0c-8c9c-11e5-8994-feff819cdc9f"},
      "modified": {"at":"1529207233", "by":"cf744732-8f12-11e5-8994-feff819cdc9f"}
    },
    { 
      "uuid": "cf744476-8f12-11e5-8994-feff819cdc9f",
      "name": "orders_colombia",
      "description": "Server transactional dedicated to receive orders", 
      "enviroment": "development_mx",
      "provisioner": "all_backend",
      "host_type": "StorageOptimized",
      "drive": "docker",
      "permissions": "xwrxwrxwr",
      "created": {"at":"1429207233", "by":"92d30f0c-8c9c-11e5-8994-feff819cdc9f"},
      "modified": {"at":"1529207233", "by":"92d31132-8c9c-11e5-8994-feff819cdc9f"}
    }    
  ]
} 
\end{lstlisting}

\paragraph{Eliminar una instance - DELETE /api/virtshell/v1/instances/:nae} ~\\

\begin{lstlisting}[style=json]
curl -sv -X DELETE \
   -H 'accept: application/json' \
   -H 'X-VirtShell-Authorization: UserId:Signature' \
   'http://<host>:<port>/api/virtshell/v1/instances/orders_colombia'
\end{lstlisting}

Response:

\begin{lstlisting}[style=json]
HTTP/1.1 200 OK
Content-Type: application/json
```
```json
{ "delete": "in progress" }
\end{lstlisting}
\subsection{Tasks}
Representan una tarea en VirtShell. Los métodos soportados son:

\begin{center}
 \captionof{table}{Métodos HTTP para tasks}
 \begin{tabular}{| l | l | l | l |}
 \hline
  \rowcolor{blueapi}
  \textbf{Acci'on} & \textbf{Metodo HTTP} & \textbf{Solicitud HTTP} & \textbf{Descripci'on} \\ [0.5ex] 
  \hline\hline
  get & GET & /tasks/:id & Gets one task by ID. \\
  \hline
  list & GET & /tasks & Retrieves the list of tasks. \\
  \hline
  get & GET & /tasks/status & Gets all task by status name. \\
  \hline 
  create & POST & /tasks/ & Creates a new task \\
  \hline  
  update & PUT & /tasks/:id & Updates an existing task. \\ [1ex] 
  \hline
\end{tabular}
\end{center}

Representaci'on del recurso de un task:

\medskip
\begin{lstlisting}[style=json]
{
  "uuid": string,
  "description": string,
  "status" : string,
  "type": string,
  "object_uuid": string,
  "created":["at":"timestamp", "by":string],
  "last_update": "timestamp",
  "log": string
}
\end{lstlisting}

Ejemplo:

\medskip
\begin{lstlisting}[style=json]
{
  "uuid": "ab8076c0-db91-11e2-82ce-0002a5d5c51b",
  "description": "clone virtual machine database_01",
  "status" : "pending|in progress|sucess|failed",
  "type": "create_instance|delete_instance|restart_instance|...",
  "object_uuid": "uuid of the object (instance, host, property, ...)",
  "created":["at":"timestamp", "by":user_id],
  "last_update": "timestamp",
  "log": "summary of the task"
}
\end{lstlisting}

\subsubsection{Ejemplos de peticiones HTTP}

\paragraph{Crear una nueva tarea - POST /api/virtshell/v1/tasks} ~\\

\begin{lstlisting}[style=json]
curl -sv -X POST \
  -H 'accept: application/json' \
    -H 'X-VirtShell-Authorization: UserId:Signature' \
  -d '{ "description": "clone virtual machine database_01",
        "status" : "in progress"}' \
   'http://localhost:8080/api/virtshell/v1/tasks'
\end{lstlisting}

Response:

\begin{lstlisting}[style=json]
HTTP/1.1 200 OK
Content-Type: application/json
{ "create": "success" }
\end{lstlisting}

\paragraph{Obtener una tarea- GET /api/virtshell/v1/tasks/:id} ~\\

\begin{lstlisting}[style=json]
curl -sv -H 'accept: application/json' 
     -H 'X-VirtShell-Authorization: UserId:Signature' \ 
     'http://<host>:<port>/api/virtshell/v1/tasks/ab8076c0-db91-11e2-82ce-0002a5d5c51b'
\end{lstlisting}

Response:

\begin{lstlisting}[style=json]
HTTP/1.1 200 OK
Content-Type: application/json
{
  "description": "clone virtual machine database_01",
  "status" : "in progress",
  "created": {"at":"1429207233", "by":"92d30f0c-8c9c-11e5-8994-feff819cdc9f"},
  "last_update": "1429207435",
  "log": "summary of the task"
}
\end{lstlisting}

\paragraph{Obtener una tarea de acuerdo a su status- GET /api/virtshell/v1/tasks/:status} ~\\

\begin{lstlisting}[style=json]
curl -sv -H 'accept: application/json' 
     -H 'X-VirtShell-Authorization: UserId:Signature' \ 
     'http://<host>:<port>/api/virtshell/v1/tasks/sucess'
\end{lstlisting}

Response:

\begin{lstlisting}[style=json]
HTTP/1.1 200 OK
Content-Type: application/json
{
  "tasks": [
    {
      "uuid": "a62ad146-ccf4-11e5-9956-625662870761",
      "description": "create container webserver_09",
      "status" : "sucess",
      "created": {"at":"1454433171", "by":"cc7f8e2c-ccf4-11e5-9956-625662870761"},
      "last_update": "1454436771",
      "log": "summary of the task"
    }
  ]
}
\end{lstlisting}

\paragraph{Obtener todas las tareas - GET /api/virtshell/v1/tasks} ~\\

\begin{lstlisting}[style=json]
curl -sv -H 'accept: application/json' 
     -H 'X-VirtShell-Authorization: UserId:Signature' \ 
     'http://<host>:<port>/api/virtshell/v1/tasks/'
\end{lstlisting}

Response:

\begin{lstlisting}[style=json]
HTTP/1.1 200 OK
Content-Type: application/json
{
  "tasks": [
    {
      "uuid": "ab8076c0-db91-11e2-82ce-0002a5d5c51b",
      "description": "clone virtual machine database_01",
      "status" : "in progress",
      "created": {"at":"1429207233", "by":"92d30f0c-8c9c-11e5-8994-feff819cdc9f"},
      "last_update": "1429207435",
      "log": "summary of the task"
    },
    {
      "uuid": "a62ad146-ccf4-11e5-9956-625662870761",
      "description": "create container webserver_09",
      "status" : "sucess",
      "created": {"at":"1454433171", "by":"cc7f8e2c-ccf4-11e5-9956-625662870761"},
      "last_update": "1454436771",
      "log": "summary of the task"
    }
  ]
}  
\end{lstlisting}

\paragraph{Actualizar una tarea - PUT /api/virtshell/v1/tasks/:id} ~\\

\begin{lstlisting}[style=json]
curl -sv -X PUT \
  -H 'accept: application/json' \
    -H 'X-VirtShell-Authorization: UserId:Signature' \
  -d '{"status": "sucess",
     "log": "....."}' \
   'http://localhost:8080/api/virtshell/v1/hosts/a62ad146-ccf4-11e5-9956-625662870761'
\end{lstlisting}

Response:

\begin{lstlisting}[style=json]
HTTP/1.1 200 OK
Content-Type: application/json

{ "update": "success" }
\end{lstlisting}

\subsection{Properties}
Representan propiedades de configuraci'on de las m'aquinas virtuales o contenedores. Los metodos soportados son:

\begin{center}
 \captionof{table}{Métodos HTTP para properties}
 \begin{tabular}{| l | l | l | l |}
 \hline
  \rowcolor{blueapi}
  \textbf{Acci'on} & \textbf{Metodo HTTP} & \textbf{Solicitud HTTP} & \textbf{Descripci'on} \\ [0.5ex] 
  \hline\hline
  get & GET & /properties/ & Install one or more packages. \\ [1ex] 
  \hline
\end{tabular}
\end{center}

\vspace{1cm}
Representaci'on del recurso de un paquete:
\vspace{1cm}

\begin{lstlisting}[style=json]
{
  "properties": [
      {"name": "propertie_name1"},
      {"name": "propertie_name2"}
  ],
  "hosts": [ 
      {"name": "Host_", "range": "[1-3]"}, 
      {"name": "database_001"}
  ],
  "tags": [
    {"name": "db"},
    {"name": "web"}
  ]
}
\end{lstlisting}

Ejemplo:

\medskip
\begin{lstlisting}[style=json]
{
  "properties": [
      {"name": "memory"},
      {"name": "cpu"}
  ],
  "hosts": [ 
      {"name": "Host_", "range": "[1-3]"}
  ]
}
\end{lstlisting}

\subsubsection{Ejemplos de peticiones HTTP}

\paragraph{Obtener una o mas propiedades de una unica instancia - POST /api/virtshell/v1/properties} ~\\

\begin{lstlisting}[style=json]
curl -sv -X GET \
  -H 'accept: application/json' \
  -H "Content-Type: text/plain" \
  -H 'X-VirtShell-Authorization: UserId:Signature' \
  -d '{ "properties": [{"name": "memory"}, {"name": "cpu"}],
        "hosts": [{"name": "WebServer"}]}' \
   'http://localhost:8080/api/virtshell/v1/properties'
\end{lstlisting}

\vspace{1cm}
Respuesta:
\vspace{1cm}

\begin{lstlisting}[style=json]
HTTP/1.1 202 OK
Content-Type: application/json
{
  "id": "kj5436c0-dc94-13tg-82ce-9992b5d5c51b",
  "name": "Database001",
  "memory": 1024
}
\end{lstlisting}

\paragraph{Obtener una o mas propiedades de una o mas instancias por tag - POST /api/virtshell/v1/properties} ~\\

\begin{lstlisting}[style=json]
curl -sv -X GET \
  -H 'accept: application/json' \
  -H "Content-Type: text/plain" \
  -H 'X-VirtShell-Authorization: UserId:Signature' \
  -d '{ "properties": [{"name": "memory"}, {"name": "cpu"}],
        "tag": [{"name": "web"}]}' \
   'http://localhost:8080/api/virtshell/v1/properties'
\end{lstlisting}

\vspace{1cm}
Respuesta:
\vspace{1cm}

\begin{lstlisting}[style=json]
HTTP/1.1 202 OK
Content-Type: application/json
{
  properties: [
    {
     "id": "kj5436c0-dc94-13tg-82ce-9992b5d5c51b",
     "name": "WebServerPhp001",
     "memory": 1024,
     "cpu": 2
    },
    {
     "id": "591b3828-7aaf-4833-a94c-ad0df44d59a4",
     "name": "WebServerPhp002",
     "memory": 1024,
     "cpu": 1  
    }
  ]
}
\end{lstlisting}

\paragraph{Obtener una o mas propiedades de una o mas instancias usando como prefijo un rango - POST /api/virtshell/v1/properties} ~\\

\begin{lstlisting}[style=json]
curl -sv -X GET \
  -H 'accept: application/json' \
  -H "Content-Type: text/plain" \
  -H 'X-VirtShell-Authorization: UserId:Signature' \
  -d '{ "properties": [{"name": "memory"}, {"name": "cpu"}],
        {"name": "Database00", "range": "[1-3]"}]}' \
   'http://localhost:8080/api/virtshell/v1/properties'
\end{lstlisting}

\vspace{1cm}
Respuesta:
\vspace{1cm}

\begin{lstlisting}[style=json]
HTTP/1.1 202 OK
Content-Type: application/json
{
  properties: [
    {
     "id": "kj5436c0-dc94-13tg-82ce-9992b5d5c51b",
     "name": "Database001",
     "memory": 4024,
     "cpu": 2
    },
    {
     "id": "591b3828-7aaf-4833-a94c-ad0df44d59a4",
     "name": "Database002",
     "memory": 4024,
     "cpu": 1  
    },
    {
     "id": "f7c81039-5c88-423b-8b0d-c124483d586b",
     "name": "Database003",
     "memory": 4024,
     "cpu": 3  
    }
  ]  
}
\end{lstlisting}

\subsection{Provisioners}
Representan los scripts que aprovisionan las m'aquinas virtuales o los contenedores. Los métodos soportados son:

\begin{center}
 \captionof{table}{Métodos HTTP para provisioners}
 \begin{tabular}{| l | l | l | l |}
 \hline
  \rowcolor{blueapi}
  \textbf{Acci'on} & \textbf{Metodo HTTP} & \textbf{Solicitud HTTP} & \textbf{Descripci'on} \\ [0.5ex] 
  \hline\hline
  get & GET & /provisioners/:name & Gets one provisioner by ID. \\
  \hline
  list & GET & /provisioners & Retrieves the list of provisioners. \\
  \hline  
  create & POST & /provisioners/ & Creates a new provisioner. \\
  \hline
  delete & DELETE & /provisioners/:name & Deletes an existing provisioner. \\
  \hline  
  update & PUT & /provisioners/:name & Updates an existing provisioner. \\ [1ex] 
  \hline
\end{tabular}
\end{center}

Representaci'on del recurso de un provisioner:

\medskip
\begin{lstlisting}[style=json]
{
  "uuid": string,
  "name": string,
  "description": string,
  "launch": number,
  "memory": number,
  "cpus": number,
  "hdsize": number,
  "image": string,
  "builder": string,
  "executor": string,
  "tag": string,
  "permissions": string,
  "depends": [ ... list of dependencies necessary for the builder ... ],
  "created": {"at":timestamp, "by":string},
  "modified": {"at":timestamp, "by":string}
}

\end{lstlisting}

Ejemplo:

\medskip
\begin{lstlisting}[style=json]
{
  "uuid": "ab8076c0-db91-11e2-82ce-0002a5d5c51b",
  "name": "backend-services-provisioner",
  "description": "Installs/Configures a backend server",
  "launch": 1,
  "memory": 4,
  "cpus": 2,
  "hdsize": 20,
  "image": "ubuntu_server_14.04.2_amd64",
  "builder": "https://github.com/janutechnology/VirtShell_Provisioners_Examples.git",
  "executor": "sh run1.sh",
  "tag": "backend",
  "permissions": "xwrxwrxwr",
  "depends": [ ... list of dependencies necessary for the builder ... ],
  "created": {"at":"1429207233", "by":"92d30f0c-8c9c-11e5-8994-feff819cdc9f"},
  "modified": {"at":"1529207233", "by":"92d31132-8c9c-11e5-8994-feff819cdc9f"}
}
\end{lstlisting}

\subsubsection{Ejemplos de peticiones HTTP}

\paragraph{Crear un nuevo provisioner - POST /api/virtshell/v1/provisioners} ~\\


\begin{lstlisting}[style=json]
curl -sv -X POST \
  -H 'accept: application/json' \
  -H 'X-VirtShell-Authorization: UserId:Signature' \
  -d '{"name": "backend-services-provisioner",
       "launch": 1,
       "memory": 4,
       "cpus": 2,
       "hdsize": 20,
       "image": "ubuntu_server_14.04.2_amd64",
       "driver": "docker",
       "builder": "https://github.com/janutechnology/VirtShell_Provisioners_Examples.git",
       "executor": "sh run1.sh",
       "tag": "backend",
       "permissions": "xwrxwrxwr",
       "depends": [
            {"provisioner_name": "db-users", "version": "2.0.0"},
            {"provisioner_name": "db-transactional"}
        ]
      }' \
   'http://localhost:8080/virtshell/api/v1/provisioners'
\end{lstlisting}

Response:

\begin{lstlisting}[style=json]
HTTP/1.1 200 OK
Content-Type: application/json
{ "create": "success" }
\end{lstlisting}

\paragraph{Obtener un provisioner- GET /api/virtshell/v1/provisioners/:name} ~\\

\begin{lstlisting}[style=json]
curl -sv -H 'accept: application/json' 
     -H 'X-VirtShell-Authorization: UserId:Signature' \ 
     'http://localhost:8080/api/virtshell/v1/provisioners/backend-services-provisioner'
\end{lstlisting}

Response:

\begin{lstlisting}[style=json]
HTTP/1.1 200 OK
Content-Type: application/json
  {
    "name": "backend-services-provisioner",
    "launch": 1,
    "memory": 4,
    "cpus": 2,
    "hdsize": 20,
    "image": "ubuntu_server_14.04.2_amd64",
    "driver": "docker",
    "permissions": "xwrxwrxwr",
    "builder": "https://github.com/janutechnology/VirtShell_Provisioners_Examples.git",
    "executor": "sh run1.sh",
    "tag": "backend",
    "depends": [
        {"provisioner_name": "db-users", "version": "2.0.0"},
        {"provisioner_name": "db-transactional"}
    ],
    "created": {"at":"1429207233", "by":"420aa2c4-8d96-11e5-8994-feff819cdc9f"},
    "modified": {"at":"1529207233", "by":"92d31132-8c9c-11e5-8994-feff819cdc9f"}    
  }
\end{lstlisting}

\paragraph{Obtener todos los provisioners - GET /api/virtshell/v1/provisioners} ~\\

\begin{lstlisting}[style=json]
curl -sv -H 'accept: application/json' 
     -H 'X-VirtShell-Authorization: UserId:Signature' \ 
     'http://localhost:8080/api/virtshell/v1/provisioners'
\end{lstlisting}

Response:

\begin{lstlisting}[style=json]
HTTP/1.1 200 OK
Content-Type: application/json
{
  "provisioners": [
    {
      "name": "backend-services-provisioner",
      "launch": 1,
      "memory": 4,
      "cpus": 2,
      "hdsize": 20,
      "image": "ubuntu_server_14.04.2_amd64",
      "driver": "docker",
      "builder": "https://github.com/janutechnology/VirtShell_Provisioners_Examples.git",
      "executor": "sh run1.sh",
      "tag": "backend",
      "permissions": "xwrxwrxwr",
      "depends": [
          {"provisioner_name": "db-users", "version": "2.0.0"},
          {"provisioner_name": "db-transactional"}
      ]
    },
    {
      "name": "db-transactional",
      "launch": 2,
      "memory": 8,
      "cpus": 2,
      "hdsize": 40,
      "image": "ubuntu_server_14.04.2_amd64",
      "driver": "docker",
      "builder": "https://github.com/janutechnology/VirtShell_Provisioners_Examples.git",
      "executor": "sh run_db.sh",
      "tag": "db",
      "permissions": "xwrxwrxwr"
    }
  ]
}
\end{lstlisting}

\paragraph{Actualizar un provisioner - PUT /api/virtshell/v1/provisioners/:name} ~\\

\begin{lstlisting}[style=json]
curl -sv -X PUT \
  -H 'accept: application/json' \
  -H 'X-VirtShell-Authorization: UserId:Signature' \
  -d '{ "executor": "run_backend.sh" }' \
   'http://localhost:8080/api/virtshell/v1/provisioners/backend-services-provisioner
\end{lstlisting}

Response:

\begin{lstlisting}[style=json]
HTTP/1.1 200 OK
Content-Type: application/json

{ "update": "success" }
\end{lstlisting}

\paragraph{Eliminar un provisioner - DELETE /api/virtshell/v1/provisioners/:name} ~\\

\begin{lstlisting}[style=json]
curl -sv -X DELETE \
   -H 'accept: application/json' \
   -H 'X-VirtShell-Authorization: UserId:Signature' \
   'http://localhost:8080/api/virtshell/v1/provisioners/backend-services-provisioner'
\end{lstlisting}

Response:

\begin{lstlisting}[style=json]
HTTP/1.1 200 OK
Content-Type: application/json
```
```json
{ "delete": "success" }
\end{lstlisting}
\subsection{Images}
Representan imagenes de m'aquinas virtuales o contenedores. Los métodos soportados son:

\begin{center}
 \captionof{table}{Métodos HTTP para images}
 \begin{tabular}{| l | l | l | l |}
 \hline
  \rowcolor{blueapi}
  \textbf{Acción} & \textbf{Método HTTP} & \textbf{Solicitud HTTP} & \textbf{Descripción} \\ [0.5ex] 
  \hline\hline
  get & GET & /images/:name & Gets one image by name. \\
  \hline
  list & GET & /images & Retrieves the list of images. \\
  \hline  
  create & POST & /images/ & Inserts a new image. \\
  \hline
  delete & DELETE & /images/:name & Deletes an existing image. \\ [1ex] 
  \hline
\end{tabular}
\end{center}

\vspace{1cm}
Representación del recurso de una imagen:
\vspace{1cm}

\begin{lstlisting}[style=json]
{
  "id": string,
  "name": string,
  "type": string,
  "os": string,
  "timezone": "America/Bogota", 
  "key": string,
  "preseed_url": url,
  "download_url": url,
  "permissions" : string,
  "created":["at": timestamp,"by": string],
  "details": string
}
\end{lstlisting}

Ejemplo:

\medskip
\begin{lstlisting}[style=json]
{
  "id": "kj5436c0-dc94-13tg-82ce-9992b5d5c51b",
  "name": "ubuntu_server_14.04.2_amd64",
  "type": "iso",
  "os": "ubuntu",
  "timezone": "America/Bogota",
  "preseed_url": "https://<host>:<port>/api/virtshell/v1/files/seeds/seed_ubuntu14-04.txt",
  "download_url": "http://releases.ubuntu.com/raring/ubuntu-14.04-server-amd64.iso",
  "permissions" : "rwxrw----",
  "details": "ubuntu-trusty, version: 14.04.2, amd64-server"
  "created":["at":"20150625105211","by":10]
}
\end{lstlisting}

\subsubsection{Ejemplos de peticiones HTTP}

\paragraph{Crear una nueva imagen - POST /virtshell/api/v1/images} ~\\

\begin{lstlisting}[style=json]
curl -sv -X PUT \
  -H 'accept: application/json' \
  -H "Content-Type: text/plain" \
  -H 'X-VirtShell-Authorization: UserId:Signature' \
  -d '{"name": "ubuntu_server_14.04.2_amd64",
     "type": "iso",
     "os": "ubuntu",
     "timezone": "America/Bogota", 
     "key": "/home/callanor/.ssh/id_rsa.pub",
     "permissions" : "rwxrwxr--",
     "preseed_url": "https://<host>:<port>/api/virtshell/v1/files/seeds/seed_ubuntu14-04.txt",
     "download_url": "http://releases.ubuntu.com/raring/ubuntu-14.04-server-amd64.iso"}' \
   'http://localhost:8080/api/virtshell/v1/image'
\end{lstlisting}

\vspace{1cm}
Respuesta:
\vspace{1cm}

\begin{lstlisting}[style=json]
HTTP/1.1 201 OK
Content-Type: application/json
{ "create": "success" }
\end{lstlisting}

\paragraph{Obtener una imagen - GET /virtshell/api/v1/images/:name} ~\\

\begin{lstlisting}[style=json]
curl -sv -H 'accept: application/json' 
     -H 'X-VirtShell-Authorization: UserId:Signature' \ 
     'http://localhost:8080/api/virtshell/v1/images/ubuntu_server_14.04.2_amd64'
\end{lstlisting}

\vspace{1cm}
Respuesta:
\vspace{1cm}

\begin{lstlisting}[style=json]
HTTP/1.1 200 OK
Content-Type: application/json
{
  "id": "kj5436c0-dc94-13tg-82ce-9992b5d5c51b",
  "name": "ubuntu_server_14.04.2_amd64",
  "type": "iso",
  "os": "ubuntu", 
  "timezone": "America/Bogota", 
  "preseed_url": "https://<host>:<port>/api/virtshell/v1/files/seeds/seed_ubuntu_14_04.txt",
  "download_url": "http://releases.ubuntu.com/raring/ubuntu-14.04-server-amd64.iso",
  "permissions" : "rwxrwxrwx",
  "created":["at":"20130625105211","by":10]
}
\end{lstlisting}

\paragraph{Obtener todas las imagenes - GET /virtshell/api/v1/images} ~\\

\begin{lstlisting}[style=json]
curl -sv -H 'accept: application/json' 
     -H 'X-VirtShell-Authorization: UserId:Signature' \ 
     'http://localhost:8080/api/virtshell/v1/images'
\end{lstlisting}

\vspace{1cm}
Respuesta:
\vspace{1cm}

\begin{lstlisting}[style=json]
HTTP/1.1 200 OK
Content-Type: application/json
{
  "images": [
    {
      "id": "b180ef2c-e798-4a8f-b23f-aaac2fb8f7e8",
      "name": "ubuntu_server_14.04.2_amd64",
      "type": "iso",
      "os": "ubuntu",  
      "timezone": "America/Bogota", 
      "preseed_file": "https://<host>:<port>/api/virtshell/v1/files/seeds/seed_file.txt",
      "download_url": "http://releases.ubuntu.com/raring/ubuntu-14.04-server-amd64.iso",
      "permissions" : "rwxrw----",
      "created":["at":"20130625105211","by":10]
    },
    {
      "id": "ca326181-bc84-4edb-bfc5-843037e7195e",
      "name": "centos:centos6",
      "type": "docker-container",
      "os": "centos", 
      "permissions" : "rwxrwxr--",
      "created":["at":"20140625105211","by":12]
    }
  ]
}  
\end{lstlisting}

\paragraph{Eliminar una imagen - DELETE \\ /virtshell/api/v1/images/:name} ~\\

\begin{lstlisting}[style=json]
curl -sv -X DELETE \
   -H 'accept: application/json' \
   -H 'X-VirtShell-Authorization: UserId:Signature' \
   'http://<host>:<port>/api/virtshell/v1/images/ubuntu_server_14.04.2_amd64'
\end{lstlisting}

\vspace{1cm}
Respuesta:
\vspace{1cm}

\begin{lstlisting}[style=json]
HTTP/1.1 200 OK
Content-Type: application/json
```
```json
{ "delete": "success" }
\end{lstlisting}

\subsection{Packages}
Representan paquetes de software que se ejecutan en las m'aquinas virtuales o contenedores. Los metodos soportados son:

\begin{center}
 \begin{tabular}{| l | l | l | l |}
 \hline
  \rowcolor{blueapi}
  \textbf{Acci'on} & \textbf{Metodo HTTP} & \textbf{Solicitud HTTP} & \textbf{Descripci'on} \\ [0.5ex] 
  \hline\hline
  install & POST & /install\_packages/ & Install one or more packages. \\
  \hline
  upgrade & POST & /upgrade\_packages/ & Upgrade one or more packages. \\
  \hline
  remove & POST & /remove\_packages/ & Remove one or more packages. \\ [1ex] 
  \hline
\end{tabular}
\end{center}

\vspace{1cm}
Representaci'on del recurso de un paquete:
\vspace{1cm}

\begin{lstlisting}[style=json]
{
  "packages": [
      {"name": "package_name1"},
      {"name": "package_name2"}
  ],
  "hosts": [ 
      {"name": "Host_", "range": "[1-3]"}, 
      {"name": "database_001"}
  ],
  "tags": [
    {"name": "db"},
    {"name": "web"}
  ]
}
\end{lstlisting}

Ejemplo:

\medskip
\begin{lstlisting}[style=json]
{
  "packages": [
      {"name": "git"},
      {"name": "nginx"}
  ],
  "hosts": [ 
      {"name": "Host_", "range": "[1-3]"}
  ]
}
\end{lstlisting}

\subsection{Ejemplos de peticiones HTTP}

\subsubsection{Instalar uno o mas paquetes - POST /virtshell/api/v1/install\_packages}

\begin{lstlisting}[style=json]
curl -sv -X PUT \
  -H 'accept: application/json' \
  -H "Content-Type: text/plain" \
  -H 'X-VirtShell-Authorization: UserId:Signature' \
  -d '{ "packages": [{"name": "git"}, {"name": "nginx"}],
        "hosts": [{"name": "WebServer_", "range": "[1-3]"}]}' \
   'http://localhost:8080/api/virtshell/v1/install_packages'
\end{lstlisting}

\vspace{1cm}
Respuesta:
\vspace{1cm}

\begin{lstlisting}[style=json]
HTTP/1.1 202 Accepted
Content-Type: application/json
{ "install_package": "accepted" }
\end{lstlisting}

\subsubsection{Actualizar uno o mas paquetes - POST /virtshell/api/v1/upgrade\_packages}

\begin{lstlisting}[style=json]
curl -sv -X PUT \
  -H 'accept: application/json' \
  -H "Content-Type: text/plain" \
  -H 'X-VirtShell-Authorization: UserId:Signature' \
  -d '{ "packages": [{"name": "git"}, {"name": "nginx"}, {"name": "mc"}],
        "hosts": [{"name": "WebServer_", "range": "[1-3]"}]}' \
   'http://localhost:8080/api/virtshell/v1/upgrade_packages'
\end{lstlisting}

\vspace{1cm}
Respuesta:
\vspace{1cm}

\begin{lstlisting}[style=json]
HTTP/1.1 202 Accepted
Content-Type: application/json
{ "install_package": "accepted" }
\end{lstlisting}

\subsubsection{Remover uno o mas paquetes - POST /virtshell/api/v1/remove\_packages}

\begin{lstlisting}[style=json]
curl -sv -X PUT \
  -H 'accept: application/json' \
  -H "Content-Type: text/plain" \
  -H 'X-VirtShell-Authorization: UserId:Signature' \
  -d '{ "packages": [{"name": "apache2"}],
        "hosts": [{"name": "WebServer_", "range": "[1-3]"}]}' \
   'http://localhost:8080/api/virtshell/v1/remove_packages'
\end{lstlisting}

\vspace{1cm}
Respuesta:
\vspace{1cm}

\begin{lstlisting}[style=json]
HTTP/1.1 202 Accepted
Content-Type: application/json
{ "install_package": "accepted" }
\end{lstlisting}
\subsection{Files}
Representan toda clase de archivos que se requieran para crear o aprovisionar m'aquinas virtuales o contenedores. Los metodos soportados son:

\begin{center}
 \begin{tabular}{| l | l | l | l |}
 \hline
  \rowcolor{blueapi}
  \textbf{Acci'on} & \textbf{Metodo HTTP} & \textbf{Solicitud HTTP} & \textbf{Descripci'on} \\ [0.5ex] 
  \hline\hline
  get & GET & /files/id & Gets one file by ID. \\
  \hline
  create & POST & /files/ & upload a new file. \\
  \hline
  delete & DELETE & /files/id & Deletes an existing file. \\
  \hline  
  update & PUT & /files/id & Updates an existing file. \\ [1ex]  
  \hline
\end{tabular}
\end{center}

\vspace{1cm}
Representaci'on del recurso de un archivo:
\vspace{1cm}

\begin{lstlisting}[style=json]
{
  "uuid": "ab8076c0-db91-11e2-82ce-0002a5d5c51b",
  "name": "file_name.extension",
  "folder_name" : "folder_name",
  "download_url": "https://<host>:<port>/api/virtshell/v1/files/folder_name/file.txt",
  "created":["at":"timestamp", "by":user_id]
}
\end{lstlisting}

Ejemplo:

\medskip
\begin{lstlisting}[style=json]
{
  "uuid": "ab8076c0-db91-11e2-82ce-0002a5d5c51b",
  "name": "ubuntu_seed_14-04.tex",
  "folder_name" : "ubuntu_seeds",
  "download_url": "https://<host>:<port>/api/virtshell/v1/files/ubuntu_seeds/ubuntu_seed_14-04.tex",
  "created": ["at":"20130625105211", "by":10]
}
\end{lstlisting}

\subsubsection{Ejemplos de peticiones HTTP}

\paragraph{Subir un nuevo archivo - POST /virtshell/api/v1/images} ~\\

\begin{lstlisting}[style=json]
curl -X POST \
  -H 'accept: application/json' \
  -H 'X-VirtShell-Authorization: UserId:Signature' \
  -H "Content-Type: multipart/form-data" \
  -F "file_data=@/path/to/file/seed_file.txt;filename=seed_file_ubuntu-14_04.txt" \
  -F "folder_name=ubuntu_seeds" \
  'http://<host>:<port>/api/virtshell/v1/files'
\end{lstlisting}

\vspace{1cm}
Respuesta:
\vspace{1cm}

\begin{lstlisting}[style=json]
HTTP/1.1 200 OK
Content-Type: application/json
{ 
  "create": "success",
  "location": "http://<host>:<port>/api/virtshell/v1/files/ubuntu_seeds/seed_file_ubuntu-14_04.txt" 
}
\end{lstlisting}

\paragraph{Obtener un archivo - GET /virtshell/api/v1/files/:id} ~\\

Para descargar un archivo, primero recibira la url apropiada que viene en la metadata provista por la url. Luego podra descargarlo usando la url.

\begin{lstlisting}[style=json]
curl -sv -H 'accept: application/json' 
     -H 'X-VirtShell-Authorization: UserId:Signature' \ 
     'http://<host>:<port>/api/virtshell/v1/files/?id=ab8076c0-db91-11e2-82ce-0002a5d5c51b'
\end{lstlisting}

\vspace{1cm}
Respuesta:
\vspace{1cm}

\begin{lstlisting}[style=json]
HTTP/1.1 200 OK
Content-Type: application/json
{
  "uuid": "ab8076c0-db91-11e2-82ce-0002a5d5c51b",
  "name": "file_name.extension",
  "folder_name" : "folder_name",
  "download_url": "http://<host>:<port>/api/virtshell/v1/files/ubuntu_seeds/seed_file_ubuntu-14_04.txt",
  "created":["at":"timestamp", "by":user_id] 
}
\end{lstlisting}

\paragraph{Actualizar un archivo - PUT /virtshell/api/v1/files/:id} ~\\

\begin{lstlisting}[style=json]
curl -sv -X PUT \
  -H 'accept: application/json' \
  -H 'X-VirtShell-Authorization: UserId:Signature' \
  -H "Content-Type: multipart/form-data" \
  -F "file_data=@/path/to/file/seed_file.txt;filename=seed_file_ubuntu-14_04_v2.txt" \
   'http://localhost:8080/api/virtshell/v1/file?id=8de7b824-d7d1-4265-a3a6-5b46cc9b8ed5'
\end{lstlisting}

\vspace{1cm}
Respuesta:
\vspace{1cm}

\begin{lstlisting}[style=json]
HTTP/1.1 200 OK
Content-Type: application/json

{ "update": "success" }
\end{lstlisting}


\paragraph{Eliminar un archivo - DELETE /virtshell/api/v1/files/:id} ~\\

\begin{lstlisting}[style=json]
curl -sv -X DELETE \
   -H 'accept: application/json' \
   -H 'X-VirtShell-Authorization: UserId:Signature' \
   'http://localhost:8080/api/virtshell/v1/fles?id=ab8076c0-db91-11e2-82ce-0002a5d5c51b'
\end{lstlisting}

\vspace{1cm}
Respuesta:
\vspace{1cm}

\begin{lstlisting}[style=json]
HTTP/1.1 200 OK
Content-Type: application/json
```
```json
{ "delete": "success" }
\end{lstlisting}



\section{API Calls}

\subsection{Start Instance}

Permite iniciar una instancia.

\paragraph{Iniciar una instance - \\ POST /virtshell/api/v1/instances/start\_instance/:id} ~\\


\begin{lstlisting}[style=json]
curl -sv -X POST \
  -H 'accept: application/json' \
  -H 'X-VirtShell-Authorization: UserId:Signature' \
   'http://localhost:8080/virtshell/api/v1/instances/start\_instance/420aa3f0-8d96-11e5-8994-feff819cdc9f'
\end{lstlisting}

Response:

\begin{lstlisting}[style=json]
HTTP/1.1 200 OK
Content-Type: application/json
{ "start": "success" }
\end{lstlisting}


\subsection{Stop Instance}

Permite detener una instancia.

\paragraph{Detener una instancia - \\ POST /virtshell/api/v1/instances/stop\_instance/:id} ~\\

\begin{lstlisting}[style=json]
curl -sv -X POST \
  -H 'accept: application/json' \
  -H 'X-VirtShell-Authorization: UserId:Signature' \
   'http://localhost:8080/virtshell/api/v1/instances/stop\_instance/420aa3f0-8d96-11e5-8994-feff819cdc9f'
\end{lstlisting}

Response:

\begin{lstlisting}[style=json]
HTTP/1.1 200 OK
Content-Type: application/json
{ "stop": "success" }
\end{lstlisting}


\subsection{Restart Instance}

Permite reiniciar una instancia.

\paragraph{Reiniciar una instancia - \\ POST /virtshell/api/v1/instances/restart\_instance/:id} ~\\

\begin{lstlisting}[style=json]
curl -sv -X POST \
  -H 'accept: application/json' \
  -H 'X-VirtShell-Authorization: UserId:Signature' \
   'http://localhost:8080/virtshell/api/v1/instances/restart\_instance/420aa3f0-8d96-11e5-8994-feff819cdc9f'
\end{lstlisting}

Response:

\begin{lstlisting}[style=json]
HTTP/1.1 200 OK
Content-Type: application/json
{ "restart": "success" }
\end{lstlisting}


\subsection{Clone Instance}

Permite clonar una instancia.

\paragraph{Clonar una instancia - \\ POST /virtshell/api/v1/instances/clone\_instance/:id} ~\\

\begin{lstlisting}[style=json]
curl -sv -X POST \
  -H 'accept: application/json' \
  -H 'X-VirtShell-Authorization: UserId:Signature' \
   'http://localhost:8080/virtshell/api/v1/instances/clone\_instance/420aa3f0-8d96-11e5-8994-feff819cdc9f'
\end{lstlisting}

Response:

\begin{lstlisting}[style=json]
HTTP/1.1 200 OK
Content-Type: application/json
{ "clone": "success" }
\end{lstlisting}

\subsection{Execute command}

Permite ejecutar un comando en una o mas instancias.

Representaci'on del recurso para ejecutar un comando:

\medskip
\begin{lstlisting}[style=json]
{
  "instances": [ ... list of instances names, patterns(*|[numeric:numeric]) or tags ...],
  "command": string,
  "created": {"at": timestamp, "by": string}
}
\end{lstlisting}

Ejemplo:

\medskip
\begin{lstlisting}[style=json]
{
  "instances": [
      {"name": "database\_server\_01"},
      {"name": "transactional\_server\_co"},      
      {"pattern": "web\_server*"},
      {"pattern": "grid\_[1:5]"},
      {"tag": "web"}
  ],
  "command": "apt-get upgrade",
  "created": {"at": timestamp, "by": string}
}
\end{lstlisting}

\paragraph{Ejecutar un comando en una o mas instancias - \\ POST /virtshell/api/v1/instances/execute\_command/} ~\\

\begin{lstlisting}[style=json]
curl -sv -X POST \
  -H 'accept: application/json' \
  -H 'X-VirtShell-Authorization: UserId:Signature' \
  -d '{ "instances": [
          {"name": "database\_server\_01"},
          {"name": "transactional\_server\_co"},          
          {"pattern": "web\_server*"},
          {"pattern": "grid\_server\_[1:5]"},
          {"tag": "web"}
        ],
        "command": "apt-get upgrade" }' \
  'http://localhost:8080/virtshell/api/v1/instances/execute\_command/'
\end{lstlisting}

Response:

\begin{lstlisting}[style=json]
HTTP/1.1 200 OK
Content-Type: application/json
{ "execute_command": "success" }
\end{lstlisting}


\subsection{Copy files}

Permite ejecutar copiar uno archivo en una o mas instancias.

Representaci'on del recurso para ejecutar un comando:

\medskip
\begin{lstlisting}[style=json]
{
  "path": string,
  "destination": string,
  "instances": [ ... list of instances names, patterns(*|[numeric:numeric]) or tags ...],
  "created": {"at": timestamp, "by": string}
}
\end{lstlisting}

Ejemplo:

\medskip
\begin{lstlisting}[style=json]
{
  "uuid_file": "0d832c60-7066-4d37-bd72-ce6ac4f61bcc",
  "destination": "$MYSQL_HOME/my.cnf"
  "instances": [
      {"name": "database\_server\_01"},
      {"name": "web\_server*"},
      {"name": "grid\_[1:5]"},
      {"name": "transactional\_server\_co"},
      {"tag": "web"}
  ]
}
\end{lstlisting}

\paragraph{Copiar un archivo en una o mas instancias - \\ POST /virtshell/api/v1/instances/copy\_files/} ~\\

\begin{lstlisting}[style=json]
curl -sv -X POST \
  -H 'accept: application/json' \
  -H 'X-VirtShell-Authorization: UserId:Signature' \
  -d '{ "uuid_file": "0d832c60-7066-4d37-bd72-ce6ac4f61bcc",
        "destination": "$MYSQL_HOME/my.cnf"
        "instances": [
            {"name": "database\_server\_01"},
            {"name": "web\_server*"},
            {"name": "grid\_[1:5]"},
            {"name": "transactional\_server\_co"},
            {"tag": "web"}
        ] }' \
  'http://localhost:8080/virtshell/api/v1/instances/copy\_files/'
\end{lstlisting}

Response:

\begin{lstlisting}[style=json]
HTTP/1.1 200 OK
Content-Type: application/json
{ "copy_files": "success" }
\end{lstlisting}


%\include{capitulo3}
\chapter{Recomendaciones}
\label{caprecomendaciones}

\begin{itemize}
\item Dise~nar e implementar politicas de seguridad para la administraci'on de los archivos.
\item Implementar una interfaz web que permita administrar los ambientes y maquinas virtuales.
\item Implementar los agentes de monitoreo de recursos.
\item Implementar algun mecanismo de seguridad que permita revisar las tramas que llegan y salen de las maquinas virtuales y los hosts.
\item Realizar un plan de pruebas funcionales para los ambientes que se aprovisionan.
\end{itemize}



\appendix
%% Cap'itulos incluidos despues del comando \appendix aparecen como ap'endices
%% de la tesis.
%\include{apendiceA}
%\include{apendiceB}
%\include{apendiceC}

%% Incluir la bibliograf'ia. Mirar el archivo "biblio.bib" para m'as detales
%% y un ejemplo.
\bibliography{biblio}

\end{document}
