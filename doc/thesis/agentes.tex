\chapter{Agentes}
\label{capagents}

En este capítulo se definen los elementos que apoyan a VirtShell a realizar su trabajo en cada uno de los anfitriones. Estos elementos son llamados agentes.\\
\\
Los agentes son servicios que se ejecutan localmente en cada anfitrión. VirtShell instala y configura los agentes en cada uno de los anfitriones de manera automática, liberando de esta tarea al administrador del sistema. Existen tres tipos de agentes: de aprovisionamiento, monitoreo y administración.

\begin{description}
\item [Agente de Aprovisionamiento]
El agente de aprovisionamiento se encarga de instalar y configurar: paquetes, librerías y aplicaciones en las instancias. 
\item [Agente de Monitoreo]
El agente de monitoreo es completamente autónomo y sus funciones consisten en supervisar al agente de aprovisionamiento y reportar el estado de salud de los anfitriones e instancias que se encuentran creados en su interior. 
\item [Agente de Administración]
El agente de administración se encarga de gestionar las instancias, permitiendo detener, iniciar, clonar, eliminar y obtener información especifica de cada instancia que se ejecuta en el anfitrión.
\end{description}