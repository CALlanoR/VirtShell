\chapter{Aprovisionamiento}
\label{capaprovisionamiento}

VirtShell cuenta con una colección de recursos que permite el aprovisionamiento de instancias,independientemente de la infraestructura de visualización que se use para estas. El conjunto de recursos de esta capa, esta diseñado para que sea una solución sencilla de usar, fiable y repetible, con una curva de aprendizaje muy baja para los administradores, desarrolladores y administradores de TI. Este capítulo busca describir todo lo que se requiere para llevar a cabo un correcto aprovisionamiento en las instancias creadas a través del sistema.

%%%%%%%%%%%%%%%%%%%%%%%%%%%%%%%%%%%%%%%%%%%%%%%%%%%%
%%%%%%%% Almacenamiento y envio de archivos %%%%%%%%
%%%%%%%%%%%%%%%%%%%%%%%%%%%%%%%%%%%%%%%%%%%%%%%%%%%%
\section{Almacenamiento y envio de archivos}
El módulo de archivos proporciona una manera de almacenar archivos en VirtShell y de enviarlos a uno mas instancias. Para enviar archivos a las instancias, estos deben estar almacenados previamente en el sistema. En la petición HTTP de envió de archivos se debe especificar la ruta donde se encuentra el archivo, en la maquina en la que se esta invocando el API, asimismo, la carpeta donde desea guardarse y los permisos con que contara el archivo. Si la carpeta no existe, el sistema la creara automáticamente. A continuación se muestra un ejemplo de como enviar un archivo a VirtShell.

\vspace{5mm}

\begin{lstlisting}[style=json, caption=Petición HTTP para subir un archivo al sistema]
curl -X POST \
  -H 'accept: application/json' \
  -H 'X-VirtShell-Authorization: UserId:Signature' \
  -H "Content-Type: multipart/form-data" \
  -F "file_data=@/path/to/file/seed_file.txt;filename=seed_file_ubuntu-14_04.txt" \
  -F "folder_name=ubuntu_seeds" \
  -F "permissions=rwxrwx---",
  'http://<host>:<port>/api/virtshell/v1/files'
\end{lstlisting}

\vspace{5mm}

La respuesta que retorna VirtShell, muestra la url en donde almaceno el archivo y el identificador UUID asignado en el sistema. Un ejemplo se muestra a continuación:

\vspace{5mm}

\begin{lstlisting}[style=json, caption=Respuesta HTTP del almacenamiento de un archivo.]
HTTP/1.1 200 OK
Content-Type: application/json
{ 
  "create": "success",
  "location": "http://<host>:<port>/api/virtshell/v1/files/ubuntu_seeds/seed_file_ubuntu-14_04.txt",
  "uuid": "51702a78-b23c-4625-bdda-3704cd0924b8" 
}
\end{lstlisting}

\vspace{5mm}

Existen dos formas para el envió de archivos a las instancias, la primera consiste en enviar un archivo a un destino en una o mas instancias, como se muestra a continuación:

\vspace{5mm}

\begin{lstlisting}[style=json, caption=Ejemplo del envio de un archivo a una instancia.]
curl -sv -X POST \
  -H 'accept: application/json' \
  -H 'X-VirtShell-Authorization: UserId:Signature' \
  -d '{ "uuid_file": "0d832c60-7066-4d37-bd72-ce6ac4f61bcc",
        "destination": "$MYSQL_HOME/my.cnf"
        "instances": [
            {"name": "database\_server\_01"}
        ] }' \
  'http://localhost:8080/virtshell/api/v1/instances/copy\_files/'
\end{lstlisting}

\vspace{5mm}

Y la segunda forma se trata de enviar el contenido una carpeta del sistema, en donde pueden existir uno o mas archivos a una carpeta destino, especificando las instancias a las cuales sera enviado. Un ejemplo se muestra a continuación:

\vspace{5mm}

\begin{lstlisting}[style=json]
curl -sv -X POST \
  -H 'accept: application/json' \
  -H 'X-VirtShell-Authorization: UserId:Signature' \
  -d '{ "folder_name": "mysql_configuration",
        "destination": "$MYSQL_HOME/my.cnf"
        "instances": [
            {"name": "database\_server\_01"},
            {"name": "database\_server\_02"},
            {"name": "database\_server\_03"}
        ] }' \
  'http://localhost:8080/virtshell/api/v1/instances/copy\_files/'
\end{lstlisting}

\vspace{5mm}

El envió de archivos a las instancias genera una tarea en el sistema. La respuesta se da la siguiente manera:

\vspace{5mm}

\begin{lstlisting}[style=json, caption=Ejemplo de respuesta HTTP para el envio de archivos]
HTTP/1.1 200 OK
Content-Type: application/json
{ "copy-files": "in progress", "task": "49d0cca2-e297-47c4-976f-845e1ce8f475" }
\end{lstlisting}

%%%%%%%%%%%%%%%%%%%%%%%%%%
%%%%%%%% Imagenes %%%%%%%%
%%%%%%%%%%%%%%%%%%%%%%%%%%
\section{Imágenes}
Como se describió en la arquitectura, las imágenes que se manejan en VirtShell son de dos tipos: ISO y \emph{templates} para contenedores. A continuación se presentan las características de cada una de ellas.1

\subsection{Imágenes tipo ISO}
Las imágenes ISO son usadas para la creación de maquinas virtuales que interactuan con un hipervisor. Estas son versiones modificadas que se encuentran en los repositorios oficiales de las distintas distribuciones de Linux o de cualquier otro sistema operativo. Para que una imagen pueda ser usada en VirtShell, esta debe cumplir con los siguientes requisitos: 
\begin{itemize}
\item Que su proceso de instalación sea completamente desatendido (o automático), esto quiere decir que no requiera intervención humana.
\item Debe tener instalado un servidor SSH con un usuario valido para VirtShell.
\end{itemize}

VirtShell cuenta con un servicio que permite crear versiones desatendidas de la distribución de Linux Ubuntu. Para crear una versión desatendida es necesario proveer un mecanismo para responder a las preguntas formuladas durante el proceso de instalación, sin tener que introducir manualmente las respuestas mientras la instalación está transcurriendo. Esto se hace posible creando un archivo de pre-configuración (\emph{pressed file}) con las respuestas y solicitud de instalación de paquetes como el del servidor SSH. Este archivo debe ser enviado al servidor de VirtShell a través del API REST. Previamente en la sección de archivos en este mismo capítulo se mostró como almacenar un archivo en el sistema. Una vez el archivo de respuestas se encuentra en VirtShell, se enviá la petición HTTP al API solicitando la creación de la nueva imagen desatendida como se muestra a continuación:

\vspace{5mm}

\begin{lstlisting}[style=json, caption=Petición HTTP para crear una imagen desatendida]
curl -sv -X PUT \
  -H 'accept: application/json' \
  -H "Content-Type: text/plain" \
  -H 'X-VirtShell-Authorization: UserId:Signature' \
  -d '{"name": "ubuntu_server_14.04.2_amd64",
       "type": "iso",
       "os": "ubuntu", 
       "timezone": "America/Bogota", 
       "permissions" : "rwxrwxr--",
       "preseed_url": "https://<host>:<port>/api/virtshell/v1/files/seeds/seed_ubuntu14-04.txt",
       "download_url": "http://releases.ubuntu.com/raring/ubuntu-14.04-server-amd64.iso"}' \
   'http://localhost:8080/api/virtshell/v1/image'
\end{lstlisting}

\vspace{5mm}

Como se observa en la anterior petición, para crear una imagen desatendida, ademas de especificar el archivo de respuestas, se debe definir la url (en el campo download\_url) donde se encuentra la imagen ISO que el servicio de VirtShell usara para la creación. Un ejemplo de un archivo de respuestas se encuentra en el apéndice C.\\
\\
La creación de una imagen desatendida es un proceso que puede tomar algunos minutos en completarse, como se mencionó en el capitulo de administración, VirtShell crea una tarea asíncrona retornando un identificador que permite consultar el estado de la creación de la imagen más adelante. La respuesta HTTP seria de la siguiente forma:

\vspace{5mm}

\begin{lstlisting}[style=json, caption=Ejemplo de respuesta HTTP para la creación de una imagen]
HTTP/1.1 200 OK
Content-Type: application/json
{ "create-image": "in progress", "task": "84de8acd-6095-40b1-a318-e7c6706213a3" }
\end{lstlisting}

\vspace{5mm}

\subsection{Imágenes tipo contenedor}
Este tipo de imágenes son usadas para la creación de contenedores que interactuan directamente con el sistema operativo. Se encuentran divididas en dos subtipos, las imágenes para contenedores creados con tecnología LXC, y las imágenes para contenedores creados con tecnología Docker.\\
\\
La Creación de un contenedor LXC, en general, implica la creación de un sistema de ficheros raíz propio para el contenedor. LXC delega este trabajo a \emph{templates} (plantillas), las cuales pueden ser encontradas en las distribuciones de Linux bajo el directorio /usr/share/lxc/templates, e incluyen \emph{templates} para crear contenedores con los sistemas operativos Ubuntu, Debian, Fedora, Oracle, Centos y Gentoo entre otros \cite{lxcubuntu16}. La idea con estos \emph{templates} es modificarlos de acuerdo a las necesidades del usuario y subirlos al sistema por medio del API REST,  una vez el \emph{template} se encuentre en el sistema, se debe crear como una imagen disponible para contenedores LXC como se muestra a continuación:

\vspace{5mm}

\begin{lstlisting}[style=json, caption=Petición HTTP para crear una imagen para contenedores LXC]
curl -sv -X PUT \
  -H 'accept: application/json' \
  -H "Content-Type: text/plain" \
  -H 'X-VirtShell-Authorization: UserId:Signature' \
  -d '{"name": "Oracle",
       "type": "lxc-container",
       "os": "oracle",
       "permissions" : "rwxrwxr--",
       "download_url": "https://<host>:<port>/api/virtshell/v1/files/images/3514296#file-lxc-oracle}' \
   'http://localhost:8080/api/virtshell/v1/image'
\end{lstlisting}

\vspace{5mm}

Como se observa en la petición HTTP, se debe especificar que la imagen es de tipo lxc-container y precisar la ubicación del template en el sistema. La creación de este tipo de imágenes no genera una tarea, dado que el template ya esta creado previamente por el usuario.\\
\\
La Creación de contenedores usando Docker, requiere de imágenes que se encuentran en repositorios públicos administrados por el equipo de Docker \footnote{Repositorio oficial de imágenes Docker: https://github.com/docker-library/official-images}. En un repositorio de imágenes Docker, las imágenes son guardadas por medio de la combinación del nombre del sistema operativo y un \emph{tag} (etiqueta) suministrado por el creador de la imagen, lo que permite contar con mas de una imagen por sistema operativo. La especificación de una imagen Docker en VirtShell, se hace de la siguiente manera:

\vspace{5mm}

\begin{lstlisting}[style=json, caption=Petición HTTP para crear una imagen para contenedores Docker]
curl -sv -X PUT \
  -H 'accept: application/json' \
  -H "Content-Type: text/plain" \
  -H 'X-VirtShell-Authorization: UserId:Signature' \
  -d '{"name": "centos:centos6",
       "type": "docker-container",
       "os": "centos",
       "permissions" : "rwxrwxr--"}' \
   'http://localhost:8080/api/virtshell/v1/image'
\end{lstlisting}

\vspace{5mm}

En la petición HTTP, con el fin de mantener el estándar con los repositorios Docker, el nombre de la imagen en VirtShell debe ser el mismo que se maneja en los repositorios públicos, esto es, la combinación del nombre del sistema operativo y el tab dado por el creador, separados por el carácter dos puntos (:). La creación de una imagen Docker en VirtShell genera una tarea, la cual consiste en verificar la existencia de la imagen en el repositorio público. Si la imagen es válida, su referencia será agregada al sistema para que el agente de aprovisionamiento sea el que la descargue del repositorio público, en caso contrario, se notificará la razón del error en la tarea. Un ejemplo del mensaje que retorna la creación de una imagen Docker se muestra a continuación:

\vspace{5mm}

\begin{lstlisting}[style=json, caption=Ejemplo de respuesta HTTP para la creación de una imagen]
HTTP/1.1 200 OK
Content-Type: application/json
{ "create-image": "checking", "task": "44ab19d5-7967-4bb9-b39e-f61f983af72b" }
\end{lstlisting}


%%%%%%%%%%%%%%%%%%%%%%%%%%%%%%%%%%
%%%%%%%% Aprovisionadores %%%%%%%%
%%%%%%%%%%%%%%%%%%%%%%%%%%%%%%%%%%
\section{Aprovisionadores}
Este modulo permite definir la configuración necesaria para crear una nueva instancia y aprovisionarla. Cuando se crea un aprovisionador se debe seleccionar una imagen del repositorio, las características de memoria RAM, numero de CPUs y el tamaño en disco que se espera tenga la instancia. \\
\\
De igual manera el aprovisionador cuenta con un elemento fundamental para el aprovisionamiento de las instancias, 


Ideas:
- Redactados usando scripts deshell.
- Debe definir todo lo que se requiere para configurar el sistema
- Se almacena en un repositorio git
- Puede tener dependencia de uno mas aprovisionadores
- Puede etiquetar una instancia para facilitar busquedas por medio de agrupaciones
- Se ejecuta por medio del executor
- Define un escenario y contiene todo lo que se requiere para apoyar ese escenario
  - Especifican los paquetes que se van a instalar y configurar y el orden en que han de ser aplicados
  - Los valores de otras dependencias
  - Plantillas






%%%%%%%%%%%%%%%%%%%%%%%%%%%%%%%%%%%%%%%%%%%%%%%%%%%%%%%%%
%%%%%%%% Instalación y Actualización de paquetes %%%%%%%%
%%%%%%%%%%%%%%%%%%%%%%%%%%%%%%%%%%%%%%%%%%%%%%%%%%%%%%%%%
\section{Instalación y Actualización de paquetes}




