\subsection{Packages}
Representan paquetes de software que se ejecutan en las m'aquinas virtuales o contenedores. Los metodos soportados son:

\begin{center}
 \begin{tabular}{| l | l | l | l |}
 \hline
  \rowcolor{blueapi}
  \textbf{Acci'on} & \textbf{Metodo HTTP} & \textbf{Solicitud HTTP} & \textbf{Descripci'on} \\ [0.5ex] 
  \hline\hline
  install & POST & /install\_packages/ & Install one or more packages. \\
  \hline
  upgrade & POST & /upgrade\_packages/ & Upgrade one or more packages. \\
  \hline
  remove & POST & /remove\_packages/ & Remove one or more packages. \\ [1ex] 
  \hline
\end{tabular}
\end{center}

\vspace{1cm}
Representaci'on del recurso de un paquete:
\vspace{1cm}

\begin{lstlisting}[style=json]
{
  "packages": [
      {"name": "package_name1"},
      {"name": "package_name2"}
  ],
  "hosts": [ 
      {"name": "Host_", "range": "[1-3]"}, 
      {"name": "database_001"}
  ],
  "tags": [
    {"name": "db"},
    {"name": "web"}
  ]
}
\end{lstlisting}

Ejemplo:

\medskip
\begin{lstlisting}[style=json]
{
  "packages": [
      {"name": "git"},
      {"name": "nginx"}
  ],
  "hosts": [ 
      {"name": "Host_", "range": "[1-3]"}
  ]
}
\end{lstlisting}

\subsubsection{Ejemplos de peticiones HTTP}

\paragraph{Instalar uno o mas paquetes - POST /virtshell/api/v1/install\_packages} ~\\

\begin{lstlisting}[style=json]
curl -sv -X PUT \
  -H 'accept: application/json' \
  -H "Content-Type: text/plain" \
  -H 'X-VirtShell-Authorization: UserId:Signature' \
  -d '{ "packages": [{"name": "git"}, {"name": "nginx"}],
        "hosts": [{"name": "WebServer_", "range": "[1-3]"}]}' \
   'http://localhost:8080/api/virtshell/v1/install_packages'
\end{lstlisting}

\vspace{1cm}
Respuesta:
\vspace{1cm}

\begin{lstlisting}[style=json]
HTTP/1.1 202 Accepted
Content-Type: application/json
{ "install_package": "accepted" }
\end{lstlisting}

\paragraph{Actualizar uno o mas paquetes - POST /virtshell/api/v1/upgrade\_packages} ~\\

\begin{lstlisting}[style=json]
curl -sv -X PUT \
  -H 'accept: application/json' \
  -H "Content-Type: text/plain" \
  -H 'X-VirtShell-Authorization: UserId:Signature' \
  -d '{ "packages": [{"name": "git"}, {"name": "nginx"}, {"name": "mc"}],
        "hosts": [{"name": "WebServer_", "range": "[1-3]"}]}' \
   'http://localhost:8080/api/virtshell/v1/upgrade_packages'
\end{lstlisting}

\vspace{1cm}
Respuesta:
\vspace{1cm}

\begin{lstlisting}[style=json]
HTTP/1.1 202 Accepted
Content-Type: application/json
{ "install_package": "accepted" }
\end{lstlisting}

\paragraph{Remover uno o mas paquetes - POST /virtshell/api/v1/remove\_packages} ~\\

\begin{lstlisting}[style=json]
curl -sv -X PUT \
  -H 'accept: application/json' \
  -H "Content-Type: text/plain" \
  -H 'X-VirtShell-Authorization: UserId:Signature' \
  -d '{ "packages": [{"name": "apache2"}],
        "hosts": [{"name": "WebServer_", "range": "[1-3]"}]}' \
   'http://localhost:8080/api/virtshell/v1/remove_packages'
\end{lstlisting}

\vspace{1cm}
Respuesta:
\vspace{1cm}

\begin{lstlisting}[style=json]
HTTP/1.1 202 Accepted
Content-Type: application/json
{ "install_package": "accepted" }
\end{lstlisting}