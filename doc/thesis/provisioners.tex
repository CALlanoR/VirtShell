\subsection{Provisioners}
Representan los scripts que aprovisionan las m'aquinas virtuales o los contenedores. Los métodos soportados son:

\begin{center}
 \captionof{table}{Métodos HTTP para provisioners}
 \begin{tabular}{| l | l | l | l |}
 \hline
  \rowcolor{blueapi}
  \textbf{Acci'on} & \textbf{Método HTTP} & \textbf{Solicitud HTTP} & \textbf{Descripci'on} \\ [0.5ex] 
  \hline\hline
  get & GET & /provisioners/:name & Gets one provisioner by ID. \\
  \hline
  list & GET & /provisioners & Retrieves the list of provisioners. \\
  \hline  
  create & POST & /provisioners/ & Creates a new provisioner. \\
  \hline
  delete & DELETE & /provisioners/:name & Deletes an existing provisioner. \\
  \hline  
  update & PUT & /provisioners/:name & Updates an existing provisioner. \\ [1ex] 
  \hline
\end{tabular}
\end{center}

Representaci'on del recurso de un provisioner:

\medskip
\begin{lstlisting}[style=json]
{
  "uuid": string,
  "name": string,
  "description": string,
  "launch": number,
  "memory": number,
  "cpus": number,
  "hdsize": number,
  "image": string,
  "builder": string,
  "executor": string,
  "tag": string,
  "permissions": string,
  "depends": [ ... list of dependencies necessary for the builder ... ],
  "created": {"at":timestamp, "by":string},
  "modified": {"at":timestamp, "by":string}
}

\end{lstlisting}

Ejemplo:

\medskip
\begin{lstlisting}[style=json]
{
  "uuid": "ab8076c0-db91-11e2-82ce-0002a5d5c51b",
  "name": "backend-services-provisioner",
  "description": "Installs/Configures a backend server",
  "launch": 1,
  "memory": 4,
  "cpus": 2,
  "hdsize": 20,
  "image": "ubuntu_server_14.04.2_amd64",
  "builder": "https://github.com/janutechnology/VirtShell_Provisioners_Examples.git",
  "executor": "sh run1.sh",
  "tag": "backend",
  "permissions": "xwrxwrxwr",
  "depends": [ ... list of dependencies necessary for the builder ... ],
  "created": {"at":"1429207233", "by":"92d30f0c-8c9c-11e5-8994-feff819cdc9f"},
  "modified": {"at":"1529207233", "by":"92d31132-8c9c-11e5-8994-feff819cdc9f"}
}
\end{lstlisting}

\subsubsection{Ejemplos de peticiones HTTP}

\paragraph{Crear un nuevo provisioner - POST /api/virtshell/v1/provisioners} ~\\


\begin{lstlisting}[style=json]
curl -sv -X POST \
  -H 'accept: application/json' \
  -H 'X-VirtShell-Authorization: UserId:Signature' \
  -d '{"name": "backend-services-provisioner",
       "launch": 1,
       "memory": 4,
       "cpus": 2,
       "hdsize": 20,
       "image": "ubuntu_server_14.04.2_amd64",
       "driver": "docker",
       "builder": "https://github.com/janutechnology/VirtShell_Provisioners_Examples.git",
       "executor": "sh run1.sh",
       "tag": "backend",
       "permissions": "xwrxwrxwr",
       "depends": [
            {"provisioner_name": "db-users", "version": "2.0.0"},
            {"provisioner_name": "db-transactional"}
        ]
      }' \
   'http://localhost:8080/virtshell/api/v1/provisioners'
\end{lstlisting}

Response:

\begin{lstlisting}[style=json]
HTTP/1.1 200 OK
Content-Type: application/json
{ "create": "success" }
\end{lstlisting}

\paragraph{Obtener un provisioner- GET /api/virtshell/v1/provisioners/:name} ~\\

\begin{lstlisting}[style=json]
curl -sv -H 'accept: application/json' 
     -H 'X-VirtShell-Authorization: UserId:Signature' \ 
     'http://localhost:8080/api/virtshell/v1/provisioners/backend-services-provisioner'
\end{lstlisting}

Response:

\begin{lstlisting}[style=json]
HTTP/1.1 200 OK
Content-Type: application/json
  {
    "name": "backend-services-provisioner",
    "launch": 1,
    "memory": 4,
    "cpus": 2,
    "hdsize": 20,
    "image": "ubuntu_server_14.04.2_amd64",
    "driver": "docker",
    "permissions": "xwrxwrxwr",
    "builder": "https://github.com/janutechnology/VirtShell_Provisioners_Examples.git",
    "executor": "sh run1.sh",
    "tag": "backend",
    "depends": [
        {"provisioner_name": "db-users", "version": "2.0.0"},
        {"provisioner_name": "db-transactional"}
    ],
    "created": {"at":"1429207233", "by":"420aa2c4-8d96-11e5-8994-feff819cdc9f"},
    "modified": {"at":"1529207233", "by":"92d31132-8c9c-11e5-8994-feff819cdc9f"}    
  }
\end{lstlisting}

\paragraph{Obtener todos los provisioners - GET /api/virtshell/v1/provisioners} ~\\

\begin{lstlisting}[style=json]
curl -sv -H 'accept: application/json' 
     -H 'X-VirtShell-Authorization: UserId:Signature' \ 
     'http://localhost:8080/api/virtshell/v1/provisioners'
\end{lstlisting}

Response:

\begin{lstlisting}[style=json]
HTTP/1.1 200 OK
Content-Type: application/json
{
  "provisioners": [
    {
      "name": "backend-services-provisioner",
      "launch": 1,
      "memory": 4,
      "cpus": 2,
      "hdsize": 20,
      "image": "ubuntu_server_14.04.2_amd64",
      "driver": "docker",
      "builder": "https://github.com/janutechnology/VirtShell_Provisioners_Examples.git",
      "executor": "sh run1.sh",
      "tag": "backend",
      "permissions": "xwrxwrxwr",
      "depends": [
          {"provisioner_name": "db-users", "version": "2.0.0"},
          {"provisioner_name": "db-transactional"}
      ]
    },
    {
      "name": "db-transactional",
      "launch": 2,
      "memory": 8,
      "cpus": 2,
      "hdsize": 40,
      "image": "ubuntu_server_14.04.2_amd64",
      "driver": "docker",
      "builder": "https://github.com/janutechnology/VirtShell_Provisioners_Examples.git",
      "executor": "sh run_db.sh",
      "tag": "db",
      "permissions": "xwrxwrxwr"
    }
  ]
}
\end{lstlisting}

\paragraph{Actualizar un provisioner - PUT /api/virtshell/v1/provisioners/:name} ~\\

\begin{lstlisting}[style=json]
curl -sv -X PUT \
  -H 'accept: application/json' \
  -H 'X-VirtShell-Authorization: UserId:Signature' \
  -d '{ "executor": "run_backend.sh" }' \
   'http://localhost:8080/api/virtshell/v1/provisioners/backend-services-provisioner
\end{lstlisting}

Response:

\begin{lstlisting}[style=json]
HTTP/1.1 200 OK
Content-Type: application/json

{ "update": "success" }
\end{lstlisting}

\paragraph{Eliminar un provisioner - DELETE /api/virtshell/v1/provisioners/:name} ~\\

\begin{lstlisting}[style=json]
curl -sv -X DELETE \
   -H 'accept: application/json' \
   -H 'X-VirtShell-Authorization: UserId:Signature' \
   'http://localhost:8080/api/virtshell/v1/provisioners/backend-services-provisioner'
\end{lstlisting}

Response:

\begin{lstlisting}[style=json]
HTTP/1.1 200 OK
Content-Type: application/json
```
```json
{ "delete": "success" }
\end{lstlisting}