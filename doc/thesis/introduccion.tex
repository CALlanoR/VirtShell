\chapter{Introducci'on}

La aparición de ambientes de computación centrados en la nube, los cuales se caracterizan por ofrecer servicios bajo demanda, ha favorecido el desarrollo de diversas herramientas que apoyan los procesos de aprovisionamiento en demanda de servicios y ambientes de computación orientados al procesamiento de tareas de larga duración y manejo de grandes volúmenes de datos. Estos ambientes dinámicos de computación son desarrollados mayormente a través de técnicas de programación ágil las cuales se caracterizan por ofrecer rápidos resultados e integración a gran escala de componentes de software. Es así como los equipos de DevOps \footnote{DevOps consiste en traer las prácticas del desarrollo ágil a la administración de sistema y el trabajo en conjunto entre desarrolladores y administradores de sistemas. DevOps no es una descripción de cargo o el uso de herramientas, sino un método de trabajo enfocado a resultados.} se convierten en un elemento fundamental ya que potencia la estabilidad y uniformidad de los distintos ambientes de prueba y producción de modo que los procesos de integración y despliegue se hagan de forma automatizada. \\
\\
Las herramientas de aprovisionamiento automático de infraestructura son el eje central de estos equipos ya que es a través de ellas que el personal de desarrollo y operaciones son capaces de hablar un mismo lenguaje y establecer los requerimientos y necesidades a satisfacer. Sin embargo, las herramientas actuales de aprovisionamiento adolecen de servicios que faciliten la especificación de infraestructura a través de un API \footnote{ API: Application Programming Interface, conjunto de subrutinas, funciones y procedimientos que ofrece un software para ser utilizado por otro software como una capa de abstracción.} estandarizado que posibilite la orquestación del despliegue de infraestructura a través de Internet.\\
\\
En este documento se presenta una herramienta de aprovisionamiento con orientación a servicios que permite el despliegue y orquestación de plataformas y servicios a través de un API RESTful \footnote{RESTful hace referencia a un servicio web que implementa la arquitectura REST}. Además de lo mencionado anteriormente, la tesis consta de 8 capítulos más y de 3 apéndices.\\
\\
El segundo capítulo presenta el planteamiento del problema, en donde se aclara el objeto y alcance del trabajo realizado. La definición del marco teórico que permite entender la importancia de la virtualización en la actualidad, las técnicas de virtualización usadas y la sinopsis de las soluciones de aprovisionamiento mas conocidas actualmente, son presentadas en el tercer capítulo.\\
\\
En el cuarto capitulo se introduce y elabora la arquitectura planteada en VirtShell. Se describe los requisitos que se tuvieron en cuenta para elaborar la estructura del framework, las alternativas estudiadas y se reseña los módulos y sus características que conforman a VirtShell.\\
\\
Los siguientes 4 capítulos se encargan de describir cada uno de los módulos diseñados, ilustrando sus funcionalidades y la forma en que interactúan de manera conjunta para administrar la infraestructura y realizar el aprovisionamiento de los recursos virtualizados. Adicionalmente se muestran ejemplos del uso del API.\\
\\
En el ultimo capitulo se muestra de forma detallada la documentación del API de VirtShell. Se indican los recursos y los métodos HTTP con que cuenta cada módulo, de igual forma se enseñan mas ejemplos de como interactuar con el API.\\
\\
Finalmente, los apéndices son utilizados para presentar información relacionada con la implementación del framework.