\subsection{Start Instance}

Permite iniciar una instancia.

\paragraph{Iniciar una instance - \\ POST /virtshell/api/v1/instances/start\_instance/:id} ~\\


\begin{lstlisting}[style=json]
curl -sv -X POST \
  -H 'accept: application/json' \
  -H 'X-VirtShell-Authorization: UserId:Signature' \
   'http://localhost:8080/virtshell/api/v1/instances/start\_instance/420aa3f0-8d96-11e5-8994-feff819cdc9f'
\end{lstlisting}

Response:

\begin{lstlisting}[style=json]
HTTP/1.1 200 OK
Content-Type: application/json
{ "start": "success" }
\end{lstlisting}


\subsection{Stop Instance}

Permite detener una instancia.

\paragraph{Detener una instancia - \\ POST /virtshell/api/v1/instances/stop\_instance/:id} ~\\

\begin{lstlisting}[style=json]
curl -sv -X POST \
  -H 'accept: application/json' \
  -H 'X-VirtShell-Authorization: UserId:Signature' \
   'http://localhost:8080/virtshell/api/v1/instances/stop\_instance/420aa3f0-8d96-11e5-8994-feff819cdc9f'
\end{lstlisting}

Response:

\begin{lstlisting}[style=json]
HTTP/1.1 200 OK
Content-Type: application/json
{ "stop": "success" }
\end{lstlisting}


\subsection{Restart Instance}

Permite reiniciar una instancia.

\paragraph{Reiniciar una instancia - \\ POST /virtshell/api/v1/instances/restart\_instance/:id} ~\\

\begin{lstlisting}[style=json]
curl -sv -X POST \
  -H 'accept: application/json' \
  -H 'X-VirtShell-Authorization: UserId:Signature' \
   'http://localhost:8080/virtshell/api/v1/instances/restart\_instance/420aa3f0-8d96-11e5-8994-feff819cdc9f'
\end{lstlisting}

Response:

\begin{lstlisting}[style=json]
HTTP/1.1 200 OK
Content-Type: application/json
{ "restart": "success" }
\end{lstlisting}


\subsection{Clone Instance}

Permite clonar una instancia.

\paragraph{Clonar una instancia - \\ POST /virtshell/api/v1/instances/clone\_instance/:id} ~\\

\begin{lstlisting}[style=json]
curl -sv -X POST \
  -H 'accept: application/json' \
  -H 'X-VirtShell-Authorization: UserId:Signature' \
   'http://localhost:8080/virtshell/api/v1/instances/clone\_instance/420aa3f0-8d96-11e5-8994-feff819cdc9f'
\end{lstlisting}

Response:

\begin{lstlisting}[style=json]
HTTP/1.1 200 OK
Content-Type: application/json
{ "clone": "success" }
\end{lstlisting}

\subsection{Execute command}

Permite ejecutar un comando en una o mas instancias.

Representaci'on del recurso para ejecutar un comando:

\medskip
\begin{lstlisting}[style=json]
{
  "instances": [ ... list of instances names, patterns(*|[numeric:numeric]) or tags ...],
  "command": string,
  "created": {"at": timestamp, "by": string}
}
\end{lstlisting}

Ejemplo:

\medskip
\begin{lstlisting}[style=json]
{
  "instances": [
      {"name": "database\_server\_01"},
      {"name": "transactional\_server\_co"},      
      {"pattern": "web\_server*"},
      {"pattern": "grid\_[1:5]"},
      {"tag": "web"}
  ],
  "command": "apt-get upgrade",
  "created": {"at": timestamp, "by": string}
}
\end{lstlisting}

\paragraph{Ejecutar un comando en una o mas instancias - \\ POST /virtshell/api/v1/instances/execute\_command/} ~\\

\begin{lstlisting}[style=json]
curl -sv -X POST \
  -H 'accept: application/json' \
  -H 'X-VirtShell-Authorization: UserId:Signature' \
  -d '{ "instances": [
          {"name": "database\_server\_01"},
          {"name": "transactional\_server\_co"},          
          {"pattern": "web\_server*"},
          {"pattern": "grid\_server\_[1:5]"},
          {"tag": "web"}
        ],
        "command": "apt-get upgrade" }' \
  'http://localhost:8080/virtshell/api/v1/instances/execute\_command/'
\end{lstlisting}

Response:

\begin{lstlisting}[style=json]
HTTP/1.1 200 OK
Content-Type: application/json
{ "execute_command": "success" }
\end{lstlisting}


\subsection{Copy files}

Permite ejecutar copiar uno archivo en una o mas instancias.

Representaci'on del recurso para ejecutar un comando:

\medskip
\begin{lstlisting}[style=json]
{
  "path": string,
  "destination": string,
  "instances": [ ... list of instances names, patterns(*|[numeric:numeric]) or tags ...],
  "created": {"at": timestamp, "by": string}
}
\end{lstlisting}

Ejemplo:

\medskip
\begin{lstlisting}[style=json]
{
  "path": "https://<host>:<port>/api/virtshell/v1/files/database_servers/msql/my.cnf",
  "destination": "$MYSQL_HOME/my.cnf"
  "instances": [
      {"name": "database\_server\_01"},
      {"name": "web\_server*"},
      {"name": "grid\_[1:5]"},
      {"name": "transactional\_server\_co"},
      {"tag": "web"}
  ]
}
\end{lstlisting}

\paragraph{Copiar un archivo en una o mas instancias - \\ POST /virtshell/api/v1/instances/copy\_files/} ~\\

\begin{lstlisting}[style=json]
curl -sv -X POST \
  -H 'accept: application/json' \
  -H 'X-VirtShell-Authorization: UserId:Signature' \
  -d '{ "path": "https://<host>:<port>/api/virtshell/v1/files/database_servers/msql/my.cnf",
        "destination": "$MYSQL_HOME/my.cnf"
        "instances": [
            {"name": "database\_server\_01"},
            {"name": "web\_server*"},
            {"name": "grid\_[1:5]"},
            {"name": "transactional\_server\_co"},
            {"tag": "web"}
        ] }' \
  'http://localhost:8080/virtshell/api/v1/instances/copy\_files/'
\end{lstlisting}

Response:

\begin{lstlisting}[style=json]
HTTP/1.1 200 OK
Content-Type: application/json
{ "copy_files": "success" }
\end{lstlisting}
