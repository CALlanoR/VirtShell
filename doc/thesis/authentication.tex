\chapter{Autenticaci'on y Protecci'on}
\label{capauthentication}

\section{Autenticaci'on}
La autenticaci'on es el proceso de demostrar la identidad al sistema. La identidad es un factor importante en las decisiones de control de acceso. Las solicitudes se conceden o deniegan en parte sobre la base de la identidad del solicitante.\\
\\
El VirtShell, el API REST utiliza un esquema HTTP personalizado basado en una llave-HMAC (Hash Message Authentication Code) para la autenticaci'on. Para autenticar una solicitud, primero se concatenan los elementos seleccionados de la solicitud para formar una cadena. A continuaci'on, utiliza una clave secreta de acceso para calcular el HMAC de esa cadena. Informalmente, se le denomina a este proceso \"la firma de la solicitud\", y se denomina a la salida del algoritmo HMAC la "firma", ya que simula las propiedades de seguridad de una firma real. Por 'ultimo, se agrega esta firma como un par'ametro de la petici'on, con la sintaxis descrita en esta secci'on.\\
\\
Cuando el sistema recibe una solicitud fehaciente, se obtiene la clave secreta de acceso que dicen tener, y lo utiliza de la misma manera que se calcula una "firma" del mensaje que recibi'o. A continuaci'on, compara la firma que se calcula con la firma presentada por el solicitante. Si las dos firmas coinciden, el sistema llega a la conclusi'on de que el solicitante debe tener acceso a la clave secreta de acceso, y por lo tanto act'ua con la autoridad del principal al que se emiti'o la clave. Si las dos firmas no coinciden, la solicitud se descarta y el sistema responde con un mensaje de error.\\
\\
Ejemplo de una petici'on autenticada:

\medskip
\begin{lstlisting}
  GET /api/virtshell/packages/{packageId} HTTP/1.1
  Host: host1.edu.co
  Date: Fri, 01 Jul 2011 19:37:58 +0000

  Authorization: 0PN5J17HBGZHT7JJ3X82:frJIUN8DYpKDtOLCwo//yllqDzg= 
\end{lstlisting}

\subsection{Authentication Header}

El API REST utiliza el encabezado de autorizaci'on HTTP para pasar informaci'on de autenticaci'on. Bajo el esquema de autenticaci'on de VirtShell, el encabezado de autorizaci'on tiene la siguiente forma.

\medskip
\begin{lstlisting}
  Authorization: UserId:Signature
\end{lstlisting}
\medskip

Los usuarios tendr'an un ID de clave de acceso (VirtShell Access Key ID) y una clave secreta de acceso (VirtShell Secret Access Key) cuando se registran. Para la petici'on de autenticaci'on, el elemento de VirtShell Access Key Id identifica la clave secreta que se utiliz'o para calcular la firma, y (indirectamente) el usuario que realiza la solicitud.\\
\\
Para la firma de los elementos de la petici'on se usa el RFC 2104HMAC-SHA1, por lo que la parte de la firma de la cabecera autorizaci'on variar'a de una petici'on a otra. Si la solicitud de la firma calculada por el sistema coincide con la firma incluida en la solicitud, el solicitante habr'a demostrado la posesi'on de la clave secreta de acceso. La solicitud ser'a procesada bajo la identidad, y con la autoridad, de la promotora que se emiti'o la clave.\\
\\
A continuaci'on se muestra la pseudo-gram'atica que ilustra la construcci'on de la cabecera de la solicitud de autorizaci'on (
\textbackslash{}n significa el punto de c'odigo Unicode U +000 A com'unmente llamado salto de l'inea).

\medskip
\begin{lstlisting}[basicstyle=\tiny]
  Authorization = VirtShellUserId + ":" + Signature;

  Signature = Base64( HMAC-SHA1( UTF-8-Encoding-Of( YourSecretAccessKeyID, StringToSign ) ) );

  StringToSign = HTTP-Verb + "\n" +
  Host + "\n" +
  Content-MD5 + "\n" +
  Content-Type + "\n" +
  Date + "\n" +
  CanonicalizedResource;

  CanonicalizedResource = <HTTP-Request-URI, from the protocol name up to the query string (resource path)>
\end{lstlisting}

HMAC-SHA1 es un algoritmo definido por la RFC 2104 (ver la RFC 2104 con llave Hashing para la autenticaci'on de mensajes).\\
\\
El algoritmo toma como entrada dos cadenas de bytes: una clave y un mensaje. Para la solicitud de autenticaci'on, se utiliza la clave secreta (YourSecretAccessKeyID) como la clave, y la codificaci'on UTF-8 del StringToSign como el mensaje. La salida de HMAC-SHA1 es tambi'en una cadena de bytes, llamado el resumen. El par'ametro de la petici'on de la Firma se construye codificada en Base64.

\subsection{Solicitud can'onica para firmar}

Cuando el sistema recibe una solicitud autenticada, compara la solicitud de firma calculada con la firma proporcionada en la solicitud de StringToSign. Por esta raz'on, se debe calcular la firma con el mismo m'etodo utilizado por VirtShell. A este proceso de poner una solicitud en una forma acordada para la firma se denomino "canonizaci'on".

\subsection{Tiempo de sello}

Un sello de tiempo v'alido (utilizando el HTTP header Date) es obligatorio para solicitudes autenticadas. Por otra parte, el tiempo del sello enviado por un usuario que se encuentra incluido en una solicitud autenticada debe estar dentro de los 15 minutos de la hora del sistema cuando se recibe la solicitud. En caso contrario, la solicitud fallar'a con el c'odigo de estado de error RequestTimeTooSkewed. La intenci'on de estas restricciones es limitar la posibilidad de que solicitudes interceptadas pueden ser reproducidos por un adversario.Para una mayor protecci'on contra las escuchas, se debe utilizar el transporte HTTPS para solicitudes autenticadas.

\subsection{Ejemplos de autenticaci'on}

\scriptsize
\begin{tabular}{|l|l|} \hline
\textbf{Parametro} & \textbf{Valor} \\ \hline
VirtShellUserId  & 13010f3e-3f46-4889-b989-592ce8fb30c6 \\ \hline
\multicolumn{1}{|m{3.5cm}|}{VirtShellSecretAccessKey} & 
\multicolumn{1}{m{9cm}|} {
                            \raggedright c991f519-bed0-4dab-9165-6d3f722dc845 \\
                            \textbf{Base64:} \\ Yzk5MWY1MTktYmVkMC00ZGFiLTkxNjUtNmQ5ZjcyMmRjODQ1
                          } \tabularnewline \hline
\end{tabular}
\normalsize

\subsubsection{Ejemplo de un objeto con GET}

Este es un ejemplo que consulta por un host dado su identificador. \\
\\
\vspace{1cm}
\scriptsize
\begin{tabular}{|l|l|} \hline
\textbf{Request} & \textbf{StringToSign} \\ \hline
\multicolumn{1}{|m{7.5cm}|}{
      \raggedright GET /api/virtshell/hosts/5713b48a-8d73-11e5-8994-feff819cdc9f HTTP/1.1 \\
      Host: host1.edu.co \\
      Date: Tue, 27 Mar 2007 19:36:42 +0000 \\
      Authorization: 13010f3e-3f46-4889-b989-592ce8fb30c6: \\
      Yzk5MWY1MmVkMC00ZGFiLTtNmQ5ZjcyMmRjODQ1 } & 
\multicolumn{1}{m{8cm}|}{
      \raggedright GET\textbackslash{}n \\
      host1.edu.co\textbackslash{}n \\
      \textbackslash{}n \\
      \textbackslash{}n \\
      Tue, 27 Mar 2007 19:36:42 +0000\textbackslash{}n \\ 
      /api/virtshell/hosts/5713b48a-8d73-11e5-8994-feff819cdc9f} \tabularnewline \hline
\end{tabular}
\normalsize

\subsubsection{Ejemplo de un objeto con DELETE}

Este ejemplo remueve un usuario.\\
\\
\vspace{1cm}
\scriptsize
\begin{tabular}{|l|l|} \hline
\textbf{Request} & \textbf{StringToSign} \\ \hline
\multicolumn{1}{|m{7.4cm}|}%
{\raggedright DELETE /api/virtshell/users/5fd13cc8-8d73-11e5-8994-feff819cdc9f HTTP/1.1 \\
 Host: host1.edu.co \\
 Date: Tue, 27 Mar 2007 21:20:27 +0000 \\
 Authorization: 13010f3e-3f46-4889-b989-592ce8fb30c6: Yzk5MWY1MmVkMC00ZGFiLTtNmQ5ZjcyMmRjODQ1 } & \multicolumn{1}{m{8cm}|}%
{\raggedright DELETE\textbackslash{}n \\
 host1.edu.co\textbackslash{}n \\
 \textbackslash{}n \\
 \textbackslash{}n \\
 Tue, 27 Mar 2007 21:20:27 +0000\textbackslash{}n \\ /api/virtshell/users/5fd13cc8-8d73-11e5-8994-feff819cdc9f} \tabularnewline \hline
\end{tabular}

\normalsize
\section{Protecci'on}

VirtShell es un framework multi-usuario que ofrece protecci'on a recursos basado en los conceptos de permisos de Unix. El mecanismo de protecci'on proporciona un acceso controlado limitando los tipos de acceso a los diferentes recursos del sistema. Al igual que en Unix, la t'ecnica usada para ofrecer protecci'on a los recursos consiste en hacer que el acceso dependa de la identidad del usuario. El esquema usado es usar una lista de accesso condensada por cada recurso.\\
\\
VirtShell clasifica los usuarios en tres grupos, en lo que respecta a cada recurso:\\
\\
\begin{itemize}
\item \textbf{Propietario}: El usuario que cre'o el recurso ser'a su propietario
\item \textbf{Grupo}: Un conjunto de usuarios que est'an compartiendo el archivo y necesitan un acceso similar al mismo es un grupo, o grupo de trabajo.
\item \textbf{Universo}: Todos los dem'as usuarios del sistema constituyen el universo.
\end{itemize}


% VirtShell is a multi-user framework that is based on the Unix permissions concepts to provide security.

% VirtShell provides mechanisms to control access by  limiting the types of
% resource access that can be made. Access is permitted or denied depending on
% several factors, one of which is the type of access requested. Several different
% types of operations may be controlled:

% Read. Read from the resouce.
% Write. Write or rewrite of resoures.
% Execute. Load the resource into host and execute it.

% Here is a quick breakdown of the access that the three basic permission types grant a user.

% Read
% ----
% Read permission allows a user to view the contents of any resource in VirtShell.

% Write
% -----
% Write permission allows a user to create, modify and delete whatever resources.

% Execute
% -------
% Execute permission allows a user to execute virtual machines or containers, for example: start, stop, pause, snapshot. (the user must also have read permission). 

